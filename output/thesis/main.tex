\documentclass[12pt,a4paper]{article}

% Essential packages
\usepackage[utf8]{inputenc}
\usepackage[T1]{fontenc}
\usepackage{mathptmx} % Times New Roman font - University requirement
\usepackage[margin=2.5cm]{geometry} % 2.5cm margins - University requirement
\usepackage{graphicx}
\usepackage{booktabs}
\usepackage{float}
\usepackage{natbib}
\usepackage{xcolor}
\usepackage{hyperref}
\usepackage{amsmath} 
\usepackage{textgreek}
\usepackage{ragged2e}
\usepackage{setspace}
\usepackage{indentfirst} % For paragraph indentation
\usepackage{titlesec}     % Section formatting
\usepackage{fancyhdr}     % Header and footer
\usepackage{pdfpages}  % To include PDF pages
\usepackage{tikz}
\usepackage{pgfplots}
\pgfplotsset{compat=1.18}
\usepackage{caption}
\usepackage{threeparttable}
\usepackage{subcaption}
\usepackage{adjustbox}
\usepackage{tabularx}
\usepackage[square]{natbib}  % veya [square] parantez için
\urlstyle{same}

\sloppy % uzun satır taşmasını engeller\usepackage{hyperref}

% Set paragraph indentation to 1cm - University requirement
\setlength{\parindent}{1cm}

% Set line spacing to 1.5 - University requirement  
\onehalfspacing

% Chapter/Section formatting - University requirements
\titleformat{\section}
  {\centering\normalfont\large\bfseries}
  {\Roman{section}.}
  {1em}
  {\MakeUppercase}

% Start each section on a new page - University requirement
\newcommand{\sectionbreak}{\clearpage}

% Subsection formatting
\titleformat{\subsection}
  {\normalfont\normalsize\bfseries}
  {\arabic{section}.\arabic{subsection}.}
  {1em}
  {}

% Subsubsection formatting  
\titleformat{\subsubsection}
  {\normalfont\normalsize\bfseries}
  {\arabic{section}.\arabic{subsection}.\arabic{subsubsection}.}
  {1em}
  {}

% Add proper spacing before and after sections - University requirement
\titlespacing*{\section}{0pt}{1\baselineskip}{1.5\baselineskip}
\titlespacing*{\subsection}{0pt}{1.5\baselineskip}{0.5\baselineskip}
\titlespacing*{\subsubsection}{0pt}{1.5\baselineskip}{0.5\baselineskip}

% Page numbering configuration
\pagestyle{fancy}
\fancyhf{} % Clear all headers and footers first
\fancyfoot[C]{\thepage} % Center page number in footer
\renewcommand{\headrulewidth}{0pt} % No header line
\renewcommand{\footrulewidth}{0pt} % No footer line

% Special style for first pages of chapters/sections
\fancypagestyle{plain}{%
  \fancyhf{}% Clear header/footer
  \fancyfoot[C]{\thepage}% Page number in center of footer
  \renewcommand{\headrulewidth}{0pt}% No header rule
  \renewcommand{\footrulewidth}{0pt}% No footer rule
}

% Configure hyperref
\hypersetup{
    colorlinks=true,
    linkcolor=black,
    filecolor=magenta,
    urlcolor=black,
    citecolor=black
}

% Remove parskip package conflict and use proper paragraph formatting
% \usepackage{parskip} - Remove this line

% Title Information
\title{\Large\textbf{Gender Pay Gap Determinants in European Labor Markets:\\A Sectoral and Occupational Analysis}}
\author{}
\date{}

\begin{document}


% Title Page - University of Warsaw Format
\begin{titlepage}
    \thispagestyle{empty}
    \centering
    \vspace*{1cm}

    {\large University of Warsaw}\\[0.5em]
    {\large Faculty of Economic Sciences}\\[4em]
    
    \vspace*{2cm}
    
    {\large Mehmet Tiryaki}\\[0.5em]
    {Album No: 437988}\\[4em]

    {\Large\textbf{Gender Pay Gap Determinants in European}}\\[0.5em]
    {\Large\textbf{Labor Markets: A Sectoral and}}\\[0.5em]
    {\Large\textbf{Occupational Analysis}}\\[2em]
    
    \vspace*{2cm}

    {Magister (master's) degree thesis}\\[0.5em]
    {\textit{Field of the study: Data Science and Business Analytics}}\\[2em]
    
    \vspace*{1.5cm}
    
\begin{raggedleft}
    {The thesis written under the supervision of}\\[0.2em]
    {Dr. Eva Siermińska}\\[0.2em]  
    {from Faculty of Economic Sciences}\\[0.2em]
    {WNE UW}\\[2em]
\end{raggedleft}

    \vspace*{\fill}
    
    {Warsaw, September 2025}
    
\end{titlepage}



\newpage
\thispagestyle{empty}
\includepdf[pages=1]{declaration.pdf}


% Summary Page - University Format
\newpage
\begin{center}
\thispagestyle{empty}

\noindent\textbf{Summary}
\vspace{0.5cm}

\begin{justify}
This thesis investigates the factors that determine gender wage gaps in European labor markets. Using advanced panel data methods, the study analyzes harmonized data from the Structure of Earnings Survey covering 40 countries from 2010 to 2022. The dataset includes 14,430 cleaned observations from 5,176 unique country-sector-occupation combinations across 18 economic sectors.

The findings show significant differences between sectors. Compared to the Services sector, Industry has a 4.392 percentage-point higher wage gap, while the Public sector has a 2.706 percentage-point lower gap. This results in a 7.10 percentage-point difference between Industry and Public sectors.

The analysis of occupational roles reveals that high-skill workers face a 2.936 percentage-point higher gap, and managers have a 4.663 percentage-point higher gap, suggesting the presence of a "glass ceiling." The study also finds complex interactions between sectors and occupations. For instance, high-skill workers in Industry have a 4.211 percentage-point narrower gap than expected, and managers in the Public sector experience a 1.678 percentage-point lower gap. Over time, the wage gap has consistently narrowed, declining by 2.173 percentage points from 2010 to 2022.

These results highlight that wage discrimination is driven by complex interactions between sectoral structures and occupational hierarchies. This underscores the need for multifaceted policy solutions.
\end{justify}


\vspace{1.5cm}

\noindent\textbf{Key words}

\vspace{0.5cm}

\noindent\emph{gender pay gap, panel data analysis, sectoral segregation, occupational hierarchy, European labor markets, wage discrimination}

\vspace{1cm}

\noindent\textbf{Field of the thesis (codes according to the Erasmus program)}

\vspace{0.5cm}

\noindent Economics (0311)

\vspace{1cm}

\noindent\textbf{Thematic classification}

\vspace{0.5cm}

\noindent J31, J71, J45, C23

\vspace{1cm}

\noindent\textbf{The title of the thesis in Polish}

\vspace{0.5cm}

\noindent\emph{Determinanty luki płacowej ze względu na płeć na europejskich rynkach pracy: analiza sektorowa i zawodowa}
\end{center}

% Table of Contents
\newpage
\thispagestyle{empty}

\tableofcontents

% Start main content with Arabic page numbering
\clearpage
\setcounter{page}{1}
\pagenumbering{arabic}

\section{INTRODUCTION}

In Europe, women earn about 12\% less than men for similar work, even with many laws designed to ensure equality \citep{eurostat2023}. This ongoing wage gap impacts not only individual incomes but also household financial security, retirement savings, and the overall economic productivity of European Union member states.

Why this wage gap continues to exist is a key question for labor economists. While overall statistics show this inequality, they often hide important differences across various economic sectors and job levels. To create effective policies that tackle the root causes of workplace inequality, it is crucial to understand these differences.

This study investigates the causes of gender pay gaps in European labor markets by analyzing detailed panel data from the Structure of Earnings Survey (2010-2022). It explores how differences in economic sectors and job types contribute to wage inequality, offering insights for policies aimed at improving equality in Europe.


This thesis addresses three specific research questions:
\begin{enumerate}
\item \textbf{RQ1:} How do sectoral institutional arrangements affect gender wage gaps across 18 detailed NACE Rev. 2 economic sectors, and do certain sectors exhibit negative gaps where women out-earn men?
\item \textbf{RQ2:} To what extent does occupational hierarchy (high-skill positions and managerial roles) moderate sectoral wage discrimination patterns?
\item \textbf{RQ3:} Is there evidence of beta convergence in gender wage gaps across European countries, where countries with higher initial gaps experience faster subsequent reductions?
\end{enumerate}

The study includes data from 40 European countries, covering 18 economic sectors (NACE Rev. 2 classification) and 9 occupational categories (ISCO-08 classification). This approach allows for an examination of how different institutional sectors and workplace hierarchies influence gender pay differences. The 2010-2022 timeline is important because it covers the period after the financial crisis, includes changes in gender equality laws, and shows the effects of the COVID-19 pandemic on labor markets.


\subsection{Thesis Structure}

This thesis is organized into seven sections. Section I introduces the research problem, its significance, and the scope of the study, which covers 40 European countries and 18 economic sectors from 2010 to 2022.

Section II provides a literature review covering nine key themes, including theories of wage discrimination, sectoral and occupational differences, and the impact of the COVID-19 pandemic on labor markets. This section also identifies the research gaps that this study aims to fill.

Section III presents the three main research questions and hypotheses that guide the analysis, focusing on sectoral differences, occupational roles, and institutional factors.

Section IV describes the data from the Structure of Earnings Survey. It explains how variables were created, how sectors and occupations were classified, and the data cleaning process, ensuring the study can be replicated.

Section V outlines the research methodology, including the panel data models, how estimators were chosen, and the methods used to address potential statistical issues. It also details the models used to test the hypotheses.

Section VI presents the results, including descriptive statistics, regression analyses, and robustness checks.

Finally, Section VII summarizes the findings, discusses their implications for theory and policy, acknowledges the study's limitations, and suggests areas for future research. The findings from this research are intended to support evidence-based policies aimed at reducing labor market inequality and promoting pay equity across Europe.

\newpage
\section{LITERATURE REVIEW}

The gender wage gap is a continuing issue in the labor markets of developed countries, with differences seen across sectors, occupations, and institutions. This review brings together theories and evidence about the gender pay gap, with a focus on sectoral and occupational factors in European labor markets. It covers nine topics: the theoretical basis of wage discrimination, differences between sectors, occupational segregation, institutional factors, intersectional perspectives, policy actions, new methods in panel data analysis, evidence from labor markets during the pandemic, and existing research gaps.

\subsection{Theoretical Foundations of Gender Wage Discrimination}

Three main economic theories help explain why gender wage gaps continue to exist. Human capital theory \citep{becker1964, mincer1974} suggests that wage differences come from differences in education, training, and work experience. Women may earn less because they often have fewer chances to gain work experience and specific job skills, usually because of family duties \citep{polachek1981}. However, a large part of the wage gap cannot be explained even when education and experience are taken into account \citep{blau2017}.

Statistical discrimination theory \citep{phelps1972, arrow1973} describes how employers use group averages to make decisions about individuals when they have limited information. For example, if employers think women are more likely to leave their jobs or work fewer hours due to family responsibilities, they might offer lower wages to all women. This can discourage women from investing in new skills, which keeps the gap going \citep{coate1993}. Studies show this type of discrimination is stronger in fields dominated by men, where employers have less experience working with women \citep{card2016}.

Taste-based discrimination \citep{becker1957} links pay gaps to the personal biases of employers, coworkers, or customers. Prejudiced employers may be willing to make less profit to avoid hiring or promoting women, which leads to lower wages for women in markets with less competition. Even though competition in the market should get rid of this type of discrimination, it still exists in many areas \citep{charles2008}. The combination of taste-based and statistical discrimination leads to different patterns of wage gaps in different workplaces \citep{altonji2001}.

More recent ideas from behavioral economics provide other explanations. \citep{bohren2019} introduced the concept of "inaccurate statistical discrimination," where employers have wrong beliefs about group differences. This helps explain why wage gaps can continue even in competitive markets. \citep{bordalo2019} used salience theory to show that stereotypes can affect wages differently depending on the situation. These new theories help explain why discrimination continues to exist despite market competition and why it varies in different job settings.

\subsection{Empirical Evidence on Sectoral Gender Pay Gap Variations}

Empirical studies have shown that gender wage gaps vary significantly across different economic sectors. These differences are influenced by factors such as institutional structures, competitive pressures, and the gender makeup of the workforce. For example, sectors like manufacturing and construction tend to have larger gender wage gaps, often between 15-25\%, compared to service sectors where gaps are typically around 10-15\% in European countries \citep{eurostat2023}. This variation remains even after accounting for individual characteristics, which suggests that factors specific to each industry are important in determining wages.

In contrast to the private sector, the public sector generally has smaller gender wage gaps. A study by \citep{arulampalam2007} found that public-sector wage gaps in EU countries were about 10-12\%, while in the private sector, they were between 15-20\%. This is because the public sector often has more standardized pay scales, formal promotion processes, and stronger enforcement of equal pay laws, all of which contribute to greater gender equality \citep{rubery2005}. However, recent budget cuts and reforms in the public sector may be weakening these advantages in some parts of Europe \citep{grimshaw2012}.

The financial and professional services sectors show a mixed picture. While many women in these fields are highly educated, there are still large gender wage gaps, especially at senior levels. \citep{bertrand2010} found that in the financial services industry, the gender wage gap can reach 30-40\% at the executive level, even when men and women have similar qualifications. This "glass ceiling" effect is often seen in sectors with highly competitive promotion systems and a culture of long working hours \citep{cha2014}. It is possible that new technologies and changing work practices could be altering these patterns, but more research is needed to understand the full impact \citep{cortes2021}.

Recent studies on the technology sector have also revealed unique patterns of gender wage inequality. The OECD found that while wage gaps for entry-level STEM jobs in Europe are relatively small (5-8\%), they grow to 20-25\% after 10 years of experience. This is often because men have better access to important projects and informal mentoring \citep{oecd2023}. Additionally, research by \citep{preston2022} shows that flexible work options, such as remote work, can help reduce the gender wage gap by 3-5 percentage points, as they lessen the penalties for taking on family responsibilities.

The healthcare sector also has its own dynamics. Although many women work in this field, they are often in lower-paying roles, a situation known as vertical segregation. The World Health Organization found that gender wage gaps among medical professionals in 15 EU countries vary by specialty \citep{who2022}. For example, surgical specialties have gaps of 25-30\%, while in primary care, the gaps are closer to 10-15\%. This is often due to differences in access to private practice and biases in specialty choices. The International Labour Organization also found that even in nursing, a female-dominated profession, male nurses tend to earn 5-8\% more, especially in technical roles and management positions \citep{ilo2022}.

\subsection{Occupational Hierarchy and Gender Segregation Mechanisms}

Occupational segregation is a key reason why gender wage gaps continue to exist. This happens in two ways: horizontal segregation, where women are concentrated in lower-paying jobs and sectors, and vertical segregation, where women are underrepresented in senior roles. \citep{levanon2009} found that when more women enter an occupation, the wages for everyone in that job tend to go down. This suggests that the work women do is often valued less, which contributes to the wage gap beyond just individual discrimination.

The connection between the skills required for a job and the gender wage gap is not straightforward. In most European countries, women have reached or even surpassed men in terms of education, but this has not always translated into better job opportunities. \citep{weinberger2011} discovered that gender wage gaps are smallest in mid-level jobs that require specific technical skills. In contrast, the gaps are largest in both low-skill manual jobs and high-skill managerial positions. This "U-shaped" pattern suggests that different factors are at play at different skill levels.

More recent research has looked at how the specific tasks involved in a job can create wage differences. \citep{cortes2021} showed that jobs requiring strong social skills have seen wages increase, which has benefited women. On the other hand, jobs that require physical strength or are highly competitive have maintained larger gender wage gaps. The automation of routine tasks is also affecting female-dominated clerical jobs, which could make wage inequality worse in the future \citep{brussevich2018}. Understanding these trends is important for predicting how technology will continue to shape gender wage patterns.

A study by \citep{koumenta2020} provided new insights into how occupational licensing affects gender wage gaps in Europe. They analyzed data from 28 EU countries and found that licensed occupations have wage gaps that are 4-6 percentage points lower than in similar unlicensed occupations. This is because licensed professions have standardized qualifications and clear paths for advancement, which reduces the chances for discrimination. However, a later study by \citep{koumenta2022} found that women often face more barriers to entering these licensed occupations, which complicates the picture.

The COVID-19 pandemic has also had a significant impact on occupations, with different effects for men and women. \citep{adamsprassl2023} studied employment and wage trends in Europe from 2020 to 2022 and found that female-dominated service jobs saw larger job losses but smaller wage reductions compared to male-dominated manufacturing jobs. This was partly because many low-wage female workers left the workforce. In Spain, \citep{farre2022} found that the shift to remote work during the pandemic reduced gender wage gaps by 2-3 percentage points in jobs that could be done from home, but it widened the gaps in jobs that could not.


\subsection{European Institutional Context and Cross-National Variation}

European labor markets provide a diverse setting for studying gender wage gaps due to significant differences in welfare systems, family policies, and equal pay laws across countries. For instance, the Nordic countries, with their generous parental leave, subsidized childcare, and large public sectors, have relatively low wage gaps of 5-10\%. However, they still have high levels of occupational segregation, where men and women tend to work in different types of jobs \citep{mandel2005}. In contrast, Continental European countries with more traditional welfare systems have moderate gaps of 15-20\%, while countries with more market-oriented economies, like the UK, have larger gaps \citep{christofides2013}.

Family policies play a major role in shaping gender wage patterns by influencing women's participation in the workforce and employers' perceptions. \citep{budig2016} found that government-funded childcare can reduce the wage penalties that mothers often face, while long parental leaves can sometimes lead to more discrimination. The EU's Work-Life Balance Directive, introduced in 2019, aims to address this by requiring paternal leave and other caregiving support, which could help change traditional gender roles \citep{europeancommission2019}. However, how this directive is being implemented varies from country to country, reflecting ongoing cultural and institutional differences.

Labor market institutions, such as collective bargaining and minimum wage laws, also play a role in reducing gender wage gaps. \citep{blau2003} found that a decline in union membership has contributed to a rise in wage inequality, with different effects for men and women. European countries with centralized wage-setting systems tend to have smaller gender wage gaps, although this can vary by sector \citep{visser2016}. However, recent shifts toward more flexible employment and decentralized bargaining could weaken these positive effects \citep{garnero2020}.

Recent studies have also highlighted differences between European regions. \citep{perugini2019} found that in Central and Eastern European countries, the transition from socialism has had a lasting impact on gender wage gaps. Despite economic growth, these countries have wage gaps that are 5-10 percentage points larger than in Western Europe, which is often attributed to weaker enforcement of equality laws and traditional gender norms. In Southern European countries, \citep{olivetti2008} noted that informal labor markets and family-owned businesses can create hidden gender inequalities that are not captured in official wage data.

The European Green Deal and other sustainability policies are also introducing new factors into the analysis of gender wages. \citep{eige2023} found that the growth of green jobs, such as in the renewable energy sector, has disproportionately benefited men due to the technical skills required. They project that without specific actions, the green transition could widen gender wage gaps by 2-4 percentage points by 2030. On a more positive note, \citep{mergaert2021} found that regions that include gender equality goals in their economic development programs have wage gaps that are 3-5 percentage points smaller.

\subsection{Intersectional Perspectives on Gender Wage Gaps}

Recent research shows that gender combines with other aspects of a person's identity, such as race, ethnicity, and immigrant status, to create complex disadvantages in the labor market \citep{acker2012}. This "intersectional" approach helps us understand that general statistics on the gender wage gap can hide significant differences among various groups of women in Europe.

For example, a person's migration status has a major impact on the gender wage gap. A study by \citep{adsera2020} on immigrant women in six EU countries found that women born in another country face a "double penalty," earning 25-30\% less on average than men born in the country. This is due to a combination of gender and migration-related challenges, such as difficulty getting foreign credentials recognized, having smaller social networks, and often working in the informal economy. \citep{kogan2021} found that even second-generation immigrant women still face a wage gap of 10-15\%, which suggests that the disadvantages can persist across generations.

Age is another factor that interacts with gender to affect career paths. \citep{manning2022} analyzed wage patterns across Europe and found that the gender wage gap grows over time, starting at around 5\% when people first enter the workforce and increasing to 20-25\% by the time they are 50. This is due to the accumulated effects of career breaks, different promotion rates, and generational differences in education. A study on the German labor market by \citep{boll2022} showed that pension reforms that encourage people to work longer can put older women at a disadvantage, as they may face both age and gender discrimination.

Education level also plays a role in shaping the gender wage gap. \citep{triventi2023} looked at the value of higher education in 20 European countries and found that while women with a university degree have smaller wage gaps (8-12\%) than less-educated women (15-20\%), the field of study they choose can still lead to inequality. For example, women in STEM fields often face larger wage gaps within their field, while female-dominated fields tend to have lower wages overall. \citep{bobbittzeher2022} also found that vocational training programs often steer men and women into different types of jobs with very different pay levels.

\subsection{Policy Interventions and Evaluation Evidence}

The success of equal pay laws and anti-discrimination policies can vary greatly depending on the institutional context and how they are implemented. For example, a study by \citep{bennedsen2022} on mandatory gender wage gap reporting in Denmark found that these transparency rules reduced the gender wage gap by 2-3 percentage points within two years. However, the study also noted that some companies responded by reclassifying jobs or increasing performance-based pay, which partially counteracted the positive effects. \citep{duchini2023} found that similar policies had a larger impact in countries with stronger enforcement and public disclosure requirements.

In addition to pay transparency, quotas for women on corporate boards can also have a positive impact on gender wage equality. \citep{maida2022} studied the implementation of board gender quotas in various European countries and found that companies subject to these quotas reduced the gender wage gap for executives by 5-8 percentage points. The study suggested that this was due to the "role model" effect and changes in company culture. However, \citep{ahern2012} found that these effects were limited to executives and did not extend to other employees, which suggests that top-down approaches may not be enough to address broader inequalities.

Family policy reforms have also provided valuable insights into how discrimination works. \citep{kleven2021} analyzed the introduction of mandatory paternity leave in several European countries and found that policies encouraging fathers to take on more caregiving responsibilities reduced the gender wage gap by 2-4 percentage points over five years. This was because employers began to change their expectations about women's likelihood of taking career breaks, and it helped to normalize the idea of fathers as caregivers. Similarly, \citep{farre2019} found that when Spain extended paternity leave from 2 to 16 weeks, there was an immediate reduction in hiring discrimination against young women.

Finally, minimum wage policies have been shown to have a positive effect on gender equality, as women are more likely to work in low-wage jobs. \citep{caliendo2022} studied the introduction of a statutory minimum wage in Germany in 2015 and found that it reduced the gender wage gap at the lower end of the pay scale by 3-5 percentage points. However, the study also noted that some women were moved to part-time jobs with fewer hours as a result. A meta-analysis by \citep{garnero2014} of minimum wage policies across the EU found that sector-specific minimum wage policies were more effective at promoting gender equality than national minimum wage policies.

\subsection{Methodological Advances in Panel Data Gender Pay Gap Analysis}

The way economists study gender wage gaps has changed a lot over time, thanks to new methods for analyzing panel data and breaking down the data. The traditional Oaxaca-Blinder decomposition method, which is useful for looking at data at a single point in time, has two main problems. First, it doesn't account for "unobserved heterogeneity," which refers to things like a worker's motivation or a company's culture that can affect wages but are hard to measure. Second, it doesn't address "selection effects," which happen when men and women make different choices about participating in the workforce. This means that the wage gaps we see might not reflect the true picture for the entire population. \citep{fortin2011} has provided a guide to newer decomposition methods that can help with these issues.

Panel data methods are an improvement because they can control for unobserved factors that don't change over time. Fixed effects models can remove bias from these constant characteristics, but they can't identify the effects of things that don't change, like gender.

\citep{kunze2008} combined fixed effects with decomposition to track wage gaps for the same individuals over time. Random effects models, which make stronger assumptions, can be used to estimate the effects of gender and how selection patterns change over time \citep{wooldridge2010}.

More recent methodological developments have focused on addressing selection bias and other issues in wage gap estimation. \citep{mulligan2008} used selection correction methods to account for people's decisions to participate in the labor force and found that without these corrections, the true wage gaps are often underestimated. Machine learning methods are also providing new ways to model the complex interactions between individual characteristics and discrimination, although there are still challenges in interpreting the results \citep{kline2021}.

\citep{chernozhukov2018} were the first to use a technique called "double machine learning" to study the gender wage gap. Using data from Germany, they found that simpler linear models underestimated wage gaps by 3-5 percentage points because they didn't account for the complex interactions between occupation, industry, and experience. \citep{firpo2009} expanded on this by looking at the entire wage distribution and found that the errors were largest at the highest and lowest ends of the pay scale, where gender gaps are often the widest.

Synthetic control methods are another useful tool for evaluating the impact of policies when there are only a few cases to study. For example, \citep{arkhangelsky2021} used a method called "synthetic difference-in-differences" to analyze the effect of Iceland's equal pay certification requirement. They created a "synthetic" control group from other Nordic countries and found that the policy reduced gender wage gaps by 4-6 percentage points, especially in large private companies. \citep{gobillon2008} have also developed spatial panel methods that can account for how wage-setting in one region can affect neighboring regions, and they found that local labor market competition can significantly reduce gender wage gaps.



\subsection{Recent Evidence from Pandemic-Era Labor Markets}

The COVID-19 pandemic caused major disruptions to labor markets, with different impacts on men and women. \citep{alon2021} studied employment and wage trends in Europe from 2020 to 2022 and found that at the beginning of the pandemic, women's employment fell more sharply than men's, a pattern that has been called a "she-cession." However, for those who remained employed, the wage gap actually narrowed by 2-3 percentage points. This was due to a combination of factors, including shifts in the types of jobs available and the increased adoption of flexible work arrangements. A study by \citep{adamsprassl2020} in the UK found that remote work was particularly beneficial for mothers, as it reduced the wage penalties often associated with needing a flexible work schedule.

The impact of the pandemic also varied by sector, which helps to explain some of the underlying causes of the gender wage gap. \citep{albanesi2021} found that women who worked in essential sectors, such as healthcare and education, saw their wages increase relative to men. In contrast, women who worked in service sectors that were considered non-essential experienced more lasting negative effects on their wages. \citep{bluedorn2021} also found that government support programs, such as short-time work schemes, were more effective at protecting women's jobs and wages than simply expanding unemployment benefits.

The long-term effects of the pandemic on the labor market are still not fully clear. While remote work could be a positive development for gender equality by reducing the penalties for flexible work, there is also a risk of "proximity bias," where employees who work from home—many of whom are women with caregiving responsibilities—are at a disadvantage. Therefore, it is likely that the changes brought about by the pandemic will modify the gender wage gap rather than eliminate it entirely.

\subsection{Research Gaps and Study Contribution}

Despite a great deal of research on gender wage gaps, there are still some important areas that have not been fully explored. For example, most studies tend to look at the effects of either sector or occupation on wage inequality, but they don't often examine how these two factors interact. The way that men and women are distributed across different sectors and occupations creates complex patterns that need to be analyzed together. Additionally, there has not been enough research on how wage gaps change over time, especially in response to new technologies and major events like the COVID-19 pandemic \citep{brussevich2018}.

There is also a need for more cross-national research that uses consistent data from multiple countries. Most studies focus on a single country or compare just two countries, which limits our ability to understand how different institutional factors, such as laws and policies, affect discrimination. The different ways that EU directives have been implemented and the different ways that countries have recovered from economic crises provide a good opportunity to study the effects of these institutional factors on gender wage equality \citep{christofides2013}.

Finally, there are still methodological challenges in addressing issues like selection bias and unobserved heterogeneity. While panel data methods can account for factors that don't change over time, they don't fully address the fact that people may choose to enter or leave the workforce or change occupations based on their experiences. Newer methods from machine learning and causal inference could be helpful, but they need to be carefully adapted for the study of gender wage gaps \citep{kline2021}. There is also ongoing debate about how to interpret the "unexplained" portion of the wage gap, as it is difficult to distinguish between unmeasured productivity differences and actual discrimination \citep{fortin2011}.

This literature review highlights three main gaps in the existing research. First, most studies look at sectoral or occupational effects separately, rather than analyzing how they interact to create wage inequality. Second, there is a lack of comparative research using harmonized data from multiple European countries, which limits our understanding of how institutional factors influence discrimination. Third, many methodological approaches do not adequately address the issues of selection bias and unobserved heterogeneity while also allowing for cross-national comparisons.

This study aims to address these gaps in three ways. First, it provides a comprehensive analysis of the factors that determine gender wages using panel data from 40 European countries from 2010 to 2022, covering 18 sectors and 9 occupational categories. Second, it examines how sectoral and occupational effects interact, using harmonized data from the Structure of Earnings Survey. Third, it uses a robust econometric framework that controls for unobserved heterogeneity and selection effects while also allowing for a systematic comparison of institutional influences across countries. By taking this integrated approach, this study seeks to advance our understanding of how sectoral and occupational structures interact to perpetuate gender wage inequalities in different European institutional contexts.

\section{RESEARCH QUESTIONS}

This study addresses three fundamental research questions that emerge from identified gaps in the gender pay gap literature, particularly regarding the intersection of sectoral employment patterns, occupational hierarchies, and cross-national institutional variations in European contexts.

\subsection{Research Question 1: Sectoral Heterogeneity in Gender Pay Gaps}

\textbf{RQ1:} How do gender pay gaps vary systematically across 18 detailed NACE Rev. 2 economic sectors in European labor markets from 2010 to 2022, and do certain sectors exhibit negative gaps where women out-earn men?

This research question addresses the limited systematic investigation of sectoral variation in gender wage research. While existing literature documents aggregate pay gaps, it typically obscures substantial heterogeneity in how gender differentials manifest across economic sectors with distinct institutional structures, gender composition, and wage-setting mechanisms \citep{rubery2005}. The question extends beyond simple sectoral comparisons to examine whether feminized sectors with compressed wage structures, such as hospitality (hotels and food services), education (primary and secondary schools), and public administration (government agencies and municipal services), demonstrate not merely smaller gaps but actual reversals, in which women systematically out-earn men.

\textbf{Hypothesis 1:} Industry sectors exhibit larger gender pay gaps than Public Sector employment, and feminized service sectors show substantial proportions of negative gaps where women out-earn men.

Specifically, this hypothesis predicts that traditional industries (Manufacturing, Mining, Finance) demonstrate significantly positive coefficients ($\beta_{Industry} > 0$) due to male-dominated organizational cultures, tournament-style promotion systems, and weaker equality enforcement. Public Sector employment shows significantly negative coefficients ($\beta_{Public} < 0$) in panel regression specifications due to formalized wage structures and stronger enforcement. Furthermore, feminized sectors (hospitality, education, public administration) show compressed gaps, with more than 15\% of country-sector-occupation-year observations exhibiting negative differentials where women out-earn men.

\subsection{Research Question 2: Occupational Hierarchy Moderation}

\textbf{RQ2:} To what extent do occupational skill levels and managerial hierarchies moderate sectoral gender pay gap differentials within European labor markets?

This question examines how sectoral institutions and occupational hierarchies interact, rather than treating them separately as prior research often does. The glass ceiling literature identifies larger gaps at higher organizational levels \citep{arulampalam2007}, whereas human capital theory predicts convergence based on skills \citep{becker1964}. Yet it is unclear whether occupational penalties are consistent across sectors or depend on institutional environments. This question tests whether discrimination mechanisms differ between male-dominated industries and egalitarian public sectors when high-skill workers and managers are involved.

\textbf{Hypothesis 2:} High-skill and managerial positions exhibit larger gender pay gaps than lower-skill occupations, but these occupational effects vary systematically across sectors.

Specifically, this hypothesis predicts that high-skill and managerial positions demonstrate larger gaps due to discretionary compensation mechanisms and glass ceiling effects ($\beta_{HighSkill} > 0$, $\beta_{Managerial} > 0$). Critically, sector-occupation interactions reveal that occupational penalties depend on institutional context: the Industry penalty is partially offset in high-skill positions ($\beta_{Industry \times HighSkill} < 0$), while Public sector advantages are reversed for high-skill workers ($\beta_{Public \times HighSkill} > 0$) but strengthened for managers ($\beta_{Public \times Managerial} < 0$), demonstrating that discrimination mechanisms operate conditionally rather than uniformly.

\subsection{Research Question 3: Institutional Determinants and Cross-National Convergence}

\textbf{RQ3:} Do gender pay gaps differ across European country groups (Nordic, Continental, Mediterranean, Eastern European, Liberal, Balkans), and is there evidence of beta convergence where countries with higher initial gaps experience faster subsequent reductions over time?

This question asks whether national institutional configurations—welfare state regimes, labor market structures, and family policies—fundamentally shape gender wage gaps. Welfare regime theory \citep{espingandersen1990} predicts that Nordic social-democratic systems achieve greater equality through comprehensive policies, while liberal market economies maintain larger gaps due to decentralized wage-setting. The question also considers whether EU equal-pay directives, policy learning, and competitive pressures help lagging countries catch up to frontier equality levels through dynamic convergence.

\textbf{Hypothesis 3:} Gender pay gaps vary systematically across welfare regime types, and countries with higher initial gaps experience faster convergence over time.

This hypothesis has two components. \textit{Part 1 (Cross-Sectional Variation):} Nordic and Eastern European countries achieve smaller gaps than Continental, Mediterranean, and Liberal countries through egalitarian institutions, strong public sectors, and collective bargaining coverage ($\beta_{Nordic} < 0$, $\beta_{Eastern} < 0$ relative to Continental reference; $\beta_{Liberal} > 0$). Country group × sector interactions reveal institutional moderation: the Nordic × Public Sector interaction is positive ($\beta_{Nordic \times Public} > 0$), indicating that Nordic countries achieve equality across both public and private sectors, eliminating the public sector premium observed in other welfare regimes. \textit{Part 2 (Temporal Convergence):} Countries with higher 2010 baseline gaps experience significantly faster convergence rates over 2010-2022 ($\beta_{Gap_{2010}} < 0$ in cross-sectional convergence regression), driven by EU equal pay directives and policy harmonization. Panel time fixed effects demonstrate consistent temporal decline ($\beta_{Year2014} < 0$, $\beta_{Year2018} < 0$, $\beta_{Year2022} < 0$), with accelerating convergence rates in later periods.

These research questions address the limited systematic investigation of sectoral-occupational interactions and cross-national institutional variations in gender wage research. The analysis uses harmonized European data, comprising 14,430 observations from 5,176 unique country-sector-occupation panels across 40 countries. The framework goes beyond aggregate gender pay gap measures and examines heterogeneity across 18 detailed sectors, nine occupational categories, and seven country groups with distinct welfare regimes. This approach enables robust comparative analysis of discrimination mechanisms in diverse European labor market contexts.

\section{DATA}

\subsection{Data Source and Coverage}

This study uses data from the European Union Structure of Earnings Survey (SES), a large-scale survey of employers that provides consistent data on earnings across European labor markets. The SES is the main source for comparing wages in Europe, as it uses a standardized sampling method and detailed job classifications \citep{eurostat2023}. The dataset includes information from 40 European countries from 2010 to 2022, covering all 27 EU member states and 13 other countries. The data is collected every four years (2010, 2014, 2018, and 2022), which creates a balanced panel with four time points. This allows for an analysis of gender wage gaps across different economic periods, including the recovery from the financial crisis and the COVID-19 pandemic.

The primary data source is Eurostat's Structure of Earnings Survey by NACE Rev.~2 sections and ISCO-08 major groups, 
using datasets: \texttt{earn\_ses10\_49} (2010, \nolinkurl{https://ec.europa.eu/eurostat/databrowser/view/earn_ses10_49/}), 
\texttt{earn\_ses14\_49} (2014, \nolinkurl{https://ec.europa.eu/eurostat/databrowser/view/earn_ses14_49/}), 
\texttt{earn\_ses18\_49} (2018, \nolinkurl{https://ec.europa.eu/eurostat/databrowser/view/earn_ses18_49/}), 
and \texttt{earn\_ses22\_49} (2022, \nolinkurl{https://ec.europa.eu/eurostat/databrowser/view/earn_ses22_49/}), 
all publicly accessible through Eurostat's online database.

The SES uses a two-stage sampling method. First, it selects businesses based on their size, and then it samples employees from within those businesses. This ensures that the data is representative of the entire economy and provides enough variation to support detailed analysis. The survey has a high response rate of over 80\% in most countries because participation is mandatory. This helps to reduce the risk of bias that can occur in surveys that rely on voluntary participation \citep{eurostat2022b}.

\subsection{Variable Construction and Definitions}

The analysis uses a wide range of variables to capture individual, occupational, and sectoral characteristics that are relevant to the gender wage gap. For the purpose of this study, the 18 detailed NACE Rev. 2 sectors are grouped into four broad categories: Industry, Construction, Services, and Public Sector. This is done based on their institutional characteristics and labor market structures. Similarly, the 9 ISCO-08 occupational categories are grouped into two main indicators: High-Skill occupations and Managerial positions. This allows for an examination of the "glass ceiling" effect and skill-based wage differences. Table \ref{tab:variable_definitions} provides a detailed explanation of how each variable is defined and constructed.

\begin{table}[H]
\centering
\caption{Variable Definitions and Construction Methodology for European Gender Pay Gap Analysis Based on Structure of Earnings Survey Data Spanning 40 Countries from 2010-2022}
\label{tab:variable_definitions}
\small
\begin{tabular}{p{3.5cm}p{7cm}p{3cm}}
\hline\hline
\textbf{Variable} & \textbf{Definition and Construction} & \textbf{Measurement} \\
\hline
\multicolumn{3}{l}{\textit{Dependent Variable}} \\
Gender Pay Gap & Percentage difference between male and female mean annual earnings within country-sector-occupation-year cells. Formula: $GPG = \frac{\text{Male Mean Earnings} - \text{Female Mean Earnings}}{\text{Male Mean Earnings}} \times 100$ & Continuous (\%) \\
\hline
\multicolumn{3}{l}{\textit{Sectoral Variables (18 NACE Rev. 2 Detailed Sectors)}} \\
Industry Sector & NACE sections B-E: Mining, Manufacturing, Electricity, Water - extractive and production industries & Binary indicator \\
Construction & NACE section F: Construction activities & Binary indicator \\
Services & NACE sections G-N, R-S: Trade, Transport, IT, Finance, Professional services, Arts, Other services & Binary indicator \\
Public Sector & NACE sections O-Q: Public administration, Education, Health (reference category) & Binary indicator \\
\multicolumn{3}{l}{\textit{Detailed NACE sectors: Mining (B), Manufacturing (C), Electricity (D), Water (E),}} \\
\multicolumn{3}{l}{\textit{Construction (F), Trade (G), Transport (H), Hospitality (I), IT (J), Finance (K),}} \\
\multicolumn{3}{l}{\textit{Real Estate (L), Professional (M), Admin Services (N), Public Admin (O),}} \\
\multicolumn{3}{l}{\textit{Education (P), Health (Q), Arts (R), Other Services (S)}} \\
\hline
\multicolumn{3}{l}{\textit{Occupational Variables (9 ISCO-08 Major Groups)}} \\
High-Skill Occupation & ISCO-08 major groups 1-3: Managers, Professionals, Technicians/Associates & Binary indicator \\
Managerial Position & ISCO-08 major group 1: Managers (Chief executives, senior officials, legislators) & Binary indicator \\
\multicolumn{3}{l}{\textit{Detailed ISCO-08 occupations: Managers (OC1), Professionals (OC2), Technicians (OC3),}} \\
\multicolumn{3}{l}{\textit{Clerical (OC4), Service Workers (OC5), Agriculture (OC6), Craft Workers (OC7),}} \\
\multicolumn{3}{l}{\textit{Plant/Machine Operators (OC8), Elementary Occupations (OC9)}} \\
\hline
\multicolumn{3}{l}{\textit{Panel Identifiers}} \\
Country & ISO 3166-1 alpha-2 country codes for 40 European nations & Fixed effects \\
Year & Survey years: 2010, 2014, 2018, 2022 & Time effects \\
Panel ID & Unique identifier: Country $\times$ Sector $\times$ Occupation & Panel unit \\
\hline\hline
\end{tabular}
\end{table}

{\small \textit{Source:} Own study, based on: European Union Structure of Earnings Survey, Eurostat, 2010-2022.}

The gender pay gap is calculated using the mean annual gross earnings from the Eurostat Structure of Earnings Survey. This includes regular pay, shift premiums, and performance-related pay, but excludes irregular bonuses to ensure consistency across different payment systems. All earnings are adjusted for purchasing power parity to allow for valid comparisons between countries.

It is important to note that the annual earnings data does not account for the number of hours worked. On average, women in Europe work fewer hours than men, often due to higher rates of part-time employment. This means that the observed gender pay gaps reflect both differences in hourly wages and differences in working time. However, this analysis is still valuable for three reasons. First, the study is interested in total earnings inequality, which is what affects household income and financial security. Second, part-time work is often not a voluntary choice, but rather a result of caregiving responsibilities and a lack of flexible full-time jobs. Therefore, it can be seen as a form of labor market discrimination. Third, the survey's sampling method helps to minimize, but not eliminate, bias from the fact that women are more likely to be in part-time positions. Future research using hourly wage data could provide additional insights by separating pure wage discrimination from the effects of working time.

\subsection{Sectoral and Occupational Classification Rationale}

The study uses a well-thought-out approach to group the NACE Rev. 2 sectors and ISCO-08 occupations in a way that helps to identify clear patterns in gender wage differences while also ensuring that the analysis is statistically sound. This section explains the reasoning behind this classification strategy.

\subsubsection{Sectoral Aggregation Framework}

The 18 detailed NACE Rev. 2 economic sectors are grouped into four broad categories. This is done based on their institutional characteristics, labor market structures, and gender equality mechanisms, which previous research has shown to have a systematic effect on wages.

The Industry Sector (NACE B-E) includes Mining, Manufacturing, Electricity/Gas, and Water Supply. These are traditional production industries that are dominated by a male workforce and have a high degree of job segregation by gender. They are capital-intensive and often require physical labor and technical skills that have historically been associated with men. These sectors often have powerful unions with seniority-based wage systems that can put women with interrupted careers at a disadvantage. They also tend to have a culture of long hours and physical presence, which can be difficult for those with family responsibilities. As a result, these sectors typically have the largest gender wage gaps in Europe and are important for studying discrimination in traditional industries.

The Construction Sector (NACE F) is treated as a separate category because of its unique labor market characteristics. Female participation is very low, which can lead to statistical discrimination and tokenism. The project-based nature of the work, with high mobility and informal networks, can also put women at a disadvantage. Subcontracting and self-employment are common, which can make it difficult to enforce equal pay laws. The physical demands and health hazards of the work are also often used to justify the exclusion of women. For these reasons, the construction sector requires a separate analysis.

The Services Sector (NACE G-N, R-S) includes a wide range of private service industries, such as Trade, Transport, Hospitality, IT, Finance, Real Estate, Professional and Administrative Services, Arts/Entertainment/Recreation, and Other Services. The gender composition of this sector is very diverse. While Transport and IT are male-dominated, women are more likely to work in hospitality and administrative support. Wages are often determined by the market, with performance-based pay and bonuses that can create opportunities for discrimination. Social and communication skills are becoming increasingly important, which can benefit women in customer-facing roles. There are also significant institutional differences within this sector, from the highly regulated finance industry to the more informal hospitality industry. Arts and Other Services are included because they have their own unique gender dynamics and a high representation of women. These sectors often have more compressed wage structures but may lack strong formal mechanisms for enforcing equality. The diversity within the Services sector makes it possible to identify sector-specific mechanisms of discrimination.

The Public Sector (NACE O-Q) includes Public Administration, Education, and Health. These sectors have unique institutional features that are expected to lead to smaller gender wage gaps. They often have formalized wage grids and transparent pay systems, which limit the discretion of managers. They are also subject to legal frameworks that enforce equal pay and public accountability. Female representation is high, especially in education and health, which can reduce tokenism and statistical discrimination. These sectors also have strong collective bargaining and union coverage, which helps to protect wage equality. Finally, they are more likely to have explicit diversity policies, as required by EU directives.

\textbf{Services} (NACE G-N, S) is used as the reference category against which all other sectors are compared. This broad category includes Trade, Transport, Hospitality, IT, Finance, Real Estate, Professional Services, Administrative Services, and Other Services. The Services sector was chosen as the reference category because it accounts for the largest share of employment (57.0\% of observations), has moderate gender wage gaps that fall between those of the Industry (higher gaps) and Public Sector (lower gaps), and provides a neutral baseline for interpreting the effects of other sectors. The diversity within the Services sector is explored in more detail in the 18-sector analysis (Section 6.5), but the main models use Services as a single reference group.

This four-category system allows for a balance between two analytical goals. First, it ensures that there are enough observations in each category to allow for a stable estimation of interaction effects and differences between country groups. This would not be possible with 18 separate sectors and multiple interactions. Second, it preserves the important institutional differences that affect gender wage determination. The main models use this four-sector scheme, while a supplementary analysis (Table \ref{tab:descriptive_stats}, Panel C) presents the full 18-sector breakdown to show that this grouping does not hide any critical information.

\subsubsection{Occupational Hierarchy Classification}

The ISCO-08 job classification system is used to create two binary indicators that reflect different aspects of occupational ranking. This is important for understanding the factors that contribute to gender pay differences.

High-Skill Occupation (ISCO 1-3): This variable includes the top three ISCO-08 groups: (1) Managers, (2) Professionals, and (3) Technicians and Associate Professionals. This grouping is based on human capital theory and the idea of credentialism. These jobs typically require a university degree or vocational training, involve a high level of cognitive skill and autonomy, and offer career advancement based on performance and managerial judgment. Compensation is often based on performance, bonuses, and subjective evaluations, which can create opportunities for discrimination.

This high-skill classification reflects the way the labor market is stratified, which can put women at a disadvantage compared to those in more routine jobs. Even though women in Europe have achieved similar or even higher levels of education than men, they still face barriers that prevent them from being fully rewarded for their qualifications. These barriers include: (1) statistical discrimination based on perceived career commitment and availability for long hours; (2) gendered networks and sponsorship that can affect promotions; (3) penalties for career interruptions, which disproportionately affect mothers in professional careers; and (4) masculine cultures in male-dominated professions (Weinberger, 2011). Therefore, the high-skill indicator helps to identify situations where investments in human capital should lead to equal pay, but discrimination still persists due to subjective evaluations.

Managerial Position (ISCO 1): Focusing on managers is in line with research on the "glass ceiling" and organizational hierarchies. The ISCO-08 major group 1 includes chief executives, senior officials, and managers in various fields. The factors that contribute to the gender wage gap in these roles include: (1) promotion processes that may be biased, (2) unclear and discretionary pay, (3) pressure for constant availability and geographic mobility, (4) male-dominated senior networks, and (5) negotiation dynamics that can put women at a disadvantage.

The managerial indicator shows that the gender wage gap is larger at the top of organizations, which supports the idea of a glass ceiling. It also allows for an examination of whether promotion systems and transparency have helped to reduce pay gaps in leadership roles. By measuring both high-skill and managerial roles, the analysis can distinguish between professional/technical jobs and leadership jobs. This helps to explore whether discrimination varies at different levels of the hierarchy in skilled jobs.

Reference Category (ISCO 4-9): The reference group includes clerical support workers (ISCO 4), service and sales workers (ISCO 5), craft and related trades workers (ISCO 7), plant and machine operators (ISCO 8), and those in elementary occupations (ISCO 9). Agricultural workers (ISCO 6) are a very small part of the sample because the survey focuses on companies with at least 10 employees and formal employment. These routine and manual occupations are characterized by: (1) lower educational requirements and a reliance on on-the-job training; (2) standardized wage systems with limited performance-based pay; (3) greater coverage by collective bargaining, especially in the manufacturing and public sectors; and (4) more easily measurable productivity, which limits the potential for discriminatory evaluations. While gender wage gaps still exist in these occupations, the reasons for them are different from those in skilled professional contexts. The focus is more on occupational sex segregation and the undervaluing of female-dominated work, rather than on barriers to career advancement (Levanon, 2009).

This occupational classification strategy allows the analysis to test whether gender wage gaps are larger in high-skill and managerial positions, where discretionary pay and promotion systems can create opportunities for discrimination, or whether transparency and equal opportunity policies have been more effective in professional contexts than in routine occupations.


\subsection{Welfare Regime Classification and Institutional Context}

The cross-national analysis uses a welfare regime typology to understand how labor market structures and gender equality policies differ across European countries. This classification, which was first developed by \citep{espingandersen1990}, has since been updated to account for the transitions of post-socialist countries \citep{mandel2007}, the family-based welfare systems of Southern Europe \citep{olivetti2008}, and the specific effects of institutions on gender \citep{mandel2005}. This updated framework helps to explain how national institutional systems influence gender wage outcomes in modern European labor markets.

Welfare regimes are composed of state policies, labor market institutions, and family support systems. Together, these factors shape employment patterns, wage-setting practices, and gender equality outcomes. Different regime types have different approaches to social protection, labor market regulations, childcare, and the balance between work and family, all of which can affect the gender wage gap.

\textbf{Regime Classification and Characteristics:}

The analysis first looks at the Nordic countries (Denmark, Finland, Iceland, Norway, and Sweden), which are known for their universal welfare systems, high taxes, large public sectors, and strong labor unions. (1) Their labor markets are characterized by centralized wage bargaining and compressed wage structures, which ensures that wages are set in a coordinated way across different sectors and that jobs are well-protected. (2) These countries have the most comprehensive family policies in Europe, including heavily subsidized childcare, generous parental leave with "use-it-or-lose-it" quotas for fathers, and support systems for dual-earner households. (3) Their gender equality policies are also the most developed, with strong enforcement of equal pay laws, active monitoring, and affirmative action in public employment.

Continental countries include Austria, Belgium, France, Germany, Luxembourg, the Netherlands, and Switzerland. These are conservative-corporatist welfare states with insurance-based social protection and a history of male-breadwinner traditions, although they are in the process of modernizing. (1) Wage-setting is based on social partnership and industry-level bargaining, with moderate labor market regulation that balances employer flexibility with worker protection. (2) Family policies have historically been shaped by traditional gender roles, with limited support for childcare, but recent reforms have significantly improved childcare provision and parental leave benefits. (3) Gender equality enforcement relies mainly on legal frameworks and collective bargaining agreements, with moderate monitoring and a gradual implementation of EU directives.

Mediterranean countries include Cyprus, Greece, Italy, Malta, Portugal, and Spain. Their welfare states developed later and are more reliant on family-based social protection. They also have large informal economies and traditional gender norms. (1) Their labor markets are divided between secure, permanent jobs and insecure, temporary jobs, which creates a dual structure with strong protection for permanent workers but weak overall regulation due to the prevalence of temporary contracts. (2) Family networks play a larger role in childcare than the state, with limited public support and below-average parental leave benefits, although there have been improvements since they joined the EU. (3) Gender equality policies have also improved since joining the EU, but enforcement is often weak due to limited resources and persistent traditional norms.

Liberal countries are Ireland and the United Kingdom. These countries favor market-driven labor regulation, minimal welfare, and weak unions. (1) Wage-setting is decentralized, with flexible employment contracts, limited job protection, and high inequality as a result of minimal labor market regulation. (2) Family policy is limited, with most childcare being privately provided and expensive, minimal parental leave compared to EU standards, and an emphasis on market-based solutions over state support. (3) Gender equality efforts focus on formal equal opportunity laws and anti-discrimination legislation rather than structural interventions, with enforcement relying on individual lawsuits rather than proactive monitoring.

Eastern European countries include Bulgaria, Croatia, Czechia, Estonia, Hungary, Latvia, Lithuania, Moldova, Poland, Romania, Slovakia, and Slovenia. These post-socialist economies inherited a legacy of formal gender equality from the communist era, including high female employment and state-supported childcare. However, they have undergone significant institutional changes since transitioning to market economies. (1) Their labor market regulation is in a state of transition, with a mix of socialist-era collective agreements and increasing market flexibility, which has led to weaker job protection and declining union coverage. (2) Their family policies still retain some of the infrastructure from the socialist era, including state-supported childcare and guaranteed parental leave, but funding constraints have reduced the quality and availability of these services. (3) While they have formal legal equality from the socialist era, enforcement has weakened during the market transition, with limited monitoring and the emergence of new forms of market-based discrimination.

Balkan countries are Albania, Bosnia and Herzegovina, Kosovo, Montenegro, North Macedonia, and Serbia. These are EU candidate or potential candidate countries with weak institutions, transitional economies, and traditional gender roles. (1) Their labor market regulation is weak and still developing, with limited collective bargaining coverage, a large informal economy that is beyond the reach of regulation, and inconsistent enforcement of job protection. (2) Their family policies are minimal, with limited public childcare, below-average parental leave benefits, and a heavy reliance on traditional family structures for caregiving. (3) Although they have formal equality laws as a requirement for EU candidacy, enforcement is weak due to a lack of monitoring institutions, limited resources, and persistent traditional gender norms.

Turkey is classified separately due to its unique institutional context, which combines Islamic cultural influences, rapid economic growth, regional diversity, and a degree of European integration. (1) Its labor market regulation is moderate but uneven, with formal protections in large companies coexisting with a large informal sector that is not covered by regulation. (2) Its family policies are limited compared to EU standards, with minimal public childcare and traditional expectations about women's caregiving roles, although recent reforms have expanded parental leave in the formal sector. (3) Its gender equality enforcement is still developing, with legal frameworks that are increasingly aligned with EU standards but with implementation that is hampered by traditional cultural norms, particularly regarding women's participation in the labor force.

This classification allows us to examine how national institutions shape sectoral and occupational wage gaps. The regime typology highlights differences in three key factors that affect gender wage equality: (1) the degree of labor market regulation (centralized or decentralized wage-setting, strength of job protection); (2) the generosity of family policies (childcare, parental leave, work-life support); and (3) the effectiveness of gender equality enforcement (laws, monitoring, affirmative action). By using country groups and sector interactions, the analysis can test whether the causes of the wage gap are the same everywhere or if they depend on the specific welfare-state context.

\subsection{Data Cleaning and Quality Assurance Procedures}

To ensure that the results of this study are valid and reliable, a rigorous data cleaning process was used. This process involved several steps to address data quality at different levels:

\textbf{Stage 1: Establishment-Level Validation}
\begin{itemize}
\item Establishments with fewer than 10 employees were removed to ensure that the gender comparisons were statistically reliable.
\item Establishments with implausible wage distributions (a coefficient of variation greater than 3) were excluded.
\item The consistency of the NACE classification was checked across all survey years.
\end{itemize}

\textbf{Stage 2: Individual-Level Cleaning}
\begin{itemize}
\item Annual earnings were trimmed at the 1st and 99th percentiles within each country and year to remove any coding errors.
\item Observations with incomplete sectoral or occupational classifications were excluded.
\item The consistency of employment data was validated by removing observations with implausible earnings distributions.
\end{itemize}

\textbf{Stage 3: Cell-Level Aggregation Quality}
\begin{itemize}
\item A minimum of 30 observations was required for each country-sector-occupation-year cell to ensure a reliable calculation of the wage gap.
\item Cells with only one gender were suppressed, as it is not possible to calculate a wage gap in these cases.
\item The calculated wage gaps were winsorized at the 5th and 95th percentiles to address any measurement errors.
\end{itemize}

These procedures were applied to a total of 14,430 panel observations, representing 5,176 unique country-sector-occupation cells from 40 European countries over four time periods (2010, 2014, 2018, and 2022). The cleaning process was designed to maintain a sufficient sample size for statistical precision while also ensuring that the methodology was consistent across all countries and time periods.

\subsection{Missing Data Analysis and Treatment}

To handle missing data, this study takes a careful approach. Missing data can occur in two ways: when countries do not participate in a survey year (unit non-response) or when some information is incomplete for a participating country (item non-response).

\textbf{Unit Non-Response Patterns:}
\begin{itemize}
\item Bulgaria and Romania joined the survey in 2010 after becoming EU members.
\item Croatia joined in 2014 after its accession.
\item Greece did not participate in the 2014 survey due to administrative issues.
\item Ireland and Denmark participated intermittently based on their national statistical priorities.
\end{itemize}

To address this, the main models use an unbalanced panel, which allows countries to be included even if they are not present in all survey years. For robustness checks, a balanced sub-sample is used, which only includes countries that participated in all years.

\textbf{Item Non-Response Treatment:}

For the main variables like sector, occupation, and wage, less than 2\% of the data is missing. The gender pay gap is calculated for each combination of country, sector, occupation, and year (referred to as a "cell"). To ensure the estimates are reliable, only cells with at least 30 observations are included in the analysis. Cells with fewer observations are excluded to prioritize accuracy. This approach avoids the need for data imputation because the analysis uses summary data from each cell rather than individual responses.

\subsection{Panel Structure and Identification Strategy}

The dataset has a hierarchical structure, with 14,430 observations grouped into 5,176 unique panels defined by country, sector, and occupation. These panels are observed over four time periods: 2010, 2014, 2018, and 2022. The number of observations per panel ranges from one to four, and 43.0\% of the panels (2,223 in total) have data for all four periods. This structure makes it possible to track how the wage gap changes within each panel over time while controlling for factors that do not change.

The main analytical strategy is to use the variation within each panel over time. This means comparing changes in the wage gap within the same country-sector-occupation group. This method helps to remove bias from stable, unobserved factors like cultural norms, industrial relations, or the prestige of an occupation, which could be related to both gender balance and wage levels. The data shows enough within-panel variation to support the use of fixed effects estimation.



\subsection{Descriptive Sample Characteristics}

The sample is well-balanced across the different sectors. The Services sector makes up 57.0\% of the observations (8,224), followed by Industry at 19.4\% (2,806), the Public Sector at 17.9\% (2,588), and Construction at 5.6\% (812). This balance ensures that the analysis covers a wide range of institutional contexts and has enough variation within each sector to produce reliable estimates.


\section{METHODOLOGICAL DESIGN}

This study uses a sophisticated panel data approach to examine the factors that determine gender wage gaps in European labor markets. The method combines several estimation strategies to ensure the results are reliable, while also addressing common challenges in wage data, such as unobserved differences, selection bias, and potential endogeneity.

\subsection{Panel Data Econometric Framework}

The study uses the panel structure of the SES data to examine the relationship between sectoral and occupational characteristics and differences in gender wages. The main model is as follows:

\begin{equation}
GPG_{it} = \alpha_i + \beta_1 SECTOR_{it} + \beta_2 OCC_{it} + \gamma_t + \epsilon_{it}
\end{equation}

In this equation, $GPG_{it}$ is the gender pay gap for a specific panel unit $i$ at time $t$. The term $\alpha_i$ accounts for fixed characteristics of the panel that do not change over time. $SECTOR_{it}$ refers to the sector (Industry, Construction, Services, Public Sector), and $OCC_{it}$ refers to the occupational level (High-Skill, Managerial). The term $\gamma_t$ accounts for time-specific effects, and $\epsilon_{it}$ is the error term \citep{wooldridge2010}.

The model is designed to fit the structure of the data and the research questions. The data from the Eurostat Structure of Earnings Survey is aggregated at the level of country, sector, occupation, and year. It includes three main variables: the gender pay gap (the dependent variable), the sector, and the occupation. Individual characteristics like education, experience, and hours worked are not available in this aggregated format. Each data point represents the average wage gap for a specific group in a given year, which has already been calculated by Eurostat from individual worker data.

These variables are directly related to the study's focus on the structural and institutional causes of wage inequality. Sectoral classification is important because men and women tend to work in different sectors with different pay structures, unionization rates, and regulations. For example, the public sector typically has stronger enforcement of equal pay and more transparent wages than the private sector. Occupational hierarchy measures (high-skill and managerial positions) help to determine whether wage gaps are due to barriers to advancement (the "glass ceiling") or discrimination at the entry level. These factors are relevant for policy, as governments can target specific sectors or address occupational segregation. The use of standardized classifications (NACE Rev. 2 and ISCO-08) allows for meaningful comparisons across 40 European countries. The panel fixed effects ($\alpha_i$) control for all stable characteristics of each country-sector-occupation group, such as workforce composition, skill requirements, and institutional features. This model directly tests the study's three main hypotheses about sectoral differences, occupational hierarchies, and institutional contexts, following established comparative approaches \citep{mandel2007, olivetti2008}.

The model is designed to examine the effects of sector and occupation separately. This allows for an analysis of their independent and combined influences on wage inequality. This approach goes beyond previous research, which has often studied these factors in isolation. As a result, it can identify whether institutional (sectoral) and hierarchical (occupational) mechanisms of wage discrimination complement or substitute for each other.

\subsection{Estimator Selection and Diagnostic Procedures}

Choosing between a fixed effects (FE) and a random effects (RE) model is an important step. The FE model uses a within-panel transformation to remove any unobserved factors that do not change over time, but it cannot identify the effects of variables that are constant over time. The equation for the FE model is:

\begin{equation}
(GPG_{it} - \overline{GPG}_i) = \beta_1(SECTOR_{it} - \overline{SECTOR}_i) + \beta_2(OCC_{it} - \overline{OCC}_i) + (\epsilon_{it} - \overline{\epsilon}_i)
\end{equation}

The RE model, on the other hand, assumes that the unobserved effects are not correlated with the other variables in the model. This allows it to identify all coefficients, but it can lead to biased results if this assumption is wrong. The equation for the RE model is:

\begin{equation}
GPG_{it} = \alpha + \beta_1 SECTOR_{it} + \beta_2 OCC_{it} + \gamma_t + (\alpha_i - \alpha + \epsilon_{it})
\end{equation}

The choice between these two models is not just a theoretical one; it needs to be tested empirically. This study uses the Hausman test to determine which model is more appropriate. The Hausman test checks whether there are systematic differences between the FE and RE estimates. If the test rejects the null hypothesis (with a p-value less than 0.05), it means that the RE model's assumption is likely violated, and the FE model is the better choice. The results of the Hausman test for all the main models are presented in Section 6.

\subsection{Robust Inference and Clustering Strategies}

Panel data from multiple countries and time periods can violate standard statistical assumptions in two important ways. First, observations within the same country-sector-occupation group are not independent; they are correlated over time (serial correlation). Second, the variability of wage gaps may differ systematically across panels (heteroskedasticity). Diagnostic tests confirm that both of these issues are present in this dataset. The Breusch-Godfrey test shows significant serial correlation ($\chi^2$ = 1144.3, p < 0.001), and the Breusch-Pagan test reveals heteroskedasticity (BP = 181.73, p < 0.001). If these problems are ignored, the standard errors will be unreliable, and the statistical conclusions will be incorrect.

To address these issues, the analysis uses cluster-robust standard errors, which account for the correlation within panels. The main approach is to use HC1 heteroskedasticity-consistent standard errors clustered at the panel level (country-sector-occupation groups). This allows observations within the same panel to be correlated while treating different panels as independent. The robustness analysis also considers other clustering methods, such as two-way clustering (by panel and year), country-level clustering (40 clusters), and wild cluster bootstrap methods (1,000 replications). These alternative methods produce slightly larger standard errors but do not change the statistical significance of the main findings, which shows that the results are robust to different assumptions about how the errors are correlated.

The Hausman test ($\chi^2$ = 104.72, df = 3, p < 2.2e-16) strongly rejects the null hypothesis that there are no systematic differences between the fixed and random effects models. This indicates that the Fixed Effects model is statistically preferred. Therefore, the analysis uses Fixed Effects as the main model to control for unobserved factors that do not change over time, with cluster-robust standard errors to address serial correlation and heteroskedasticity. Random Effects models with interaction terms are used as a complementary approach to examine the interactions between sector and occupation, as these cannot be identified in the Fixed Effects model due to collinearity with the panel fixed effects.

\subsection{Endogeneity Concerns and Identification Strategies}

Potential endogeneity is a concern in this analysis, which could arise from three main sources. Each of these requires a different strategy to address it:

1.  \textbf{Reverse Causality:} It is possible that wage gaps could influence sectoral employment, for example, through selection effects. This analysis relies on an employer-based sampling frame, which helps to ensure that individual choices about where to work do not affect the wage structures of establishments during the survey periods.

2.  \textbf{Omitted Variable Bias:} If there are unobserved differences in productivity that are related to gender composition, this could bias the results. The use of panel fixed effects helps to remove any confounding factors that do not change over time. The inclusion of detailed occupational controls also helps to account for different skill requirements. The robustness analysis checks how stable the coefficients are across different models with varying sets of controls to see how sensitive the results are to potential unobserved factors.

3.  \textbf{Measurement Error:} If there are errors in the calculation of the wage gap, this could lead to an underestimation of the coefficients (a phenomenon known as classical attenuation bias). To minimize this, the analysis requires a minimum of 30 observations per cell and winsorizes extreme values at the 5th and 95th percentiles. This helps to ensure that measurement errors do not have an outsized impact on the results.


\subsubsection{Identification Assumptions and Limitations}

Panel data methods are an improvement over cross-sectional analysis, but they still rely on several key assumptions. This section discusses the main assumptions of the empirical strategy and acknowledges its important limitations.

\textbf{Assumption 1 (Exogeneity of Sectoral and Occupational Classifications):} The analysis assumes that the sectoral and occupational categories are not systematically related to unobserved factors that affect gender pay gaps. This assumption is more plausible than it might seem because the classifications are based on structural and institutional features rather than individual choices. NACE sectoral codes (Industry, Construction, Services, Public Sector) are determined by the type of economic activity and production technology—a manufacturing firm is classified as Industry regardless of its wage structure. Similarly, ISCO occupational codes (Managers, Professionals, Technicians, etc.) are based on the skill level and tasks of jobs, not on the gender composition or pay levels within those roles. This means that a group's gender pay gap does not determine how it is classified; the classification is based on objective criteria about what the sector produces and what tasks the occupation involves. However, this assumption could still be violated if women systematically choose to work in certain sectors or occupations based on anticipated discrimination, which is a limitation that this study cannot fully address with aggregated data.

\textbf{Assumption 2 (Time-Invariant Unobserved Heterogeneity):} The panel fixed effects ($\alpha_i$) control for stable characteristics of each country-sector-occupation group that do not change over the study period. These might include deep-rooted cultural attitudes toward gender roles, long-standing industrial relations traditions, or the historical prestige of certain occupations. However, the 12-year study period (2010-2022) is long enough that some of these factors may have changed. For example, cultural attitudes toward working mothers may have shifted, new equal pay laws may have been passed, or unionization rates may have declined in certain sectors. The panel fixed effects cannot account for these time-varying changes, which is an important limitation. The time fixed effects ($\gamma_t$) capture common trends across all panels but cannot control for panel-specific changes. This suggests that the estimates should be interpreted as associations between sectoral/occupational structure and wage gaps rather than definitive causal effects.

\textbf{Assumption 3 (Common Trends for Beta Convergence Analysis):} The convergence analysis examines whether countries with larger initial gender pay gaps in 2010 experienced faster reductions by 2022. Beta convergence is the idea that countries starting from a worse position tend to "catch up" to better-performing countries over time, leading to a convergence of outcomes. In this context, it means that a country with a 25\% gender pay gap in 2010 would be expected to reduce its gap more quickly than a country starting at 10\%, possibly due to greater policy attention, more room for improvement, or EU-wide harmonization pressures. The identification of beta convergence assumes that countries would have experienced similar trends if they had not started from different positions—that is, the only reason high-gap countries reduced their gaps faster is because they started higher, not because of other concurrent factors. This "parallel trends" assumption cannot be directly tested. However, it is supported by three observations: all countries faced similar macroeconomic shocks (the financial crisis recovery, COVID-19), EU member states share common regulatory frameworks (equal pay directives, employment protection legislation), and robustness checks limited to Eurozone countries (which share monetary policy) show consistent convergence patterns.

\textbf{Limitations and Robustness Checks:}

This study has several limitations that could not be addressed with the available aggregated data. First, if gender pay gaps influence women's decisions to participate in the labor force, the observed gaps may underestimate the true level of wage discrimination because women in high-inequality sectors may choose to leave employment altogether (selection bias). The use of an employer-based survey reduces but does not eliminate this concern. Second, the share of employment in different sectors changes over time (e.g., a decline in manufacturing and growth in the service sector), which could be confused with temporal trends. The panel fixed effects address this by examining changes within country-sector-occupation groups, but changes in occupational roles within sectors could still be a confounding factor. Third, gender pay gaps calculated from cell averages can have sampling variability, especially in smaller cells. The requirement of a minimum cell size ($N \geq 30$) and the winsorization of extreme values help to reduce but do not eliminate measurement error, which may lead to an underestimation of the coefficients.

To assess the robustness of the findings, the study includes comprehensive sensitivity analyses, which are reported in Section 6.8 and summarized in Tables 13-15. These include alternative sample restrictions (balanced panels, large countries only, exclusion of extreme values), alternative estimation methods (quantile regression, different clustering structures), and subsample analyses (EU-15 vs. new member states, pre-pandemic vs. post-2014 periods, geographic restrictions). The consistency of the results across more than 15 alternative specifications provides confidence that the main findings reflect genuine empirical patterns rather than being artifacts of particular modeling choices or data limitations.

\subsection{Heterogeneous Treatment Effects and Interaction Analyses}

The analysis goes beyond average effects to look at how the impacts differ across various institutional contexts. This is done using several different models, each designed to address one of the research hypotheses.

\textbf{Model Specification 1: Detailed Sectoral Analysis (Hypothesis 1)}

To test Hypothesis 1, which deals with sectoral factors and negative gaps, the analysis uses a model with all 18 NACE Rev. 2 sectors:

\begin{equation}
\begin{split}
GPG_{it} = \alpha + \beta_1 Mining_{it} + \beta_2 Manufacturing_{it} + \beta_3 Electricity_{it} \\
+ \beta_4 Water_{it} + \beta_5 Construction_{it} + \beta_6 Trade_{it} + \beta_7 Transport_{it} \\
+ \beta_8 Hospitality_{it} + \beta_9 IT_{it} + \beta_{10} Finance_{it} + \beta_{11} RealEstate_{it} \\
+ \beta_{12} Professional_{it} + \beta_{13} Admin_{it} + \beta_{14} PublicAdmin_{it} \\
+ \beta_{15} Education_{it} + \beta_{16} Health_{it} + \beta_{17} Arts_{it} \\
+ \beta_{18} HighSkill_{it} + \beta_{19} Managerial_{it} + \gamma_t + \epsilon_{it}
\end{split}
\end{equation}

In this model, each sector has its own indicator variable, which is 1 if the observation belongs to that sector and 0 otherwise. "Other Services" is used as the reference category. $HighSkill_{it}$ is an indicator for high-skill occupations (ISCO 1-3), and $Managerial_{it}$ is for managerial positions (ISCO 1). The coefficients $\beta_1$ to $\beta_{17}$ measure the gap in each sector relative to Other Services, after controlling for occupation. $\gamma_t$ accounts for year effects, and $\epsilon_{it}$ is the error term. This Random Effects model allows for the identification of all sector coefficients. This detailed approach helps to uncover differences across all 18 industrial classifications.

\textbf{Model Specification 2: Sector-Occupation Interactions (Hypothesis 2)}

Hypothesis 2 suggests that the penalties for being in certain occupations vary depending on the sector. This interaction model examines how institutional and hierarchical factors work together:

\begin{equation}
\begin{split}
GPG_{it} = \alpha_i + \beta_1 Industry_{it} + \beta_2 Construction_{it} + \beta_3 PublicSector_{it} \\
+ \beta_4 HighSkill_{it} + \beta_5 Managerial_{it} \\
+ \beta_6(Industry \times HighSkill)_{it} + \beta_7(Industry \times Managerial)_{it} \\
+ \beta_8(PublicSector \times HighSkill)_{it} + \beta_9(PublicSector \times Managerial)_{it} \\
+ \gamma_t + \epsilon_{it}
\end{split}
\end{equation}

Here, "Services" is the reference category for sectors. The interaction coefficients $\beta_6$ to $\beta_9$ show how occupational wage gaps differ across sectors. If these interactions are significant, it means that the "glass ceiling" effect and skill-based wage differences are not the same everywhere but depend on the specific labor market segment.

\textbf{Model Specification 3: Country Group Analysis (Hypothesis 3 - Part 1)}

To test how national institutions affect wage gaps, the analysis includes country group classifications based on welfare regimes:

\begin{equation}
\begin{split}
GPG_{it} = \alpha_i + \beta_1 Industry_{it} + \beta_2 Construction_{it} + \beta_3 PublicSector_{it} \\
+ \beta_4 HighSkill_{it} + \beta_5 Managerial_{it} \\
+ \delta_1 Nordic_i + \delta_2 Mediterranean_i + \delta_3 Eastern_i + \delta_4 Liberal_i \\
+ \delta_5 Balkans_i + \delta_6 Other_i \\
+ \theta_1(Nordic \times PublicSector)_{it} + \theta_2(Mediterranean \times Industry)_{it} \\
+ \theta_3(Eastern \times Industry)_{it} + \gamma_t + \epsilon_{it}
\end{split}
\end{equation}

In this model, "Continental" is the reference group for countries, and "Services" is the reference for sectors. The model includes 4 sector categories, 2 occupational indicators, and 7 country groups. The coefficients $\delta_1$ to $\delta_6$ show the baseline differences in gender wage gaps across different institutional regimes compared to Continental countries. The interaction terms $\theta_1$, $\theta_2$, and $\theta_3$ test whether specific sectoral wage structures are influenced by the type of welfare regime.

\textbf{Model Specification 4: Beta Convergence Analysis (Hypothesis 3 - Part 2)}

The convergence hypothesis predicts that countries with larger initial gaps will see them shrink faster due to EU policy harmonization and competitive pressures. This model uses a cross-sectional regression with data from 30 European countries:

\begin{equation}
\Delta GPG_i = \alpha + \beta \cdot GAP_{2010,i} + \epsilon_i, \quad i = 1, \ldots, 30
\end{equation}

Here, $\Delta GPG_i$ is the total change in a country's gender pay gap over the 12-year period, and $GAP_{2010,i}$ is the gap in 2010. Each country provides one observation for this regression. A negative coefficient ($\beta < 0$) indicates beta convergence, meaning that countries that started with higher gaps experienced larger reductions. This cross-sectional approach is different from the panel models (Models 1-3), as it uses the country as the unit of analysis rather than the country-sector-occupation panel. It tests whether the initial conditions can predict the rate of convergence across European labor markets.

\subsection{Robustness and Sensitivity Analyses}

To check that the findings are stable, the analysis includes 17 robustness checks, which are organized into three categories (the detailed results are in Section 6.8, Tables 13-15):

\textbf{Table 13 - Alternative Specifications (7 checks):} The main random effects model is compared against six other models: (1) a balanced panel that only includes data from 2010-2022, (2) a model that excludes extreme gaps (those above or below 3 standard deviations), (3) a subsample of large countries with populations over 5 million, (4) a quantile regression at the median to reduce the influence of outliers, (5) a fixed effects model, and (6) a model where the dependent variable is winsorized at the 1st and 99th percentiles.

\textbf{Table 14 - Clustering Structures (4 checks):} The standard errors are recalculated using four different clustering methods: (1) panel-level clustering (the main approach), (2) two-way clustering by panel and year, (3) country-level clustering with 40 clusters, and (4) a wild cluster bootstrap with 1,000 replications to address concerns about having a small number of clusters.

\textbf{Table 15 - Geographic and Temporal Subsamples (6 checks):} The stability of the coefficients is checked across different subsamples: (1) only the EU-15 founding members, (2) only the new member states that joined after 2004, (3) only Eurozone countries, (4) the pre-pandemic period (2010-2018), (5) the post-2014 period after survey methods changed, and (6) a 2022 cross-section to see if the patterns still hold in the most recent data.

\subsection{Statistical Software and Computational Implementation}

All analyses were performed using R (version 4.3.1) to ensure that the results can be reproduced. The main estimation was done with the `plm` package for panel data models, `lmtest` and `sandwich` for robust inference, and `fixest` for high-dimensional fixed effects. Custom functions were used for two-way clustering and wild bootstrap procedures, and these were validated against Stata to ensure consistency across different software.

Efficiency was a key consideration, especially for the bootstrap procedures, which require repeated estimations. Parallel processing across eight cores was used to reduce the computation time for the wild bootstrap confidence intervals (with 10,000 replications) from 4 hours to 35 minutes. Memory-efficient sparse matrix representations were used to handle the high-dimensional fixed effects without computational issues.

This integrated methodological framework provides a rigorous way to identify the factors that determine the gender wage gap, while also acknowledging the limitations of observational data. Through the use of multiple estimation strategies, robust inference procedures, and comprehensive sensitivity analyses, this approach generates credible causal estimates that help to advance our understanding of the mechanisms behind discrimination in European labor markets.

\section{RESULTS}

The analysis shows that gender-based pay differences in Europe are complex and vary across different sectors and job levels. This section presents the findings from the Structure of Earnings Survey data, which includes 14,430 observations from 5,176 unique country-sector-occupation groups in 40 European countries between 2010 and 2022. The data was cleaned using the methods described in Section 4.2 to ensure its quality.

\subsection{Descriptive Statistics and Sample Characteristics}

Table \ref{tab:descriptive_stats} presents summary statistics disaggregated by key analytical dimensions, revealing substantial heterogeneity in the magnitudes of the gender pay gap across observational units.


\begin{table}[htbp]
\centering
\caption{Descriptive Statistics: Gender Pay Gap by 18 Detailed NACE Sectors}
\label{tab:descriptive_stats}
\small
\begin{tabularx}{\textwidth}{l *{5}{>{\centering\arraybackslash}X} r}
\hline\hline
\textbf{Sector (NACE Rev. 2)} & \textbf{Mean} & \textbf{SD} & \textbf{Min} & \textbf{Max} & \textbf{N} & \textbf{\% Neg.} \\
\hline
\multicolumn{7}{l}{\textit{Panel A: High-Gap Sectors (Mean > 15\%)}} \\
Manufacturing (C) & 17.7 & 9.91 & -18.7 & 55.4 & 1,131 & 4.1 \\
Mining \& Quarrying (B) & 19.3 & 14.6 & -19.7 & 64.5 & 461 & 12.8 \\
Finance \& Insurance (K) & 18.3 & 12.1 & -19.9 & 62.2 & 649 & 8.2 \\
\hline
\multicolumn{7}{l}{\textit{Panel B: Medium-Gap Sectors (12-15\%)}} \\
Wholesale \& Retail Trade (G) & 15.3 & 10.7 & -18.3 & 54.8 & 1,067 & 7.7 \\
Electricity \& Gas Supply (D) & 15.5 & 11.5 & -18.8 & 60.0 & 570 & 8.2 \\
Transportation \& Storage (H) & 15.1 & 12.2 & -19.9 & 78.9 & 868 & 11.1 \\
Construction (F) & 14.7 & 11.3 & -19.3 & 55.5 & 812 & 11.4 \\
Professional Services (M) & 14.2 & 11.8 & -17.8 & 57.0 & 861 & 12.2 \\
IT \& Communication (J) & 13.7 & 11.3 & -19.8 & 56.8 & 760 & 11.1 \\
Arts \& Entertainment (R) & 13.2 & 13.1 & -19.0 & 79.5 & 800 & 13.4 \\
Real Estate (L) & 13.6 & 13.1 & -19.8 & 87.9 & 631 & 15.0 \\
Human Health (Q) & 12.7 & 12.6 & -19.3 & 85.3 & 997 & 15.1 \\
\hline
\multicolumn{7}{l}{\textit{Panel C: Low-Gap Sectors (Mean < 12\%)}} \\
Other Services (S) & 13.1 & 13.3 & -19.8 & 61.0 & 822 & 16.8 \\
Admin \& Support Services (N) & 12.0 & 11.7 & -19.5 & 61.4 & 968 & 16.0 \\
Water Supply \& Waste (E) & 11.1 & 11.7 & -18.7 & 75.4 & 644 & 17.2 \\
Public Administration (O) & 10.8 & 10.6 & -19.3 & 55.5 & 734 & 14.9 \\
Education (P) & 9.78 & 10.9 & -18.9 & 66.8 & 857 & 16.9 \\
Hospitality \& Food Services (I) & 9.71 & 11.1 & -18.6 & 60.0 & 798 & 20.6 \\
\hline
\multicolumn{7}{l}{\textit{Panel D: Temporal Evolution (All Sectors)}} \\
2010 & 15.4 & 12.9 & -317.9 & 87.9 & 4,424 & 10.2 \\
2014 & 13.4 & 11.7 & -190.2 & 78.9 & 3,211 & 11.8 \\
2018 & 13.5 & 11.9 & -176.7 & 79.5 & 3,418 & 12.4 \\
2022 & 12.3 & 11.2 & -195.3 & 85.3 & 3,377 & 13.9 \\
\hline\hline
\end{tabularx}
\begin{tablenotes}[para,flushleft]
\small
\textit{Source:} Own calculation based on: Structure of Earnings Survey, Eurostat, 2010-2022.
\textit{Notes:} Sample: 14,430 observations across 18 NACE Rev. 2 sectors, 40 countries, 2010-2022. Mean and SD in percentage points. \% Neg. indicates the proportion of observations with negative gaps (women out-earning men). Sectors ordered by mean gap within panels. Panel D aggregates across all sectors. Extreme values result from small-cell observations with high sampling variability; the main analysis uses robust estimation with cluster-standard errors.
\end{tablenotes}
\end{table}
\newpage


The industry sector has the highest average gender pay gap at 16.0\%, followed by construction at 14.7\% and services at 13.9\%. The public sector shows the lowest gap at 11.2\%. This initial difference of 4.8 percentage points between the industry and public sectors supports Hypothesis 1, which suggests that different sectors have different effects on pay equality. A closer look at the 18 sectors in Table \ref{tab:descriptive_stats} shows large differences within these broad categories. For example, manufacturing (17.7\%) and mining (19.3\%) have high gaps, driving up the average for the industry sector. In contrast, service sectors where many women work, like hospitality (9.71\%) and education (9.78\%), have the smallest gaps. It is also worth noting that in the sectors with the lowest pay gaps, it is more common for women to earn more than men. This is most frequent in hospitality (20.6\% of cases), water supply (17.2\%), and education (16.9\%). This suggests that in sectors with more women and more standardized pay scales, women have a better chance of achieving equal or higher pay.

Over time, the gender pay gap has been shrinking. The average gap fell from 15.4\% in 2010 to 12.3\% in 2022, which is a reduction of 3.1 percentage points over 12 years. At the same time, the number of cases where women earn more than men has slowly increased, from 10.2\% in 2010 to 13.9\% in 2022. This shows that progress is being made in two ways: the pay gap is getting smaller in sectors dominated by men, and women are gaining a pay advantage in sectors where more women work.

\subsection{Panel Regression Estimates}

Table \ref{tab:main_results} shows the results from the panel regression analysis, which examines the different factors that influence the gender pay gap. To choose the right model, a Hausman test was performed. The results of the test ($\chi^2$ = 104.72, df = 3, p < 2.2e-16) showed that a Fixed Effects model is the best fit for controlling for factors that do not change over time. The main model used is a Fixed Effects model with time dummies. A Random Effects model with interactions between sectors and occupations is also used to keep all the theoretical predictors in the analysis. Both models use HC1 cluster-robust standard errors to correct for issues like serial correlation and heteroskedasticity.

\begin{table}[H]
\centering
\caption{Gender Pay Gap Determinants: Panel Data Models with Sector-Occupation Interactions}
\label{tab:main_results}
\small
\begin{tabularx}{\textwidth}{l *{2}{>{\centering\arraybackslash}X}}
\hline\hline
\textbf{Variable} & \textbf{Fixed Effects} & \textbf{Random Effects} \\
\hline
\multicolumn{3}{l}{\textit{Sectoral Variables (RE only)}} \\
Industry & -- & 4.392*** \\
 & & (0.512) \\
Construction & -- & 0.829 \\
 & & (0.613) \\
Public Sector & -- & -2.706*** \\
 & & (0.494) \\
& & \\
\multicolumn{3}{l}{\textit{Occupational Hierarchy (RE only)}} \\
High-Skill Occupation & -- & 2.936*** \\
 & & (0.411) \\
Managerial Position & -- & 4.663*** \\
 & & (0.602) \\
& & \\
\multicolumn{3}{l}{\textit{Time Fixed Effects}} \\
Year 2014 & -1.361*** & -1.747*** \\
 & (0.204) & (0.200) \\
Year 2018 & -1.015*** & -1.482*** \\
 & (0.218) & (0.200) \\
Year 2022 & -2.173*** & -2.692*** \\
 & (0.216) & (0.203) \\
& & \\
Constant & -- & 13.406*** \\
 & & (0.257) \\
\hline
\multicolumn{3}{l}{\textit{Model Statistics}} \\
Observations & 14,430 & 14,430 \\
Number of panels & 5,176 & 5,176 \\
Countries & 40 & 40 \\
R$^2$ (within) & 0.0121 & 0.0601 \\
Hausman test & \multicolumn{2}{c}{$\chi^2$ = 104.72*** (FE preferred)} \\
\hline\hline
\end{tabularx}
\begin{tablenotes}[para,flushleft]
\small
\textit{Source:} Own calculation based on: Structure of Earnings Survey, Eurostat, 2010-2022.
\textit{Notes:} Fixed Effects (FE) model uses within-panel transformation, eliminating time-invariant regressors. Random Effects (RE) model includes all variables with sector-occupation interactions. HC1 cluster-robust standard errors in parentheses. Reference categories: Services sector, non-high-skill and non-managerial occupations, Year 2010. All sectoral coefficients represent deviations from Services baseline. Sample: 40 European countries, 18 NACE sectors aggregated into 4 broad categories (Industry, Construction, Services, Public) with 9 ISCO occupations, 2010-2022. *** p<0.001, ** p<0.01, * p<0.05
\end{tablenotes}
\end{table}

The results show that gender pay gaps vary a lot depending on the sector and occupation. The Random Effects model, which includes all theoretical factors, shows large differences between sectors when compared to the Services sector (which is used as a baseline). The pay gap is much higher in the Industry sector (+4.392 percentage points, p < 0.001), but much lower in the Public Sector (-2.706 percentage points, p < 0.001). This creates a 7.10 percentage-point difference between these two sectors. This finding supports the idea that sectors dominated by men tend to have older pay practices and less pressure to ensure equal pay.

The Public Sector has the lowest gender pay gaps because it has clear rules for pay and promotions, and stronger enforcement of equality laws. The Construction sector is in the middle, with a slightly higher gap (+0.829 percentage points, p = 0.142), but this result is not statistically significant. The Services sector is used as the baseline for comparison and has moderate pay gaps.

The effects of job level on the gender pay gap are complex. In the basic model (without interactions), high-skill jobs (including Managers, Professionals, and Technicians, ISCO 1-3) have larger pay gaps (+2.936 percentage points, p < 0.001). This goes against simple human capital theory but supports the "glass ceiling" idea, which suggests that discrimination is stronger at higher job levels. Managerial positions have even larger gaps (+4.663 percentage points, p < 0.001), showing that there are still major barriers to equal pay at the top of organizations, where pay and promotion decisions are more subjective. However, these general effects hide important differences that become clear when we look at the interactions between sectors and occupations (discussed later). The pay gaps for high-skill and managerial jobs are similar because managers are a large part of the high-skill group. The extra gap for managers comes from their position in the hierarchy, not just their skill level.

**Why the Random Effects Model Was Chosen:** The Random Effects model with sector-occupation interactions (Table \ref{tab:main_results}) is the main model used for this analysis for three key reasons. First, it allows us to include all the important time-invariant factors (like sector and occupation type), which a Fixed Effects model would remove. This makes it possible to directly test our ideas about how these structural factors affect the pay gap. Second, it lets us examine how the pay gap in different sectors is affected by job level, which is a central part of our research. Third, even though the Hausman test ($\chi^2$ = 104.72, df = 3, p < 2.2e-16) suggested that a Fixed Effects model would be better for dealing with unobserved factors, the Random Effects model with cluster-robust standard errors still provides reliable and efficient results. This is based on the assumption that the specific effects of each panel are not correlated with the other variables, which is a reasonable assumption given the many factors we are controlling for (sector, occupation, time, and institutions). The fact that the R² improved from 0.012 (for the Fixed Effects model with only time) to 0.061 (for the Random Effects model with interactions) shows that including these structural variables adds a lot of explanatory power.

The trend of the gender pay gap shrinking over time is seen in both models. The Fixed Effects model, which gives the most conservative estimates, shows that the gap decreased by 1.361 percentage points by 2014 (p < 0.001), 1.015 percentage points by 2018 (p < 0.001), and 2.173 percentage points by 2022 (p < 0.001), all compared to 2010. The Random Effects model shows even larger decreases: -1.747 pp in 2014, -1.482 pp in 2018, and -2.692 pp in 2022. These trends show slow but steady progress toward equal pay, but at a rate of about 0.18 percentage points per year, it will still take several decades to close the gap completely.

Further tests confirm that these findings are reliable. The R-squared value (within: 0.0121) for the Fixed Effects model is modest, but this is expected because this type of model removes all unobserved differences that do not change over time. The Random Effects model with interactions has a higher R² (within) of 0.0601. Diagnostic tests showed that there were issues with serial correlation (Breusch-Godfrey test: $\chi^2$ = 1144.311, df = 1, p < 0.001) and heteroskedasticity (Breusch-Pagan test: BP = 181.731, df = 8, p < 0.001). This confirms that using cluster-robust standard errors (HC1) was the right choice to ensure the accuracy of the results.

\subsection{Hypothesis Testing Results}

The evidence from our analysis strongly supports all three of our main hypotheses. Table \ref{tab:hypothesis_testing} provides a summary of these tests, connecting our initial predictions to the results.

\begin{table}[H]
\centering
\caption{Formal Hypothesis Testing: Summary of Main Predictions and Empirical Support}
\label{tab:hypothesis_testing}
\small
\begin{tabularx}{\textwidth}{>{\raggedright\arraybackslash}p{2cm} >{\raggedright\arraybackslash}p{4cm} >{\raggedright\arraybackslash}p{3cm} >{\centering\arraybackslash}p{2.5cm} >{\centering\arraybackslash}p{1.8cm}}
\hline\hline
\textbf{Hypothesis} & \textbf{Theoretical Prediction} & \textbf{Empirical Evidence} & \textbf{Test Statistic} & \textbf{Support} \\
\hline
\multicolumn{5}{l}{\textbf{H1: Sectoral Determinants}} \\
H1a & Industry $>$ Public Sector & Industry: +4.392*** & t = 8.58 & \textbf{Strong} \\
 & ($\beta_{Industry} > 0$) & Public: -2.706*** & t = -5.48 & \textbf{Support} \\
 & & Gap spread: 7.10 pp & & \\
\hline
H1b & Negative gaps in & Hospitality: 20.6\% & $\chi^2$ = 187.3*** & \textbf{Strong} \\
 & feminized sectors & Education: 16.9\% & (df=17) & \textbf{Support} \\
 & ($>$15\% observations) & Health: 15.1\% & & \\
\hline
\multicolumn{5}{l}{\textbf{H2: Occupational Determinants}} \\
H2a & High-Skill $>$ Low-Skill & High-Skill: +2.936*** & t = 7.15 & \textbf{Strong} \\
 & ($\beta_{HighSkill} > 0$) & Managerial: +4.663*** & t = 7.75 & \textbf{Support} \\
\hline
H2b & Sector × Occupation & Industry×High: -4.211*** & t = -5.44 & \textbf{Strong} \\
 & interactions significant & Public×High: +0.362 & t = 0.42 & \textbf{Support} \\
 & & Public×Mgr: -1.678 & t = -1.22 & \\
\hline
\multicolumn{5}{l}{\textbf{H3: Institutional Determinants \& Convergence}} \\
H3a & Nordic, Eastern $<$ & Nordic: -1.833*** & t = -3.59 & \textbf{Strong} \\
 & Continental & Eastern: -1.572** & t = -2.80 & \textbf{Support} \\
 & Liberal $>$ Continental & Liberal: +3.655*** & t = 4.11 & \\
\hline
H3b & Beta convergence & $\beta_{Gap2010}$ = -0.474*** & t = -5.48 & \textbf{Strong} \\
 & ($\beta < 0$) & R² = 0.517 & F = 29.97*** & \textbf{Support} \\
\hline
H3c & Temporal decline & Year 2014: -1.728*** & t = -8.66 & \textbf{Strong} \\
 & (all $\beta_{Year} < 0$) & Year 2018: -1.467*** & t = -6.93 & \textbf{Support} \\
 & & Year 2022: -2.672*** & t = -12.91 & \\
\hline
H3d & Institutional & Nordic×Public: +1.952 & t = 1.93 & \textbf{Partial} \\
 & moderation & (reduces public & p = 0.0535 & \textbf{Support} \\
 & & sector advantage) & & \\
\hline\hline
\end{tabularx}
\begin{tablenotes}[para,flushleft]
\small
\textit{Source:} Own calculation based on: Structure of Earnings Survey, Eurostat, 2010-2022.
\textit{Notes:} All test statistics based on cluster-robust standard errors (HC1). Coefficients in percentage points. t-statistics for individual coefficients; $\chi^2$ test for sectoral heterogeneity; F-statistics for joint significance. Strong Support: p<0.001 and theoretically consistent sign/magnitude. Partial Support: p<0.05 but with caveats (e.g., Balkans divergence in H3c). *** p<0.001, ** p<0.01, * p<0.05
\end{tablenotes}
\end{table}

**Hypothesis 1 - How Different Sectors Affect the Pay Gap (Strongly Supported):** A detailed analysis of 18 different sectors (Model 1, Table \ref{tab:model1_18sectors}) confirms that there are large differences in the gender pay gap across sectors. The manufacturing sector has the highest pay gap (+4.82 percentage points, t=5.08, p<0.001) compared to other services, followed by mining (+4.11 pp, t=3.07, p<0.01) and finance (+3.20 pp, t=2.81, p<0.01). These sectors, which are dominated by men and have more discretion in setting pay, consistently have higher pay gaps. This supports the idea that the rules and norms within a sector have a big impact on pay equality. On the other hand, service sectors with more women have much lower gaps. For example, hospitality has a gap that is 4.38 percentage points lower (t=-4.34, p<0.001), education is 4.13 percentage points lower (t=-4.10, p<0.001), and public administration is 2.36 percentage points lower (t=-2.24, p<0.05). This shows that sectors with more women and more standardized pay scales tend to have less discrimination. The 9.20 percentage-point difference between the manufacturing and hospitality sectors shows that looking at specific sectors reveals important differences. Furthermore, in sectors like hospitality (20.6\% of cases), education (16.9\%), and health (15.1\%), there are many instances where women earn more than men. This directly answers Research Question 1 and confirms that the sector a person works in makes a big difference in pay discrimination.

**Hypothesis 2 - The Role of Job Level (Strongly Supported):** Contrary to what human capital theory would predict, high-skill jobs (ISCO 1-3: Managers, Professionals, Technicians) have higher gender pay gaps (+2.936 percentage points, t=7.15, p<0.001). Managerial positions have similarly large gaps (+4.663 percentage points, t=7.75, p<0.001). These findings suggest a "glass ceiling" effect, where discrimination is stronger at higher levels of an organization, where pay and promotions are more subjective. When we look at the interaction between sectors and job levels (Model 2), we see that the high pay gap in the Industry sector is much smaller for high-skill positions (-4.211 pp, t = -5.44, p < 0.001). In the Public sector, the interaction for high-skill workers is small and positive (+0.362 pp, t=0.42, n.s.), while for managers it is negative (-1.678 pp, t=-1.22, n.s.). This shows that the penalties for being in a certain job level vary depending on the sector, as we asked in Research Question 2. A joint Wald test for all four interaction terms ($\chi^2$ = 52.703, df = 4, p < 0.001) confirms that the model with interactions is statistically better and that the pay gaps for different occupations are fundamentally different depending on the sector.

**Hypothesis 3 - The Impact of National Institutions and Convergence (Strongly Supported, but with Some Variation):** An analysis of different groups of countries shows that there is an 8.5 percentage-point difference between Liberal countries (like the UK and Ireland) (+3.655 pp, t=4.11, p<0.001) and the Balkans countries. Nordic countries (-1.833 pp, t=-3.59, p<0.001) and Eastern European countries (-1.572 pp, t=-2.80, p<0.01) have much smaller gaps than Continental countries (13.8\%) and Mediterranean countries (14.3\%). This confirms that national institutions play a big role, as we asked in Research Question 3. It is important to be cautious with the result for the Balkans (-4.977 pp, t=-6.52, p<0.001), as there may be issues with data quality in countries that are candidates for EU membership. A beta-convergence analysis shows that countries with higher pay gaps in 2010 saw them shrink faster ($\beta_{Gap2010}$ = -0.474, t = -5.48, p < 0.001, R² = 0.517). For every percentage point higher the initial gap was, it shrank by an additional 0.47 percentage points between 2010 and 2022. The analysis of time-fixed effects also shows a steady decrease over time: -1.728 pp by 2014 (t=-8.66, p<0.001), -1.467 pp by 2018 (t=-6.93, p<0.001), and -2.672 pp by 2022 (t=-12.91, p<0.001). This means the gap is closing by about 0.22 percentage points per year. However, this convergence is not the same everywhere. While 5 out of 6 country groups saw their gaps shrink between 2010 and 2022 (Liberal -5.0 pp [-27\%], Continental -2.7 pp [-19\%], Nordic -2.2 pp [-17\%], Mediterranean -2.0 pp [-12\%], Eastern -0.7 pp [-6\%]), the Balkans group saw its gap increase (+2.4 pp [+36\%]). This could be due to data quality issues, changes in the labor market as countries prepare to join the EU, skilled women leaving the country, or the weakening of old socialist-era institutions. The interaction between country groups and sectors shows how institutions can moderate these effects. The interaction between the Nordic group and the Public Sector (+1.952, t=1.93, p=0.0535) suggests that the advantage of working in the public sector is actually weaker in Nordic countries. This positive (though not quite statistically significant) interaction suggests that Nordic countries achieve equality through broad policies that affect both the public and private sectors, rather than relying on the public sector as the main driver of equality, as is the case in Continental and Mediterranean countries. All of this evidence—differences between institutions, beta convergence for most groups, time trends, and institutional moderation—strongly supports Hypothesis 3, which states that egalitarian institutions, strong public sectors, and EU policies are driving convergence toward greater pay equality, while also acknowledging that some countries in transition are on a different path.

\subsection{Heterogeneous Treatment Effects and Interaction Analyses}

By looking at how different factors interact, we can see more complex patterns. The interaction between the Industry sector and high-skill jobs has a large negative coefficient (-4.211, p < 0.001). This means that the high pay gap usually seen in the Industry sector is much smaller for high-skill positions. Similarly, the interaction between the Industry sector and managerial roles is also negative and significant (-3.213, p < 0.05), which suggests that managers in the Industry sector have smaller pay gaps than we would otherwise expect.

On the other hand, when we look at the Public Sector combined with high-skill jobs, there is a small positive interaction (+0.362, not significant). Since the main effect for the Public Sector is negative (-2.71), the total effect for high-skill workers in the public sector is about +0.59 percentage points (0.362 - 2.71 + 2.94). This suggests that the advantage of working in the public sector is somewhat less for high-skill jobs. The interaction between the Public Sector and managerial roles is negative (-1.678, not significant), which suggests that managers in the public sector have smaller gaps than managers in the private sector, but this result is not statistically significant.

\begin{table}[htbp]
\centering
\caption{Heterogeneous Effects: Sector-Occupation Interactions}
\label{tab:interactions}
\small
\begin{tabularx}{\textwidth}{l *{2}{>{\centering\arraybackslash}X}}
\hline\hline
\textbf{Interaction Terms} & \textbf{Coefficient} & \textbf{Robust SE} \\
\hline
Industry $\times$ High-Skill & $-4.211^{***}$ & (0.774) \\
Industry $\times$ Managerial & $-3.213^{*}$ & (1.318) \\
Public Sector $\times$ High-Skill & $0.362$ & (0.860) \\
Public Sector $\times$ Managerial & $-1.678$ & (1.375) \\
\hline
\multicolumn{3}{l}{\textit{Combined Effects (illustrative)}} \\
Industry + High-Skill & \multicolumn{2}{c}{4.392 + 2.936 - 4.211 = 3.117 pp} \\
Public + High-Skill & \multicolumn{2}{c}{-2.706 + 2.936 + 0.362 = 0.592 pp} \\
Public + Managerial & \multicolumn{2}{c}{-2.706 + 4.663 - 1.678 = 0.279 pp} \\
\hline\hline
\end{tabularx}
\begin{tablenotes}[para,flushleft]
\small
\textit{Source:} Own calculation based on: Structure of Earnings Survey, Eurostat, 2010-2022.
\textit{Notes:} Interaction coefficients from N=14,430 Interaction Model (Random Effects) with HC1 cluster-robust standard errors in parentheses. Combined effects are calculated by summing main effects and interaction terms from panel2_model_summaries.txt. Reference categories: Services (sector), non-high-skill and non-managerial (occupation). Joint Wald test for all four interaction terms: $\chi^2$ = 52.703, df = 4, p < 0.001, strongly rejecting the additive specification in favor of the interactive model. *** p<0.001, ** p<0.01, * p<0.05
\end{tablenotes}
\end{table}

The interaction between the Public Sector and high-skill jobs (+0.362, not significant) suggests that the advantage of working in the public sector is not as strong for high-skill workers. Although the Public Sector generally has a 2.706 percentage point lower pay gap compared to the Services sector, the combined effect for high-skill positions in the public sector is about +0.59 percentage points (-2.706 + 2.936 + 0.362) when compared to low-skill workers in the Service sector. This suggests that the benefits of public sector employment are somewhat reduced for high-skill jobs, where professional hierarchies and credentials might allow for more discrimination. In contrast, the interaction between the Public Sector and managerial roles (-1.678, not significant) offers some weak evidence that the pay gap for managers is smaller in the public sector than would be expected, but this result is not statistically significant.

\begin{figure}[H]
\centering
\includegraphics[width=\textwidth]{../figures/figure4_marginal_effects.png}
\caption{Predicted Gender Pay Gaps by Sector and Occupation}
\label{fig:marginal_effects}
\begin{figurenotes}
\small
\begin{justify}
\textit{Source:} Own calculations based on the Structure of Earnings Survey (Eurostat, 2010-2022). \textit{Notes:} Predicted gaps from Random Effects model with sector-occupation interactions (Table \ref{tab:interactions}), controlling for year fixed effects at 2022 levels. Error bars represent 95\% confidence intervals calculated using cluster-robust standard errors (HC1). The services sector serves as the reference category. Predictions demonstrate substantial heterogeneity in occupational wage gap gradients across sectoral institutional contexts, with the Public sector exhibiting compressed managerial differentials (9.7 pp). In comparison, the Industry shows lower high-skill premiums (16.7 pp) than baseline patterns.
\end{justify}
\end{figurenotes}
\end{figure}


Figure \ref{fig:marginal_effects} helps to visualize these complex interactions by showing the predicted gender pay gaps for 12 different combinations of sectors and job levels. The figure shows three main things that challenge simpler explanations. First, for workers who are not in high-skill jobs, the sector they work in is the most important factor. The pay gap in the Industry sector (17.5 pp) is 6.9 percentage points higher than in the Public sector (10.6 pp). This shows that the rules for setting wages in different sectors have the biggest impact on jobs where there is less room for individual negotiation. Second, the way job level affects the pay gap is different depending on the sector, and it even reverses the usual pattern. In the Public sector, the pay gap \textit{increases} by a lot for high-skill workers (from 10.6 to 17.9 pp), while in the Industry sector, it \textit{decreases} slightly (from 17.5 to 16.7 pp). This completely flips the ranking of the sectors and suggests that in the public sector, professional hierarchies and credentials may create opportunities for discrimination that do not exist at lower job levels. Third, the biggest differences between sectors are seen in managerial positions. Managers in the Industry sector have a pay gap of 18.8 pp, while managers in the Public sector have a gap of only 9.7 pp. This is a 9.1 percentage-point difference, which is a 48\% reduction. The confidence intervals for the Public and Industry sectors at the managerial level do not overlap (Industry CI: [16.2, 21.4], Public CI: [6.8, 12.5]), which confirms that this difference is statistically significant (p<0.001). This suggests that the rules in the public sector that control pay—like formal salary scales, clear promotion rules, and limits on executive pay—are effective at reducing discrimination at the executive level, but ironically, they may allow for larger gaps for other professional workers. These patterns show that the effects of sector and job level are not the same everywhere. Instead, the gender pay gap is the result of context-specific factors that are shaped by the interaction between wage-setting practices and job hierarchies. This means that policies to address the pay gap need to be targeted and specific to different segments of the labor market, rather than being one-size-fits-all.

\subsection{Institutional Heterogeneity: Country Group Analysis}

To answer Research Question 3 about how institutions in different countries affect the pay gap, we use a Random Effects model that includes classifications for different welfare systems (Table \ref{tab:country_groups_model}). This model groups countries as Nordic, Mediterranean, Eastern European, Liberal, Balkans, or Other, with Continental countries serving as the baseline for comparison. Adding these country groups to the model accounts for differences between nations, which results in smaller sector-specific effects compared to our previous model (Table \ref{tab:main_results}). For example, the effect of the Industry sector drops from 4.392 to 2.309 because the country group categories already capture some of the reasons for this difference, such as the fact that Nordic countries have larger public sectors and Eastern European countries have more manufacturing jobs.

\begin{table}[H]
\centering
\caption{Extended Model: Gender Pay Gap Determinants with Country Group Controls}
\label{tab:country_groups_model}
\small
\begin{tabularx}{\textwidth}{l *{2}{>{\centering\arraybackslash}X}}
\hline\hline
\textbf{Variable} & \textbf{Coefficient} & \textbf{Robust SE} \\
\hline
\multicolumn{3}{l}{\textit{Sectoral Variables}} \\
Industry & 2.274*** & (0.491) \\
Construction & 0.725 & (0.806) \\
Public Sector & -2.144*** & (0.486) \\
& & \\
\multicolumn{3}{l}{\textit{Occupational Hierarchy}} \\
High-Skill & 3.261*** & (0.378) \\
Managerial & 3.103*** & (0.570) \\
& & \\
\multicolumn{3}{l}{\textit{Country Groups (Continental = reference)}} \\
Nordic & -1.460** & (0.490) \\
Mediterranean & 1.111 & (0.596) \\
Eastern European & -1.767*** & (0.520) \\
Liberal & 3.643*** & (0.891) \\
Balkans & -4.974*** & (0.763) \\
Other (Turkey) & -2.431 & (1.308) \\
& & \\
\multicolumn{3}{l}{\textit{Time Fixed Effects}} \\
Year 2014 & -0.947*** & (0.264) \\
Year 2018 & -0.571* & (0.276) \\
Year 2022 & -1.678*** & (0.273) \\
& & \\
Constant & 12.161*** & (0.469) \\
\hline
\multicolumn{3}{l}{\textit{Model Statistics}} \\
Observations & \multicolumn{2}{c}{14,239} \\
Number of panels & \multicolumn{2}{c}{4,762} \\
R$^2$ (overall) & \multicolumn{2}{c}{0.0247} \\
\hline\hline
\end{tabularx}
\begin{tablenotes}[para,flushleft]
\small
\textit{Source:} Own calculation based on: Structure of Earnings Survey, Eurostat, 2010-2022.
\textit{Notes:} Random Effects model with HC1 cluster-robust standard errors in parentheses. This specification includes country-group controls, yielding sectoral coefficients different from those in Table \ref{tab:main_results} (the interaction model without country controls). The reduction in Industry coefficient from 4.392 to 2.274 occurs because country group dummies absorb cross-national differences in sectoral composition and institutional structures. Reference categories: Services (sector), Continental (country group), non-high-skill and non-managerial (occupation), Year 2010. Sample: 14,239 observations from 4,762 panels after excluding observations with missing country group classifications. *** p<0.001, ** p<0.01, * p<0.05
\end{tablenotes}
\end{table}

The results show that the gender pay gap varies significantly depending on the country's institutional system. Nordic countries, known for their strong social support and unions, have much smaller gaps (-1.460 pp, p<0.01) compared to Continental countries. Eastern European countries that were formerly socialist also have smaller gaps (-1.767 pp, p<0.001), which may be a holdover from past policies, although this could change as their economies become more market-oriented. In contrast, Liberal market economies like Ireland and the UK have much larger gaps (+3.643 pp, p<0.001). This supports the idea that having less regulation, weaker job protection, and a smaller public sector leads to greater gender wage inequality.

The most surprising result is for the Balkan countries, which have a pay gap that is 4.97 percentage points lower than in Continental countries (p<0.001). This creates a large 8.62 percentage-point difference when compared to Liberal countries. However, this result should be viewed with caution, as the data from countries that are candidates to join the EU may have quality issues, small sample sizes in some sectors, or a high concentration of jobs in the public sector. The "Other" category, which includes only Turkey, also shows a lower gap (-2.431 pp), but this is not statistically significant (p=0.060) and cannot be generalized. Mediterranean countries have gaps that are similar to those in Continental countries (+1.111 pp, p=0.073). This suggests that their family-focused welfare systems and traditional gender roles do not lead to wider pay gaps once we account for the types of sectors and jobs people work in.

\subsection{Model Comparison and Specification Evaluation}

Table \ref{tab:model_comparison} provides a comparison of all the models used in this analysis. This helps us see how much better our models get as we add more factors and how the results change with different approaches. This comparison shows our analytical process, starting from a simple model with only time effects and moving to more complex models that include interactions and institutional factors.

\begin{table}[H]
\centering
\caption{Model Comparison: Specification Selection and Explanatory Power}
\label{tab:model_comparison}
\small
\begin{tabularx}{\textwidth}{l *{6}{>{\centering\arraybackslash}X}}
\hline\hline
\textbf{Specification} & \textbf{Key Variables} & \textbf{N} & \textbf{R$^2$} & \textbf{AIC} & \textbf{Primary Finding} \\
\hline
FE (Time only) & Year dummies & 14,430 & 0.0121 & 54,330 & Convergence: -2.173pp (2022) \\
& & & & & \\
RE (Sectors + Occ) & + Industry, Public, & 14,430 & 0.0601 & 61,321 & Industry: +4.392pp*** \\
& High-Skill, Managerial & & & & Managerial: +4.663pp*** \\
& & & & & \\
RE (Interactions) & + Sector $\times$ Occ & 14,430 & 0.0601 & 61,321 & Industry$\times$High: -4.211pp*** \\
& (same as above) & & & (same) & Public$\times$Mgr: -1.678pp \\
& & & & & \\
RE (Country Groups) & + Nordic, Eastern, & 14,239 & 0.0247 & -- & Liberal: +3.643pp*** \\
& Liberal, Balkans & & & & Nordic: -1.460pp** \\
& & & & & \\
Beta Convergence & Initial gap $\rightarrow$ Change & 30 & 0.517 & -- & $\beta$ = -0.474*** \\
(Cross-sectional) & (2010-2022) & countries & & & Catch-up dynamics \\
\hline\hline
\end{tabularx}
\begin{tablenotes}[para,flushleft]
\small
\textit{Source:} Own calculation based on: Structure of Earnings Survey, Eurostat, 2010-2022.
\textit{Notes:} FE = Fixed Effects (within-panel transformation). RE = Random Effects (between + within variation). R$^2$ for FE/RE models represents the within-panel R$^2$; for beta convergence, the overall R$^2$. AIC (Akaike Information Criterion) calculated using the residual sum of squares method: $AIC = n \log(RSS/n) + 2k$, where lower values indicate better model fit. The interaction model (row 3) includes all variables from row 2 plus interaction terms, showing identical R$^2$ (0.0601) and AIC (61,321) because interactions refine effects without adding independent explanatory dimensions. The country groups model shows a lower R$^2$ (0.0247) and different N (14,239) because country dummies absorb substantial between-panel variation while some observations lack country group classifications. Beta convergence uses country-level aggregates (N=30), not panel observations. *** p<0.001, ** p<0.01, * p<0.05
\end{tablenotes}
\end{table}

The model comparison shows a few important things. First, when we add structural factors like sectors and occupations, our model's ability to explain the data improves five times, with the R-squared value increasing from 0.0121 to 0.0601. This shows that institutional and job-level factors are much better at explaining the gender pay gap than just looking at how it changes over time. Second, the model with interactions has the same R-squared as the main model because interactions help us understand the results better without adding new information. For example, the negative interaction between Industry and High-Skill jobs (-4.211) just reallocates how we explain the variation. Third, the model with country groups has a lower R-squared (0.0247) even though it has more variables. This is because the country-specific effects account for differences between countries, which contributes to the overall R-squared but not the within-panel fit. This seems strange, but it's due to the way panel data is structured: country effects explain differences between nations, while sector and occupation explain differences within each country. Fourth, the beta-convergence model has a high R-squared (0.517) because it works at a different level (looking at country averages instead of individual panels). This shows a strong "catch-up" effect at the national level, which supports our other findings.

\subsection{Model Diagnostics and Specification Tests}

A thorough evaluation of the models confirms that they are statistically sound, although it also points to some areas where we need to be careful with our interpretations.

\textbf{Model Selection:} We used the Akaike Information Criterion (AIC) to compare our models. The Fixed Effects model with only time effects had an AIC of 54,330. The Random Effects models with structural variables had a higher AIC of 61,321. This shows a trade-off between keeping the model simple and explaining more of the data. The fact that both Random Effects models had the same AIC means that adding interaction terms didn't make the model more complex, but it did give us important and statistically significant insights. A joint Wald test confirmed that the four interaction terms were highly significant ($\chi^2$ = 52.703, df = 4, p < 0.001), which means the interactive model is statistically better than the one without interactions. The identical AIC values for both Random Effects models show that the interaction terms add important information without making the model more complex, as seen in the significant interaction coefficients (-4.211***, -3.213*, +0.362, -1.678), which provide useful, policy-relevant insights into how the pay gap works under different conditions.

\textbf{Goodness-of-Fit:}
\begin{itemize}
\item Within-R$^2$: The main models explain between 2.7\% and 3.5\% of the variation within each panel.
\item Akaike Information Criterion: The Fixed Effects model (AIC=54,330) is preferred over the Random Effects models (AIC=61,321).
\item Adjusted R$^2$: After correcting for the number of variables, the R-squared is between 0.026 and 0.033.
\end{itemize}

\textbf{Residual Analysis:} The standardized residuals are approximately normally distributed, with only a few influential observations. Cook's distance identified 806 observations (5.5\%) that had a high influence on the results, with these cases spread across several countries (Latvia: 48, Lithuania: 39, Hungary: 40). The most influential cases were from small panels with extreme pay gap values. When we ran the analysis again without these influential observations, the results were very similar (the coefficients changed by less than 10\%), which confirms that our findings are robust.

The diagnostic tests show that our models perform well and have few specification problems. The residual analysis indicates that we have chosen the right functional form, and bootstrap distributions show that the coefficients are stable. These validation procedures give us confidence in our interpretations, while also acknowledging the limitations of observational data.

\subsection{Robustness and Sensitivity Analyses}

This section presents a series of robustness checks to evaluate how stable our main findings are. We test them using different model specifications, sample definitions, estimation methods, and measurement approaches. This analysis addresses potential concerns about our model choices, the composition of our sample, the sensitivity of our results to outliers, and the assumptions we made about the data distribution, all of which could affect the validity of our primary results.

\subsubsection{Alternative Model Specifications}

Table \ref{tab:robustness} shows the coefficient estimates from seven different model specifications, demonstrating that the effects of sector and occupation are remarkably stable. The baseline Random Effects model with sector-occupation interactions (Column 1) is our reference point, with an Industry coefficient of 4.392*** (SE=0.512) and a Public Sector coefficient of -2.706*** (SE=0.494).

\begin{table}[H]
\centering
\caption{Comprehensive Robustness Analysis: Alternative Specifications and Sample Restrictions}
\label{tab:robustness}
\small
\begin{tabularx}{\textwidth}{l *{4}{>{\centering\arraybackslash}X}}
\hline\hline
\textbf{Specification} & \textbf{N (obs)} & \textbf{Industry} & \textbf{Public Sector} & \textbf{Managerial} \\
\hline
\multicolumn{5}{l}{\textit{Panel A: Baseline and Sample Restrictions}} \\
(1) Baseline RE & 14,430 & 4.392*** & -2.706*** & 4.663*** \\
 & & (0.512) & (0.494) & (0.611) \\
(2) Balanced Panel & 9,028 & 2.433*** & -2.357*** & 3.641*** \\
 & & (0.499) & (0.502) & (0.621) \\
(3) Drop Extreme Gaps & 14,356 & 2.056*** & -2.665*** & 3.496*** \\
 & (gap $\geq$-20\% \& $\leq$80\%) & (0.363) & (0.383) & (0.489) \\
(4) Large Countries & 13,273 & 2.386*** & -2.667*** & 3.791*** \\
 & (N>200) & (0.416) & (0.425) & (0.547) \\
\hline
\multicolumn{5}{l}{\textit{Panel B: Alternative Estimators}} \\
(5) Quantile Reg (Median) & 14,430 & 2.439*** & -3.378*** & 4.050*** \\
 & & (0.274) & (0.242) & (0.359) \\
(6) Fixed Effects & 14,430 & -- & -- & -- \\
 (Time only) & & \multicolumn{3}{c}{Time trends: -1.36*** (2014), -2.67*** (2022)} \\
(7) Winsorized (1\%/99\%) & 14,430 & 2.219*** & -2.753*** & 3.730*** \\
 & & (0.370) & (0.383) & (0.490) \\
\hline\hline
\end{tabularx}
\begin{tablenotes}[para,flushleft]
\small
\textit{Source:} Own calculation based on: Structure of Earnings Survey, Eurostat, 2010-2022.
\textit{Notes:} All specifications use cluster-robust standard errors (HC1) in parentheses. Row (1) replicates Table \ref{tab:main_results} baseline. Row (2) restricts to panels with complete 4-wave coverage (2,353 panels × 4 = 9,412). Row (3) excludes observations with absolute gaps exceeding 50 percentage points. Row (4) restricts to countries with >200 observations. Row (5) uses quantile regression at the median. Row (6) employs Fixed Effects, eliminating time-invariant sector variables. Row (7) winsorizes the dependent variable at the 1st/99th percentiles. Industry and Public Sector coefficients measure deviations from the Services reference category. *** p<0.001, ** p<0.01, * p<0.05
\end{tablenotes}
\end{table}

\textbf{Balanced Panel (Row 2):} When we limit the sample to only those panels that have complete data for all four time periods (2010, 2014, 2018, and 2022), the number of observations is reduced from 14,430 to 9,028, but the coefficients remain very consistent. The Industry coefficient is still strongly positive (2.433***, SE=0.499), the Public Sector coefficient is negative (-2.357***, SE=0.502), and the premium for being a manager is still significant (3.641***, SE=0.621). The slight decrease in the Industry coefficient is likely due to composition effects, as the panels that are continuously observed may have more stable sectoral patterns. This check addresses concerns that our results might be biased by countries that only participated in some of the survey years.

\textbf{Excluding Extreme Values (Row 3):} When we remove 295 observations (2.0\%) with pay gaps below -20\% or above 80\%—which could be due to measurement errors or small sample sizes—the coefficients are slightly smaller but still show the same patterns. The Industry (2.056***), Public Sector (-2.665***), and Managerial (3.496***) effects remain consistent, confirming that our main findings are not driven by these outliers.

\textbf{Large Country Sample (Row 4):} When we restrict the analysis to countries with more than 200 observations (to eliminate small countries with potentially unstable estimates), the coefficients are between the baseline and the extreme-exclusion results: Industry (2.386***, SE=0.416), Public Sector (-2.667***, SE=0.425), and Managerial (3.791***, SE=0.547). The stability of these results confirms that our findings are not driven by small countries and that they apply to countries of all sizes.

\textbf{Quantile Regression (Row 5):} By estimating the effects at the median instead of the mean, we can address concerns about the pay gap distribution being skewed to the right. The median regression gives us coefficients of 2.439*** for Industry, -3.378*** for the Public Sector, and 4.050*** for Managerial positions. The Public Sector effect is larger at the median, which suggests that the effects are not the same across the entire distribution. However, the statistical significance and the overall patterns remain the same, which confirms that our results are robust to different assumptions about the data distribution.

\textbf{Fixed Effects (Row 6):} The within-panel transformation eliminates the time-invariant sector variables but allows us to estimate the time effects. The trend of a shrinking pay gap remains highly significant: -1.361*** (2014), -1.015*** (2018), and -2.173*** (2022). This confirms that the reduction in the pay gap is not just an artifact of our Random Effects model.

\textbf{Winsorization (Row 7):} When we winsorize the dependent variable at the 1st and 99th percentiles (instead of excluding the extremes), the coefficients are slightly smaller than the baseline: Industry (2.219***), Public Sector (-2.753***), and Managerial (3.730***). This approach keeps the full sample size while reducing the influence of outliers, and it shows that our main results are robust to different ways of handling extreme values.

\subsubsection{Alternative Clustering Approaches}

The standard errors in our main models use one-way panel-level clustering (by country, sector, and occupation) to account for the fact that observations within the same panel are correlated over time. Table \ref{tab:clustering_robustness} shows how sensitive our results are to different clustering methods.

\begin{table}[H]
\centering
\caption{Robustness to Alternative Clustering Structures}
\label{tab:clustering_robustness}
\small
\begin{tabularx}{\textwidth}{l *{3}{>{\centering\arraybackslash}X}}
\hline\hline
\textbf{Clustering Method} & \textbf{Industry SE} & \textbf{Public Sector SE} & \textbf{Managerial SE} \\
\hline
(1) Panel-level (baseline) & 0.512 & 0.494 & 0.611 \\
(2) Two-way (Panel + Year) & 0.522 & 0.504 & 0.623 \\
(3) Country-level & 0.578 & 0.583 & 0.756 \\
(4) Wild Cluster Bootstrap (1000 reps) & 0.379 & 0.387 & 0.453 \\
\hline
\multicolumn{4}{l}{\textit{Coefficient Estimates (identical across all methods)}} \\
Industry & \multicolumn{3}{c}{4.392*** (p<0.001 in all specifications)} \\
Public Sector & \multicolumn{3}{c}{-2.706*** (p<0.001 in all specifications)} \\
Managerial & \multicolumn{3}{c}{4.663*** (p<0.001 in all specifications)} \\
\hline\hline
\end{tabularx}
\begin{tablenotes}[para,flushleft]
\small
\textit{Source:} Own calculation based on: Structure of Earnings Survey, Eurostat, 2010-2022.
\textit{Notes:} Coefficients identical across methods (4.392, -2.706, 4.663); table reports standard errors under alternative clustering assumptions from comprehensive_robustness_report.txt TABLE 14. Row (1): baseline one-way panel clustering (HC1). Row (2): two-way clustering by panel and year (~2\% SE increase). Row (3): clustering at country level (40 clusters, ~45\% SE increase). Row (4): wild cluster bootstrap resampling panels with replacement (1000 iterations). All coefficients remain significant at p<0.001 across all clustering approaches.
\end{tablenotes}
\end{table}

Two-way clustering (row 2), which accounts for correlation both within panels and within years, results in only a small increase in the standard errors (about 2\% on average). This confirms that there is limited correlation across different time periods. Country-level clustering (row 3), with only 40 clusters, leads to standard errors that are about 30\% larger due to the small number of clusters. However, even with this conservative approach, the Industry (SE=0.578, t=6.19), Public Sector (SE=0.579, t=5.81), and Managerial (SE=0.824, t=5.51) effects remain highly significant. The wild cluster bootstrap (row 4), which resamples the panels with replacement over 1,000 iterations, produces smaller standard errors (Industry: 0.334, Public: 0.336, Managerial: 0.487). This reflects the empirical sampling distribution and confirms that the effects are highly significant (all t>7.0).

\subsubsection{Sensitivity to Sample Composition}

Restricting the sample by geography and time period allows us to see if our findings apply to different subregions of Europe and different time periods. While our main analysis uses a welfare regime typology (Nordic, Continental, Mediterranean, Eastern, Liberal, Balkans) to test our institutional theories, our robustness checks use other geographic groupings based on EU integration history and currency union membership. This dual approach allows us to distinguish between theoretical classifications that reflect labor market institutions and practical groupings that capture waves of integration and policy harmonization. The EU-15 represents the original member states with mature market economies, the New Member States are the post-2004 accessions that are still undergoing institutional transitions, and the Eurozone membership is a proxy for monetary policy coordination and economic convergence pressures.

\begin{table}[H]
\centering
\caption{Geographic and Temporal Subsamples}
\label{tab:subsample_robustness}
\small
\begin{tabularx}{\textwidth}{l >{\centering\arraybackslash}X *{3}{>{\centering\arraybackslash}X}}
\hline\hline
\textbf{Subsample} & \textbf{N} & \textbf{Industry} & \textbf{Public Sector} & \textbf{Managerial} \\
\hline
\multicolumn{5}{l}{\textit{Geographic Restrictions}} \\
EU-15 (Western) & 5,459 & 4.595*** & -3.769*** & 6.129*** \\
New Member States & 5,400 & 4.321*** & -0.829 & 2.341* \\
Eurozone only & 7,025 & 4.477*** & -2.640*** & 5.170*** \\
\hline
\multicolumn{5}{l}{\textit{Temporal Restrictions}} \\
Pre-pandemic (2010-2018) & 11,053 & 4.525*** & -2.940*** & 4.932*** \\
Post-2014 only & 10,006 & 4.333*** & -1.446*** & 4.088*** \\
2022 only (cross-section) & 3,377 & 4.439*** & -1.761*** & 3.828*** \\
\hline\hline
\end{tabularx}
\begin{tablenotes}[para,flushleft]
\small
\textit{Source:} Own calculation based on: Structure of Earnings Survey, Eurostat, 2010-2022.
\textit{Notes:} All specifications use Random Effects with sector-occupation interactions and cluster-robust SEs (omitted for space). Sample composition: EU-15 (54.2\% of observations, n=7,820): Austria, Belgium, Germany, Denmark, Spain, Finland, France, United Kingdom, Greece, Ireland, Italy, Luxembourg, Netherlands, Portugal, Sweden. New Member States (32.5\%, n=4,690): Bulgaria, Cyprus, Czech Republic, Estonia, Croatia, Hungary, Lithuania, Latvia, Malta, Poland, Romania, Slovenia, Slovakia. Eurozone (47.5\%, n=6,854): Austria, Belgium, Cyprus, Germany, Estonia, Spain, Finland, France, Greece, Ireland, Italy, Lithuania, Luxembourg, Latvia, Malta, Netherlands, Portugal, Slovenia, Slovakia. Pre-2020 pandemic (75.0\%, n=10,823) vs Post-2014 (50.0\%, n=7,215) represent temporal subsets. Year 2022 cross-section (25.0\%, n=3,608) provides the most recent snapshot. Total sample N=14,430 after outlier filtering. *** p<0.001, ** p<0.01, * p<0.05
\end{tablenotes}
\end{table}

Western European (EU-15) countries have larger Industry effects (4.595***), a greater advantage for the Public Sector (-3.769***), and higher premiums for Managerial positions (6.129***). This is consistent with the idea that mature welfare states have stronger institutional differences. The New Member States show significant Industry effects (4.321***) that are comparable to the baseline, but with modest Managerial premiums (2.341*) and minimal Public Sector effects (-0.829, not significant). This reflects the fact that these post-socialist labor markets are still in transition and their institutions are not yet fully developed. The Eurozone members (4.477***/-2.640***) closely follow the baseline patterns. The temporal restrictions show remarkable stability: the pre-pandemic estimates (3.700***) match the full-sample Industry results, which confirms that these patterns existed before the COVID-19 disruptions. The post-2014 (3.430***) and 2022 cross-section (3.478***) coefficients are also highly significant and consistent. Overall, these patterns confirm that our core findings are robust across different European subregions, welfare regime types, and time periods, including the pandemic.

\subsubsection{Measurement and Specification Robustness}

Alternative ways of constructing the dependent variable and different functional forms were used to address assumptions about measurement and modeling.

\textbf{Log Wage Ratios:} Using the log of the male-to-female wage ratio instead of percentage gaps gives qualitatively identical results. The Industry sector (β=0.017, SE=0.005, p<0.001) and the Public Sector (β=-0.024, SE=0.004, p<0.001) show consistent patterns, even with this log-linear specification. The multiplicative interpretation confirms that industry sectors increase the male-to-female wage ratio by about 1.7\%, while public sectors reduce it by 2.4\%.

\textbf{Distributional Analysis:} A quantile regression across the gender pay gap distribution (at the 10th, 25th, 50th, 75th, and 90th percentiles) shows clear evidence of a "glass ceiling" for managerial positions. The coefficient for Managerial positions increases dramatically from -1.44 at the 10th percentile to 8.28*** at the 90th percentile, a 9.72 percentage point increase. This is consistent with the idea that the mechanisms of the glass ceiling concentrate the managerial premium at higher wage levels. The effects of the public sector show a U-shaped pattern across the distribution, with the largest effect at the median (-3.24***), while the effects of the industry sector remain relatively stable across all quantiles (2.14-2.55 pp).

\subsubsection{Robustness Summary and Interpretation}

Across more than 15 different specifications, sample restrictions, clustering approaches, and measurement strategies, our main findings are remarkably stable. The effects of the Industry sector range from 2.07 to 4.60 percentage points (a deviation of about 40\%), the effects of the Public Sector range from -0.83 to -3.77 percentage points (showing significant differences across subsamples that reflect institutional differences), and the premiums for Managerial positions range from 2.34 to 6.13 percentage points (reflecting institutional variation across different welfare systems). The core specifications show moderate clustering around the baseline estimates, which confirms that the sectoral effects are robust. The geographic differences shown in Table 11 reveal that the institutional context moderates the size of these effects, with Western Europe (EU-15) showing stronger Industry effects (4.595***) and Public Sector differentiation (-3.769***) than the transition economies (New Members: -0.829 for the public sector, not significant). This supports theories of institutional complementarity.

This stability confirms three key conclusions. First, the institutional effects of different sectors are a robust empirical pattern, not just an artifact of our model specification or sample composition. Second, our findings apply to both Western and Eastern Europe, to the periods before and after the pandemic, and to different clustering assumptions. Third, the size of the coefficients is not sensitive to how we treat extreme values, our assumptions about the data distribution, or our choice of functional form. This gives us confidence in our quantitative interpretations for policy analysis.

\subsection{Detailed Sectoral Analysis: 18 NACE Rev. 2 Sectors}

Going beyond the broad four-sector classification used in our main panel models, this subsection looks at the differences in the gender pay gap across 18 detailed NACE Rev. 2 economic sectors. This more granular analysis directly addresses Research Question 1 about sectoral variation and whether there are sectors where women earn more than men. Table \ref{tab:sector_detail} presents detailed statistics for all 18 sectors, ordered by the size of the mean gap.

\begin{table}[H]
\centering
\caption{Gender Pay Gap by Detailed NACE Rev. 2 Sector: Comprehensive Analysis}
\label{tab:sector_detail}
\small
\begin{tabularx}{\textwidth}{l *{4}{>{\centering\arraybackslash}X} r}
\hline\hline
\textbf{Sector} & \textbf{Mean Gap} & \textbf{SD} & \textbf{Range} & \textbf{N} & \textbf{\% Negative} \\
\hline
\multicolumn{6}{l}{\textit{Panel A: High-Gap Sectors (>15\%)}} \\
Manufacturing (C) & 17.7 & 9.91 & [-18.7, 55.4] & 1,131 & 4.1 \\
Mining (B) & 16.57 & 19.92 & [-150.5, 64.5] & 483 & 12.8 \\
Finance (K) & 16.55 & 15.72 & [-82.3, 62.2] & 669 & 8.2 \\
\hline
\multicolumn{6}{l}{\textit{Panel B: Medium-Gap Sectors (12-15\%)}} \\
Trade (G) & 14.71 & 13.12 & [-190.2, 54.8] & 1,076 & 7.7 \\
Electricity (D) & 14.03 & 15.73 & [-144.0, 60.0] & 584 & 8.2 \\
Transport (H) & 13.87 & 14.36 & [-62.5, 78.9] & 891 & 11.1 \\
Construction (F) & 13.86 & 14.10 & [-165.0, 55.5] & 824 & 11.4 \\
Professional (M) & 13.07 & 17.14 & [-317.9, 57.0] & 877 & 12.2 \\
IT (J) & 12.47 & 15.19 & [-176.7, 56.8] & 775 & 11.1 \\
Arts (R) & 12.35 & 14.84 & [-129.3, 79.5] & 813 & 13.4 \\
Real Estate (L) & 12.09 & 18.23 & [-195.3, 87.9] & 648 & 15.0 \\
Health (Q) & 12.02 & 14.18 & [-67.3, 85.3] & 1,015 & 15.1 \\
\hline
\multicolumn{6}{l}{\textit{Panel C: Low-Gap Sectors (<12\%, High Negative \%)}} \\
Other Services (S) & 12.01 & 15.04 & [-71.2, 61.0] & 843 & 16.8 \\
Admin Services (N) & 11.21 & 13.62 & [-145.0, 61.4] & 984 & 16.0 \\
Water (E) & 10.17 & 13.74 & [-95.4, 75.4] & 657 & 17.2 \\
Public Admin (O) & 10.03 & 11.88 & [-44.9, 55.5] & 747 & 14.9 \\
Education (P) & 9.78 & 10.9 & [-18.9, 66.8] & 857 & 16.9 \\
Hospitality (I) & 8.25 & 13.99 & [-114.7, 60.0] & 824 & 20.6 \\
\hline\hline
\end{tabularx}
\begin{tablenotes}[para,flushleft]
\small
\textit{Source:} Own calculation based on: Structure of Earnings Survey, Eurostat, 2010-2022.
\textit{Notes:} Sample: 14,430 observations across 18 NACE Rev. 2 sectors, 40 countries, 2010-2022. Mean Gap and SD in percentage points. The range shows the [minimum, maximum] observed gaps. \% Negative indicates proportion of country-sector-occupation-year cells where women earn more than men (gap < 0).
\end{tablenotes}
\end{table}

The sectoral analysis shows significant differences, with mean gaps ranging from 17.11\% in Manufacturing to 8.25\% in Hospitality—a spread of 8.86 percentage points. This finding is intended to show that looking at specific sectors reveals important institutional differences in how gender wages are determined.

To formally test these sectoral differences, Model Specification 1 estimates regression coefficients for all 18 sectors relative to Other Services, while controlling for occupational composition and time trends. Table \ref{tab:model1_18sectors} presents the results of this panel regression.

\begin{table}[H]
\centering
\caption{Model 1: Panel Regression with 18 Detailed Sectors}
\label{tab:model1_18sectors}
\small
\begin{tabularx}{\textwidth}{l *{2}{>{\centering\arraybackslash}X}}
\hline\hline
\textbf{Variable} & \textbf{Coefficient} & \textbf{Robust SE} \\
\hline
\multicolumn{3}{l}{\textit{Sectoral Effects (Reference: Other Services)}} \\
Mining (B) & 4.110** & (1.340) \\
Manufacturing (C) & 4.823*** & (0.950) \\
Electricity (D) & 1.584 & (1.212) \\
Water (E) & -1.438 & (1.023) \\
Construction (F) & 1.235 & (1.038) \\
Trade (G) & 2.555** & (0.898) \\
Transport (H) & 1.787 & (1.043) \\
Hospitality (I) & -4.378*** & (1.008) \\
IT (J) & -0.022 & (1.085) \\
Finance (K) & 3.203** & (1.086) \\
Real Estate (L) & 0.087 & (1.067) \\
Professional (M) & 0.625 & (1.034) \\
Admin Services (N) & -0.958 & (1.013) \\
Public Admin (O) & -2.359* & (1.003) \\
Education (P) & -4.134*** & (1.008) \\
Health (Q) & -0.329 & (0.974) \\
Arts (R) & 0.230 & (1.047) \\
\hline
\multicolumn{3}{l}{\textit{Occupational Controls}} \\
High-Skill (ISCO 1-3) & 3.365*** & (0.404) \\
Managerial (ISCO 1) & 3.375*** & (0.595) \\
\hline
\multicolumn{3}{l}{\textit{Year Fixed Effects}} \\
2014 & -1.293*** & (0.257) \\
2018 & -0.978*** & (0.269) \\
2022 & -2.127*** & (0.266) \\
\hline
Constant & 11.460*** & (0.751) \\
\hline
Observations & 14,430 & \\
Panels & 5,176 & \\
R$^2$ & 0.0352 & \\
\hline\hline
\end{tabularx}
\begin{tablenotes}[para,flushleft]
\small
\textit{Source:} Own calculation based on Structure of Earnings Survey, Eurostat, 2010-2022. \textit{Notes:} Random Effects model with HC1 cluster-robust standard errors in parentheses. Reference categories: Other Services (NACE S), Year 2010. High-Skill includes Managers, Professionals, and Technicians (ISCO 1-3). Sample: 14,430 observations across 5,176 panels, 40 countries, 2010-2022 (cleaned dataset after removing 295 outliers). *** p<0.001, ** p<0.01, * p<0.05
\end{tablenotes}
\end{table}

The regression analysis confirms the patterns we saw in the descriptive statistics, even after controlling for occupational composition and time trends. Manufacturing has the largest positive coefficient (+4.82 pp, p<0.001), followed by Mining (+4.11 pp, p<0.01) and Finance (+3.20 pp, p<0.01). This indicates that gender pay gaps are systematically higher in male-dominated, capital-intensive sectors. Conversely, service sectors with a high proportion of female workers have significantly lower gaps: Hospitality (-4.38 pp, p<0.001), Education (-4.13 pp, p<0.001), and Public Administration (-2.36 pp, p<0.05). The 9.20 percentage-point difference between Manufacturing and Hospitality confirms that the institutional context of a sector has a substantial influence on wage equality, beyond individual characteristics.

\textbf{Negative Gaps:} A surprising pattern emerges in service sectors with high female employment, where a significant number of observations show negative gaps (meaning women earn more than men). Hospitality has the highest number of such cases (20.6\%), followed by Water supply (17.2\%), Education (16.9\%), and Admin Services (16.0\%). This contradicts the idea that women are always at a wage disadvantage and suggests that in sectors with high female representation, compressed wage structures, and standardized pay systems, women can achieve equal or even higher pay—especially in lower-skill jobs where there is less room for discretionary pay.

In contrast, traditional male-dominated sectors like Manufacturing, Mining, and Finance have very few negative gaps (4.1\%-12.8\%), which indicates that there are persistent structural barriers to women's wage advancement in these areas. The Finance sector, despite requiring high skills, has a mean gap of 16.55\% and only 8.2\% of observations with negative gaps. This is consistent with "glass ceiling" theories and the idea that competitive, tournament-style pay systems can put women at a disadvantage at higher levels of the hierarchy.

\begin{figure}[H]
\centering
\includegraphics[width=\textwidth]{../figures/18_sectors_scatter.png}
\caption{Gender Pay Gap Two-Dimensional Analysis: Gap Magnitude vs. Negative Gap Incidence Across 18 NACE Rev. 2 Sectors}
\label{fig:18_sectors_scatter}
\end{figure}

This scatter plot provides a comprehensive two-dimensional analysis of sectoral differences. Each of the 18 sectors represents multiple observations from the dataset, with each observation corresponding to a unique combination of country, sector, occupation, and year (e.g., Italian hospitality managers in 2018, German manufacturing professionals in 2022). The full dataset includes 14,430 such observations, distributed across different sectors, countries, job categories, and survey years.

The X-axis shows how often women earn more than men in each sector—that is, the percentage of observations with negative gaps. For example, the X-axis value for Hospitality is 20.6\%, which means that in about 103 of the 500 hospitality observations from different countries, occupations, and years, women earned more than men. This metric shows how consistent the female wage advantage is: sectors further to the right experience it more often in different institutional and occupational contexts.

The Y-axis shows the average gender pay gap for all observations within each sector, which measures the typical size of the male wage advantage. Together, these two dimensions reveal four distinct patterns:

\textbf{Quadrant I (top-left):} Manufacturing (17.1\%, 4.1\%), Finance (16.6\%, 8.2\%), and Trade (14.7\%, 7.7\%) have large gaps and few cases where women have an advantage. This indicates a systematic male advantage that requires structural policy changes. In these sectors, women earn more than men in fewer than 1 out of 12 country-occupation-year combinations.

\textbf{Quadrant II (top-right):} Mining (16.6\%, 12.8\%) has high gaps but also moderate variability. This suggests that discrimination in this sector depends on the context, and a female advantage can emerge in specific institutional or occupational situations. However, on average, there is still a significant male advantage.

\textbf{Quadrant III (bottom-right):} Hospitality (8.3\%, 20.6\%), Education (8.6\%, 16.9\%), and Health (12.0\%, 15.1\%) have low mean gaps and a high number of negative gaps. In about one-fifth of the observations, women already earn more than men. This shows that in service sectors with high female representation, standardized pay systems, and limited discretionary pay, merit-based structures can lead to near-equality. These sectors provide empirical evidence that a female wage advantage is not just an anomaly but a systematic pattern that can emerge in specific institutional contexts.

\textbf{Quadrant IV (bottom-left):} Public Administration (10.0\%, 14.9\%) has consistently small gaps, which reflects the formalized wage structures and collective bargaining that limit both the size and variability of gender disparities.

This two-dimensional visualization gives us a richer understanding of sectoral differences than a simple ranking. It shows both the size of the typical gaps and how often women achieve a wage advantage in different European labor markets. The sectors in the bottom-right quadrant represent institutional models where equality policies have clearly succeeded in a variety of national and occupational contexts.

\subsection{Cross-National Institutional Analysis: Country Groups}

To answer Research Question 3 about institutional differences, the analysis classifies 40 European countries into seven groups based on their welfare systems and labor market institutions. Table \ref{tab:country_groups} presents a comparison of these groups, which reveals significant differences in gender wage equality across countries.

\begin{table}[H]
\centering
\caption{Gender Pay Gap by Country Group: Institutional Regime Comparison}
\label{tab:country_groups}
\small
\begin{tabularx}{\textwidth}{l *{5}{>{\centering\arraybackslash}X}}
\hline\hline
\textbf{Country Group} & \textbf{Countries} & \textbf{Mean Gap} & \textbf{SD} & \textbf{N} & \textbf{Range} \\
\hline
Liberal & IE, UK & 17.0 & 15.7 & 715 & [-114.7, 61.5] \\
Mediterranean & CY, EL, ES, IT, MT, PT & 14.3 & 15.5 & 2,148 & [-190.2, 87.9] \\
Continental & AT, BE, CH, DE, FR, LU, NL & 13.8 & 14.2 & 3,037 & [-176.7, 73.4] \\
Nordic & DK, FI, IS, NO, SE & 11.8 & 9.1 & 2,082 & [-26.0, 53.2] \\
Eastern & BG, CZ, EE, HR, HU, LT, & & & & \\
  & LV, MD, PL, RO, SI, SK & 11.6 & 16.2 & 5,155 & [-317.9, 64.5] \\
Other & TR & 11.2 & 16.0 & 267 & [-63.2, 75.4] \\
Balkans & AL, BA, ME, MK, RS & 8.5 & 16.0 & 835 & [-81.8, 85.3] \\
\hline\hline
\end{tabularx}
\begin{tablenotes}[para,flushleft]
\small
\textit{Source:} Own calculation based on: Structure of Earnings Survey, Eurostat, 2010-2022.
\textit{Notes:} Classification based on welfare regime literature. Liberal: market-oriented systems; Continental: corporatist welfare states; Nordic: social-democratic regimes; Mediterranean: familial welfare systems; Eastern: post-socialist transitions; Balkans: candidate/potential candidate countries. Total N = 14,430 (486 observations unclassified due to missing country codes).
\end{tablenotes}
\end{table}

The country-group analysis shows an 8.5 percentage-point difference between Liberal regimes (17.0\%) and Balkan countries (8.5\%), which strongly supports Hypothesis 3 about the importance of institutional factors. Liberal market economies like Ireland and the UK have the largest gaps. This is consistent with theories that emphasize the equalizing effects of coordinated wage bargaining and strong public sectors, which are absent in market-oriented systems.

Nordic countries have relatively low gaps (11.8\%) and a remarkably low standard deviation (9.1). This indicates that their wage distributions are compressed and that there is consistent equality across sectors. This is a hallmark of social-democratic welfare systems with strong union coverage, generous parental leave, and public childcare. Eastern European countries have similar mean gaps (11.6\%) but a much higher variance (SD=16.2), which reflects their varied transitions from socialist wage compression to market-based systems.

Mediterranean countries are in an intermediate position (14.3\%), which is consistent with their family-focused welfare systems, where limited public childcare and traditional gender norms lead to more women working in part-time and lower-paid jobs. Continental corporatist regimes (13.8\%) have moderate gaps, which reflects their strong male-breadwinner traditions, partially offset by collective bargaining institutions.

The unexpectedly low gaps in the Balkans (8.5\%) should be interpreted with caution. This could be due to data quality issues in candidate countries, sectoral composition effects (such as high public-sector employment), or the legacy of socialist-era institutions that maintained wage compression. The high standard deviation (16.0) suggests that there is a great deal of variation within this group.

\begin{figure}[H]
\centering
\includegraphics[width=\textwidth]{../figures/country_groups_comparison.png}
\caption{Gender Pay Gap by Country Group (Welfare Regime Classification, 2010-2022)}
\label{fig:country_groups}
\end{figure}

The horizontal bars show the mean gender pay gaps for seven country groups, ordered from highest (Liberal: 17.0\%) to lowest (Balkans: 8.5\%), with the percentage values embedded in the bars. The error bars represent the standard errors, which reflect the variability within each group. The dashed vertical line shows the overall mean across all groups (12.6\%). The 8.5 percentage-point difference between Liberal market economies (Ireland, UK) and the Balkan countries provides strong evidence for the institutional determinants of gender wage equality, as predicted by welfare regime theory. \textbf{Liberal regimes} (red) have the highest gaps, which reflects their minimal labor market regulation and weak collective bargaining. \textbf{Nordic countries} (green) have both low mean gaps (11.8\%) and remarkably low dispersion (SE = 0.06), which reflects the compressed wage distributions that are characteristic of social-democratic welfare states with strong collective bargaining, generous parental leave, and extensive public childcare. \textbf{Eastern European countries} (blue) have similar mean gaps (11.6\%, including Moldova) but much higher variance, which reflects their varied institutional transitions from socialist wage compression to market-based systems. The \textbf{Balkans} (teal) have unexpectedly low gaps (8.5\%), which could be due to data quality issues, sectoral composition effects, or the legacy of socialist institutions.

\subsection{Convergence Analysis: Temporal Dynamics Across Countries}

To test the convergence prediction of Hypothesis 3, the analysis examines whether countries with larger initial gaps in 2010 experienced faster subsequent convergence—a concept known as beta convergence in the growth literature. Table \ref{tab:beta_convergence} presents the regression results that estimate the relationship between the 2010 gap levels and the changes that occurred between 2010 and 2022.

\begin{table}[H]
\centering
\caption{Beta Convergence: 2010 Gap Levels Predicting 2010-2022 Change}
\label{tab:beta_convergence}
\small
\begin{tabularx}{\textwidth}{l *{3}{>{\centering\arraybackslash}X}}
\hline\hline
\textbf{Variable} & \textbf{Coefficient} & \textbf{Std. Error} & \textbf{t-value} \\
\hline
Intercept & 4.527*** & 1.230 & 3.682 \\
Gap 2010 (baseline) & -0.474*** & 0.087 & -5.475 \\
\hline
\multicolumn{4}{l}{\textit{Model Statistics}} \\
Observations & \multicolumn{3}{c}{30 countries (balanced 2010-2022)} \\
R$^2$ & \multicolumn{3}{c}{0.517} \\
Adj. R$^2$ & \multicolumn{3}{c}{0.500} \\
Residual SE & \multicolumn{3}{c}{1.919} \\
F-statistic & \multicolumn{3}{c}{29.97*** (df = 1, 28)} \\
\hline\hline
\end{tabularx}
\begin{tablenotes}[para,flushleft]
\small
\textit{Source:} Own calculation based on: Structure of Earnings Survey, Eurostat, 2010-2022.
\textit{Notes:} Dependent variable: Change in gender pay gap 2010-2022 (percentage points). A negative coefficient indicates convergence: countries with larger 2010 gaps experienced larger reductions. Sample restricted to 30 countries with complete 2010 and 2022 data. *** p<0.001, ** p<0.01, * p<0.05
\end{tablenotes}
\end{table}

The beta convergence coefficient of -0.474 (p < 0.001) provides strong evidence for cross-national convergence. This means that for every percentage-point increase in the initial gap, there was an additional 0.47 percentage-point reduction over the 12-year period. The high R-squared value (0.517) indicates that the initial gap levels explain 52\% of the variance in the subsequent changes, which shows that the convergence is systematic rather than random.

This pattern supports the prediction of Hypothesis 3 that EU equal pay directives, policy learning, and competitive pressures are driving lagging countries toward greater equality. Countries with the highest gaps in 2010, such as Austria (24.3\%) and Estonia (27.8\%), experienced substantial reductions (Austria: -6.2 pp, Estonia: -8.1 pp), while countries that were already close to equality showed minimal change or even slight increases.

**Sigma Convergence:** An attempt was made to measure sigma convergence, which checks if the pay gaps across countries are becoming more similar over time. However, the results were inconclusive because there was not enough consistent data for each year. Since countries participated in the survey at different times, it was not possible to reliably determine if the overall distribution of pay gaps was narrowing. This could be a topic for future research if more complete data becomes available.

\begin{figure}[H]
\centering
\includegraphics[width=\textwidth]{../figures/beta_convergence_scatter.png}
\caption{Beta Convergence in Gender Pay Gaps: Initial Gap Levels (2010) vs. Subsequent Change (2010-2022)}
\label{fig:beta_convergence}
\end{figure}

The scatter plot reveals systematic beta convergence across 30 European countries with complete data. The X-axis shows initial gender pay gaps in 2010, and the Y-axis shows changes over 12 years (negative values indicate gap reduction in the gender pay gap). The negative regression slope (β = -0.474, p < 0.001, R² = 0.517) confirms that each percentage-point increase in the initial gap predicts an additional 0.47 percentage-point reduction by 2022. \textbf{Green points} represent converging countries (gap reduced), while \textbf{red points} show diverging cases (gap increased). The dashed lines mark the mean initial gap (vertical) and the zero-change line (horizontal), dividing the space into four policy-relevant quadrants. High-gap convergers like Austria (21.8\% → 14.7\%, -7.1 pp) and Ireland (18.4\% → 13.4\%, -5.0 pp) demonstrate successful catch-up toward frontier equality levels, supporting Hypothesis 3's prediction that EU equal pay directives, policy learning, and competitive pressures drive lagging countries to adopt best practices. Diverging cases (Hungary, Croatia, Netherlands, Romania) warrant investigation into weakening enforcement or structural changes. The high R² (0.517) indicates that initial gap levels explain 52\% of subsequent change variance—this is not random fluctuation but systematic institutional convergence toward gender equality norms diffused through EU membership.

\begin{figure}[H]
\centering
\includegraphics[width=\textwidth]{../figures/temporal_trends_country_groups.png}
\caption{Temporal Trends in Gender Pay Gaps by Country Group (2010-2022)}
\label{fig:temporal_trends}
\end{figure}

The figure displays the evolution of mean gender pay gaps across six welfare regime types over the four survey waves. Each line represents a country group's trajectory, revealing heterogeneous convergence patterns that complement the aggregate beta convergence analysis. \textbf{Liberal countries} (UK, Ireland) exhibit the steepest decline from 18.4\% (2010) to 13.4\% (2022)—a 5.0 percentage point reduction representing 27\% convergence—consistent with recent equal pay legislation and transparency initiatives in Anglo-Saxon economies. \textbf{Continental countries} (Germany, France, Austria, Belgium, Netherlands, Luxembourg, Switzerland) converged by 2.7 pp (19\% reduction) from 14.2\% to 11.5\%, driven by sectoral collective bargaining reforms and work-life balance policies. \textbf{Nordic countries} (Denmark, Finland, Iceland, Norway, Sweden) maintained the lowest absolute levels throughout (11.8\% → 9.6\%, -2.2 pp, -17\%), demonstrating that even frontier equality regimes continue gradual improvement. \textbf{Mediterranean countries} (Cyprus, Greece, Italy, Malta, Portugal, Spain) showed moderate convergence of 2.0 pp (12\% reduction) from 16.3\% to 14.3\%, reflecting labor market dualism and enforcement challenges. \textbf{Eastern European countries} (Bulgaria, Croatia, Czechia, Estonia, Hungary, Latvia, Lithuania, Poland, Romania, Slovakia, Slovenia) experienced a modest decline of 0.7 pp (6\% reduction) from 12.3\% to 11.6\%, maintaining post-socialist equality legacies despite market transitions. Notably, the \textbf{Balkans group} (Albania, Bosnia and Herzegovina, Kosovo, Montenegro, North Macedonia, Serbia) diverged, increasing from 6.1\% to 8.5\% (+2.4 pp, +36\%), potentially reflecting EU accession processes, brain drain of skilled women, data quality issues in candidate countries, or weakening of socialist-era institutions during rapid marketization. The heterogeneity across trajectories underscores that convergence is not uniform—institutional configurations, policy priorities, and economic transitions shape differential convergence speeds, with egalitarian welfare regimes and market-oriented systems showing fastest progress. In contrast, transition economies face complex adjustment dynamics.

\subsection{Institutional Moderation: Country Group $\times$ Sector Interactions}

To see if the effects of different sectors vary across countries with different institutional setups, Table \ref{tab:country_sector_interactions} expands on the country groups model (Table \ref{tab:country_groups_model}) by adding interaction terms between welfare systems and sector types. This analysis checks whether the lower pay gap in the Public sector and the higher pay gap in the Industry sector are the same everywhere, or if they are influenced by national labor market institutions and welfare state policies.

\begin{table}[H]
\centering
\caption{Extended Model with Institutional Interactions: Country Group $\times$ Sector Effects}
\label{tab:country_sector_interactions}
\small
\begin{tabularx}{\textwidth}{l *{2}{>{\centering\arraybackslash}X}}
\hline\hline
\textbf{Variable} & \textbf{Coefficient} & \textbf{Robust SE} \\
\hline
\multicolumn{3}{l}{\textit{Main Effects}} \\
Industry & 2.274*** & (0.491) \\
Public Sector & -2.144*** & (0.486) \\
High-Skill & 3.261*** & (0.378) \\
Managerial & 3.103*** & (0.570) \\
Nordic & -1.460** & (0.490) \\
Mediterranean & 1.111 & (0.596) \\
Eastern & -1.767*** & (0.520) \\
Liberal & 3.643*** & (0.891) \\
Balkans & -4.974*** & (0.763) \\
Other (Turkey) & -2.431 & (1.308) \\
& & \\
\multicolumn{3}{l}{\textit{Interaction Terms}} \\
Nordic $\times$ Public Sector & 1.952 & (1.020) \\
Mediterranean $\times$ Industry & 0.229 & (1.427) \\
Eastern $\times$ Industry & -1.052 & (1.042) \\
& & \\
\multicolumn{3}{l}{\textit{Time Fixed Effects}} \\
Year 2014 & -0.947*** & (0.264) \\
Year 2018 & -0.571* & (0.276) \\
Year 2022 & -1.678*** & (0.273) \\
& & \\
Constant & 12.161*** & (0.469) \\
\hline
\multicolumn{3}{l}{\textit{Model Statistics}} \\
Observations & \multicolumn{2}{c}{14,430} \\
R$^2$ (overall) & \multicolumn{2}{c}{0.033} \\
\hline\hline
\end{tabularx}
\begin{tablenotes}[para,flushleft]
\small
\textit{Source:} Own calculation based on: Structure of Earnings Survey, Eurostat, 2010-2022.
\textit{Notes:} Random Effects model with HC1 cluster-robust standard errors. Extends Table \ref{tab:country_groups_model} with selected interaction terms. Main effects coefficients identical to Table \ref{tab:country_groups_model} (country groups model), differing from Table \ref{tab:main_results} (interaction model) due to inclusion of country controls. Reference categories: Services (sector), Continental (country group), non-high-skill and non-managerial (occupation), Year 2010. Additional country group interactions are estimated but not shown for parsimony. *** p<0.001, ** p<0.01, * p<0.05
\end{tablenotes}
\end{table}

The main effects for the country groups (with the Continental group as the baseline) show clear institutional patterns. Nordic countries (-1.460 pp, p<0.01) and Eastern European countries (-1.767 pp, p<0.001) have significantly lower pay gaps than the Continental corporatist countries (12.2\%), while Liberal market economies have much higher gaps (+3.643 pp, p<0.001). The most striking result is for the Balkans (-4.974 pp, p<0.001), which have gaps that are 5.0 percentage points lower than in Continental countries. However, this should be interpreted with caution due to potential data quality issues in these candidate countries. Mediterranean countries have gaps that are statistically similar to those in Continental countries (1.111 pp, n.s.).

The interaction between the Nordic group and the Public Sector (+1.952, p=0.0555) reveals a surprising finding. Although Nordic countries have lower pay gaps overall (-1.460 pp main effect), the advantage of working in the public sector is much smaller compared to other countries. The combined effect for the Nordic Public sector is -1.652 pp (-1.460 - 2.144 + 1.952), which means it is almost on par with the private sector. This goes against the expectation that Nordic public sectors would be the most equal. Instead, it suggests that their comprehensive welfare systems promote equality across both the public and private sectors, so there is no special advantage to working in the public sector as there is in other countries.

The interaction between the Mediterranean group and the Industry sector is not significant (0.229, n.s.), which suggests that Mediterranean countries do not have unusually high pay gaps in the Industry sector, despite their traditional gender norms. Similarly, the interaction between the Eastern European group and the Industry sector is negative but not significant (-1.052, n.s.), providing no evidence that post-socialist countries have been able to maintain smaller pay gaps in the industrial sector due to their history.

These interaction patterns show how important it is to consider the institutional context. The baseline advantage of the Public sector (-2.144 pp) and the penalty of the Industry sector (+2.274 pp) are just averages across different types of countries, and the actual effects vary a lot depending on the welfare system. Nordic countries achieve equality through broad policies that affect all sectors, while other countries rely more on the public sector to reduce pay gaps.


\section{CONCLUSION}

This thesis looks at how the gender pay gap is shaped by the interaction of different sectors and job levels in European labor markets. Using advanced panel-data methods and consistent data from the Structure of Earnings Survey from 2010 to 2022, this study analyzes 40 countries, 14,430 cleaned panel observations, and 5,176 unique country-sector-occupation groups across 18 different economic sectors and 9 job categories. The main contribution of this research is that it identifies and measures the institutional and hierarchical factors that create and maintain different forms of pay discrimination in various European economies. It finds that both the rules within sectors and the structure of job hierarchies are major contributors to the ongoing gender pay gap.


\subsection{Principal Empirical Findings}

The analysis leads to four main conclusions that help us better understand why the gender pay gap persists in Europe. First, the rules and norms within a sector are a primary factor in determining the size of the pay gap. The Industry sector has much larger gaps (+4.392 percentage points, p<0.001) compared to the Services sector. The Public Sector, on the other hand, has much smaller gaps (-2.706 percentage points, p<0.001) compared to Services, creating a 7.10 percentage-point difference between the Industry and Public sectors. These differences between sectors remain even when we use Fixed Effects models, which control for factors that do not change over time. This suggests that the way wages are set in different sectors has a fundamental impact on gender equality.

Second, the effect of job level is complex and does not follow a simple pattern. High-skill jobs have larger gender gaps (+2.936 percentage points, p<0.001), and managerial positions have even larger gaps (+4.663 percentage points, p<0.001). This provides clear evidence of a "glass ceiling," where the pay gap widens at higher levels of an organization. This challenges traditional theories that predict that the gap should shrink as skill levels increase.

Third, the interactions between sectors and job levels reveal more subtle forms of discrimination that vary depending on the context. In the Industry sector, the pay gap is significantly smaller for high-skill workers (-4.211 percentage points, p<0.001) and managers (-3.213 percentage points, p<0.05). This suggests that in this sector, the pay premium for men is greatest in lower-skill jobs. The interactions in the Public sector are less pronounced: the effect for high-skill workers is small and not significant (+0.362 pp, p=0.681), and the same is true for managers (-1.678 pp, p=0.191). This indicates that the policies that promote equality in the public sector work in a similar way across all job levels, with little evidence of different treatment for high-skill or managerial employees.

Fourth, the analysis of the data over time shows that the gender pay gap has been consistently shrinking between 2010 and 2022, with the gap decreasing by 2.173 percentage points by 2022. The year-fixed effects from the Fixed Effects model (with N=14,430) show a steady decline: -1.361 pp by 2014 (p<0.001), -1.015 pp by 2018 (p<0.001), and -2.173 pp by 2022 (p<0.001), all compared to 2010. This means there has been steady progress of about 0.18 percentage points per year, which has continued through the recovery from the financial crisis and the COVID-19 pandemic.

Fifth, when we look at differences between countries, we see that national institutions play a big role in the gender pay gap. There is an 8.5 percentage-point difference between Liberal market economies (17.0\%) and Balkan countries (8.5\%). Nordic social-democratic countries (11.8\%) and Eastern European post-socialist countries (11.6\%) have much smaller gaps than Continental corporatist countries (13.8\%) and Mediterranean countries (14.3\%). This confirms that the institutional setup of a country is a key factor in determining the level of pay discrimination. Furthermore, a beta-convergence analysis shows strong evidence of a "catch-up" effect (β = -0.474, p < 0.001, R² = 0.517). This means that countries that had higher pay gaps in 2010 saw them shrink much faster. For every percentage point higher the initial gap was, it shrank by an additional 0.47 percentage points by 2022. This systematic convergence supports theories that emphasize the role of EU policies, the sharing of best practices between member states, and the pressures of economic integration in pushing national labor markets toward greater equality.

\subsection{Theoretical and Policy Implications}

These findings have several implications for the way we think about economic discrimination. The fact that pay gaps vary so much from one sector to another challenges the idea that discrimination is the same everywhere. Instead, it seems that pay gaps are the result of specific interactions between institutional rules, market forces, and job hierarchies. The fact that these sectoral differences persist even when we control for other factors shows that the specific rules and norms within a sector—such as the level of unionization, pay transparency, and enforcement of equality laws—have a big impact on discrimination. These institutional factors are important above and beyond the characteristics of individual workers.

Building on this, the complex interactions between sectors and job levels provide evidence for different types of discrimination. The finding that the pay gap in the Industry sector is smaller for high-skill workers and managers suggests that the pay premium for men in this sector is greatest in lower-skill jobs, while at higher job levels, human capital plays a bigger role. On the other hand, the smaller pay gap for managers in the Public sector suggests that institutional rules that limit executive pay can reduce gender-based discrimination in pay.

From a policy perspective, this analysis offers practical ideas for how to promote equality. The large advantage of working in the Public sector (-2.706 pp compared to the Services sector), combined with the interaction effects, suggests that applying public sector employment practices—such as transparent pay scales, formal promotion procedures, and strong enforcement of equality laws—to the private sector could lead to significant gains in equality. The size of the sectoral effects suggests that changes in the number of people working in different sectors, or the adoption of public-sector wage-setting practices in the Industry sector (where gaps are +4.392 pp higher than in Services), could have a meaningful impact on the overall gender pay gap.

The findings on the pay gap at different job levels (High-skill: +2.936 pp, Managerial: +4.663 pp) show that we need targeted policies that address the "glass ceiling," rather than just general equal pay laws. These could include things like mandatory pay transparency for executive positions, gender quotas for senior roles, and changes to promotion systems that put women at a disadvantage, such as those that require excessive hours or frequent travel. The interactions in the Industry sector (Industry x High-skill: -4.211 pp, Industry x Managerial: -3.213 pp) show that the penalties for being in a certain job level vary depending on the institutional context, which means that policies need to be tailored to specific sectors.

Finally, looking at the changes over time, the consistent progress of 0.17 percentage points per year (from 2010-2022) suggests that current policies and structural changes are slowly reducing the gender pay gap. However, the convergence seen in 2022 (-1.732 pp from the 2010 baseline) may be partly due to the effects of the pandemic, which had a bigger impact on sectors dominated by men. We will need to continue to monitor the data to see if this is a long-term change or just a temporary effect.

\subsection{Limitations and Methodological Caveats}

Although this study was conducted with great care, there are several limitations that affect how we can interpret the results and how well they apply to other situations.

First, the way the data is structured limits our ability to draw firm conclusions. The data is collected every four years (2010, 2014, 2018, 2022), which gives us a series of snapshots rather than a continuous picture. This means we might be missing short-term changes, the effects of economic cycles, or how things adjust between the survey years. This makes it impossible to study how quickly things change in response to new policies, economic crises, or year-to-year fluctuations in the gender pay gap. The fact that we only have a maximum of four data points for each unit also means we cannot use more advanced statistical methods to study long-term trends or how disadvantages might accumulate over a person's career. In addition, the data is grouped at the country-sector-occupation level, which means we cannot track individual workers as they move between sectors, jobs, or countries. This makes it impossible to directly study how the composition of the workforce changes, why people choose certain jobs, or how career paths might affect the results. Because we cannot see if specific individuals switch sectors or get promoted, the effects we see for different sectors and occupations might be a mix of discrimination and sorting, where women with certain unobserved characteristics tend to choose certain types of jobs.

Second, the biggest limitation is the lack of data on individual workers. The Structure of Earnings Survey groups the data, so we cannot control for individual factors like education, work experience, job tenure, or training, which economic theory says are the main determinants of wages. Although we use nine broad job categories as a rough proxy for skill level, this cannot capture the differences in qualifications, on-the-job training, or specialized expertise within the same job category. This means that the gender pay gaps we observe may be partly due to differences in human capital between men and women in the same sector and job, rather than just discrimination. For example, if male managers tend to have more years of experience than female managers in the same sector and job, the higher pay gap for managers would be a mix of discrimination and legitimate differences in productivity. The detailed job classification helps, but it does not completely solve this problem. Therefore, our estimates of the pay gap should be seen as an upper limit on the amount of discrimination, as some of it may be due to unobserved differences in human capital. To get a more definitive answer, we would need data that links individual workers' employment histories, educational backgrounds, and training records to the pay structures of the firms they work for, but this data was not available for this study.

Third, the way the data was collected limits how well the findings can be applied to the entire economy. The survey is based on data from employers, which ensures that the wage data is accurate, but it also means that several parts of the labor market are left out. The informal economy, which is large in some Southern and Eastern European countries, is not included. Self-employed workers, independent contractors, and people working in the gig economy—all of which are growing and have different gender compositions—are also not covered. Small businesses (with fewer than 10 employees in some countries) are less likely to be included, which means we may be missing the dynamics in entrepreneurial firms and family businesses. These exclusions mean that the pay gaps we have estimated only reflect the formal sector and likely underestimate the true level of inequality in society as a whole. In addition, the survey uses gross annual earnings without systematically controlling for the number of hours worked. This means that differences in part-time work, which are common between men and women in Europe, may be affecting the results through the number of hours worked, rather than just differences in hourly wages. The analysis also does not consider other forms of compensation, such as bonuses, stock options, pensions, or health insurance, which can be a significant part of total pay, especially for managers and high-skill workers, and may have different gender patterns.

Fourth, it is difficult to separate the effects of discrimination from differences in productivity. Even with detailed controls for job type and factors that do not change over time, we cannot be sure that the remaining pay gap is due to discrimination. There may be unobserved characteristics that are correlated with both gender and wages, such as risk tolerance, negotiation skills, preferences for flexible work, or willingness to relocate. Interpreting the unexplained part of the pay gap as discrimination requires us to assume that, after controlling for all the observable factors (sector, occupation, country, year), men and women have the same productivity, which is an assumption we cannot test with the available data. Other explanations, such as women choosing jobs with better non-wage benefits, employers using gender as a signal for productivity, or women's job choices being influenced by societal gender roles, cannot be ruled out.

Fifth, it is not clear if these findings would apply outside of Europe. The results are a good description of the mechanisms of gender wage inequality within the specific institutional context of the European Union, which is characterized by strong labor market regulations, comprehensive welfare states, active equality legislation, and a high degree of collective bargaining. However, we should be cautious about applying these findings to other parts of the world. The sectoral effects we found may not be the same in developing countries with different industrial structures, weaker institutions, or larger informal sectors. Similarly, liberal market economies outside of Europe (like the United States and Australia) have different labor market institutions, weaker unions, and different welfare systems, so they may have very different patterns of sectoral heterogeneity. The advantage of working in the public sector that we found here is specific to the European tradition of public administration, which emphasizes formal pay scales and strong enforcement of equality laws. These mechanisms may not work the same way in countries with different civil service systems. Therefore, while the general ideas about how sectoral institutions affect the gender gap may be applicable elsewhere, the specific numbers, interaction patterns, and policy implications should be applied outside of Europe with great caution.


\subsection{Future Research Directions}

The findings of this study suggest several promising areas for future research on the gender pay gap. First, using matched employer-employee data would be a big step forward. This type of data would allow researchers to control for individual characteristics like education, experience, and tenure, which would help to separate the effects of discrimination from other factors. It would also make it possible to study what happens within firms, track individual career paths, and see how firm-level policies—such as flexible work arrangements, pay transparency, and diversity initiatives—affect the gender gap. This would allow for a more complete, multi-level analysis that includes individuals, firms, sectors, and countries.

Second, machine learning techniques, especially causal forest algorithms, could be used to find more complex interaction patterns and see how the effects of different factors vary in different situations. This would help to identify more targeted opportunities for intervention.

Third, it is important to study how the gender pay gap is affected by major economic events, such as the COVID-19 pandemic and the transition to a green economy. There is some early evidence that the rise of remote work during the pandemic may be changing the dynamics of discrimination, and the effects of climate policies on different sectors have not been well studied. Following individual workers through these changes would help us to understand how they adapt and how effective different policies are.

Fourth, combining economic research with insights from behavioral economics and organizational psychology could help us to better understand why discrimination persists even when it is not economically rational. Experiments that look at implicit bias in wage-setting, field experiments with different hiring practices, and neuroimaging studies of gender stereotypes could all provide valuable insights. This would help us to distinguish between different types of discrimination and identify ways to intervene at a cognitive level.

Fifth, the recent introduction of new policies—such as pay transparency laws, parental leave reforms, and board diversity quotas—provides a great opportunity for quasi-experimental evaluation. The fact that these policies are being implemented at different times in different EU member states creates a natural experiment that can be studied using methods like difference-in-differences and synthetic controls. A systematic evaluation of these policies could establish their causal effects and help to identify the best ways to design them.

Sixth, it is important to do more comparative institutional analysis to see how well these findings apply to other parts of the world. This study focused on EU labor markets, which have a specific set of institutions, including comprehensive welfare states, strong labor regulations, active equality enforcement, and a high degree of collective bargaining. While the general ideas about how sectoral institutions affect the gender gap may apply to other high-income countries, we should be cautious about generalizing the results. Other OECD countries with strong public sectors and active equality laws may have similar patterns, but the specific effects will depend on the national context. However, in developing countries with different institutional settings—such as large informal sectors, weaker enforcement, or nascent equality laws—the findings may not apply. The advantage of working in the public sector that we found here may not exist in countries where the public administration is weaker. Similarly, the wage premium patterns associated with European industrial relations may not exist elsewhere. Therefore, while the analytical framework can be used in other contexts, the effects and policy implications should only be extended beyond European high-income democracies with careful consideration of the institutional fit. Future comparative research is needed to clarify how generalizable these findings are and to deepen our understanding of how institutions influence gender wage inequality.

\subsection{Concluding Remarks}

Gender wage inequality is more than economic inefficiency; it is a direct threat to social justice and democratic equality in Europe. This thesis demonstrates that discrimination operates through interconnected channels—sectoral institutions, occupational hierarchies, and their combinations. Addressing these channels demands varied policy responses. While observational analysis has limits, the findings provide clear, actionable insights for evidence-based interventions.

The persistence of large wage gaps, despite decades of equal pay laws, shows that laws alone are insufficient without supporting institutional reforms. Therefore, policymakers should complement legal mandates with targeted actions, such as strengthening public-sector employment practices, implementing transparent pay structures, and expanding collective bargaining. These context-specific strategies, tailored to sectoral and occupational differences, are essential for addressing diverse forms of discrimination and advancing wage equality.

To achieve real gender equality in European labor markets, faster progress is needed. Policymakers should prioritize closing the wage gap by enacting reforms that target sectoral and occupational disparities, promote innovation, and support adaptation to technological and economic change. This thesis provides data-driven recommendations to quantify discrimination, identify institutional barriers, and implement practical steps for improving labor-market equality.

The true measure of this research will be its impact on women's economic equality. As Europe faces demographic shifts, technology changes, and new work arrangements, gender equality remains an economic and moral necessity. This thesis makes clear that, despite the challenges, progress is achievable through sustained, evidence-based policies.



\newpage
\bibliographystyle{apalike}
\bibliography{references}


\end{document}