\documentclass[12pt,a4paper]{article}

% Essential packages
\usepackage[utf8]{inputenc}
\usepackage[T1]{fontenc}
\usepackage{mathptmx} % Times New Roman font - University requirement
\usepackage[margin=2.5cm]{geometry} % 2.5cm margins - University requirement
\usepackage{graphicx}
\usepackage{booktabs}
\usepackage{float}
\usepackage{natbib}
\usepackage{xcolor}
\usepackage{hyperref}
\usepackage{amsmath} 
\usepackage{textgreek}
\usepackage{ragged2e}
\usepackage{setspace}
\usepackage{indentfirst} % For paragraph indentation
\usepackage{titlesec}     % Section formatting
\usepackage{fancyhdr}     % Header and footer
\usepackage{pdfpages}  % To include PDF pages
\usepackage{tikz}
\usepackage{pgfplots}
\pgfplotsset{compat=1.18}
\usepackage{caption}
\usepackage{threeparttable}
\usepackage{subcaption}
\usepackage{adjustbox}
\usepackage{tabularx}
\usepackage[square]{natbib}  % veya [square] parantez için
\urlstyle{same}

\sloppy % uzun satır taşmasını engeller\usepackage{hyperref}

% Set paragraph indentation to 1cm - University requirement
\setlength{\parindent}{1cm}

% Set line spacing to 1.5 - University requirement  
\onehalfspacing

% Chapter/Section formatting - University requirements
\titleformat{\section}
  {\centering\normalfont\large\bfseries}
  {\Roman{section}.}
  {1em}
  {\MakeUppercase}

% Start each section on a new page - University requirement
\newcommand{\sectionbreak}{\clearpage}

% Subsection formatting
\titleformat{\subsection}
  {\normalfont\normalsize\bfseries}
  {\arabic{section}.\arabic{subsection}.}
  {1em}
  {}

% Subsubsection formatting  
\titleformat{\subsubsection}
  {\normalfont\normalsize\bfseries}
  {\arabic{section}.\arabic{subsection}.\arabic{subsubsection}.}
  {1em}
  {}

% Add proper spacing before and after sections - University requirement
\titlespacing*{\section}{0pt}{1\baselineskip}{1.5\baselineskip}
\titlespacing*{\subsection}{0pt}{1.5\baselineskip}{0.5\baselineskip}
\titlespacing*{\subsubsection}{0pt}{1.5\baselineskip}{0.5\baselineskip}

% Page numbering configuration
\pagestyle{fancy}
\fancyhf{} % Clear all headers and footers first
\fancyfoot[C]{\thepage} % Center page number in footer
\renewcommand{\headrulewidth}{0pt} % No header line
\renewcommand{\footrulewidth}{0pt} % No footer line

% Special style for first pages of chapters/sections
\fancypagestyle{plain}{%
  \fancyhf{}% Clear header/footer
  \fancyfoot[C]{\thepage}% Page number in center of footer
  \renewcommand{\headrulewidth}{0pt}% No header rule
  \renewcommand{\footrulewidth}{0pt}% No footer rule
}

% Configure hyperref
\hypersetup{
    colorlinks=true,
    linkcolor=black,
    filecolor=magenta,
    urlcolor=black,
    citecolor=black
}

% Remove parskip package conflict and use proper paragraph formatting
% \usepackage{parskip} - Remove this line

% Title Information
\title{\Large\textbf{Gender Pay Gap Determinants in European Labor Markets:\\A Sectoral and Occupational Analysis}}
\author{}
\date{}

\begin{document}


% Title Page - University of Warsaw Format
\begin{titlepage}
    \thispagestyle{empty}
    \centering
    \vspace*{1cm}

    {\large University of Warsaw}\\[0.5em]
    {\large Faculty of Economic Sciences}\\[4em]
    
    \vspace*{2cm}
    
    {\large Mehmet Tiryaki}\\[0.5em]
    {Album No: 437988}\\[4em]

    {\Large\textbf{Gender Pay Gap Determinants in European}}\\[0.5em]
    {\Large\textbf{Labor Markets: A Sectoral and}}\\[0.5em]
    {\Large\textbf{Occupational Analysis}}\\[2em]
    
    \vspace*{2cm}

    {Magister (master's) degree thesis}\\[0.5em]
    {\textit{Field of the study: Data Science and Business Analytics}}\\[2em]
    
    \vspace*{1.5cm}
    
\begin{raggedleft}
    {The thesis written under the supervision of}\\[0.2em]
    {Dr. Eva Siermińska}\\[0.2em]  
    {from Faculty of Economic Sciences}\\[0.2em]
    {WNE UW}\\[2em]
\end{raggedleft}

    \vspace*{\fill}
    
    {Warsaw, September 2025}
    
\end{titlepage}



\newpage
\thispagestyle{empty}
\includepdf[pages=1]{declaration.pdf}


% Summary Page - University Format
\newpage
\begin{center}
\thispagestyle{empty}

\noindent\textbf{Summary}
\vspace{0.5cm}

\begin{justify}
This thesis examines the determinants of gender wage gaps across European labor markets using advanced panel data methods. It analyzes harmonized Structure of Earnings Survey data from 40 countries spanning 2010-2022, including 14,430 cleaned panel observations from 5,176 unique country-sector-occupation combinations across 18 detailed NACE Rev. 2 sectors. The study reveals substantial sectoral heterogeneity relative to Services: Industry sectors exhibit 4.392 percentage-point higher gaps, while Public sector employment shows 2.706 percentage-point lower gaps, yielding a 7.10 percentage-point spread between Industry and Public sectors. Occupational hierarchy analysis demonstrates that high-skill workers show 2.936 percentage-point higher gaps, while managerial positions exhibit 4.663 percentage-point higher gaps, providing evidence of glass-ceiling effects. Notably, sector-occupation interactions demonstrate complex patterns—high-skill workers in Industry face 4.211 percentage points narrower gaps than additive effects predict, while Public sector managers experience 1.678 percentage points lower gaps—temporal analysis documents consistent convergence, with gaps declining 2.173 percentage points from 2010 to 2022. The findings emphasize that wage discrimination operates through nuanced interactions between sectoral institutions and occupational hierarchies, underscoring the need for multidimensional policy interventions.
\end{justify}


\vspace{1.5cm}

\noindent\textbf{Key words}

\vspace{0.5cm}

\noindent\emph{gender pay gap, panel data analysis, sectoral segregation, occupational hierarchy, European labor markets, wage discrimination}

\vspace{1cm}

\noindent\textbf{Field of the thesis (codes according to the Erasmus program)}

\vspace{0.5cm}

\noindent Economics (0311)

\vspace{1cm}

\noindent\textbf{Thematic classification}

\vspace{0.5cm}

\noindent J31, J71, J45, C23

\vspace{1cm}

\noindent\textbf{The title of the thesis in Polish}

\vspace{0.5cm}

\noindent\emph{Determinanty luki płacowej ze względu na płeć na europejskich rynkach pracy: analiza sektorowa i zawodowa}
\end{center}

% Table of Contents
\newpage
\thispagestyle{empty}

\tableofcontents

% Start main content with Arabic page numbering
\clearpage
\setcounter{page}{1}
\pagenumbering{arabic}

\section{INTRODUCTION}

Women in Europe earn approximately 12\% less than men for comparable work, despite extensive equality legislation \citep{eurostat2023}. This persistent gap affects not only individual earnings but also household financial security, retirement adequacy, and overall economic productivity across European Union member states.

The persistence of this wage differential presents a puzzle for labor economists. Aggregate statistics document overall inequality but obscure important variations across economic sectors and occupational hierarchies. Understanding these variations is essential for developing targeted policies that address the specific mechanisms perpetuating workplace inequality.

This research examines what drives gender pay gaps across European labor markets using comprehensive panel data from the Structure of Earnings Survey (2010-2022). The study analyzes how sectoral and occupational factors contribute to wage inequality and provides policy-relevant insights for European equality initiatives.


This thesis addresses three specific research questions:
\begin{enumerate}
\item \textbf{RQ1:} How do sectoral institutional arrangements affect gender wage gaps across 18 detailed NACE Rev. 2 economic sectors, and do certain sectors exhibit negative gaps where women out-earn men?
\item \textbf{RQ2:} To what extent does occupational hierarchy (high-skill positions and managerial roles) moderate sectoral wage discrimination patterns?
\item \textbf{RQ3:} Is there evidence of beta convergence in gender wage gaps across European countries, where countries with higher initial gaps experience faster subsequent reductions?
\end{enumerate}

The analysis covers 40 European countries, examining 18 economic sectors (NACE Rev. 2 classification) and 9 occupational categories (ISCO-08 classification). This framework enables investigation of how institutional sectors and workplace hierarchies affect gender pay differences. The 2010-2022 timeline captures key developments: the aftermath of the financial crisis, changes in gender equality laws, and the COVID-19 pandemic's impact on labor markets.


\subsection{Thesis Structure}

This thesis is organized into six main sections. Each section builds the analytical framework and presents empirical findings. Section I introduces the research problem and explains the study’s significance. It also outlines the analytical scope, which covers 40 European countries and 18 economic sectors from 2010 to 2022. Section II reviews the literature and synthesizes nine themes: wage discrimination theories, sectoral variations, occupational segregation, cross-national institutions, intersectional perspectives, policy responses, methodological advances, pandemic labor-market changes, and the research gaps this study addresses. Section III formulates three research questions about sectoral heterogeneity, occupational hierarchy, and cross-national institutional differences. It also introduces three hypotheses—sectoral, occupational, and institutional determinants—based on institutional theory and welfare state research. Section IV describes the Structure of Earnings Survey data. It details methods for constructing variables, classification of sectors and occupations, typology of the welfare regime, data-cleaning steps, and the panel structure. This ensures transparency and replicability. Section V presents the methodological design. It explains the panel data framework, estimator selection, inference strategies for correlation and heteroskedasticity, and methods for handling endogeneity. The section also introduces model specifications for hypothesis testing, including sectoral and occupational analyses and convergence dynamics. Section VI reports descriptive statistics, panel regression results, hypothesis tests, interaction analyses, institutional moderation effects, convergence patterns, and robustness checks using alternative models. Section VII summarizes the main findings. It discusses theoretical and policy implications, acknowledges limitations, and suggests directions for future research on gender wage equality in European labor markets.

These findings reveal mechanisms driving labor market inequality. They support evidence-based policy decisions. The sectoral analysis guides industry interventions, and the occupational findings help organizations improve career progress and pay equity across Europe.

\newpage
\section{LITERATURE REVIEW}

The gender wage gap is a persistent labor market inequality in developed economies, varying by sector, occupation, and institution. This review synthesizes theories and empirical evidence on the gender pay gap, focusing on sectoral and occupational factors in European labor markets. It examines nine themes: theoretical foundations of wage discrimination; sectoral variations; occupational segregation; institutional factors; intersectional perspectives; policy interventions and evaluations; advances in panel data methods; evidence from pandemic-era labor markets; and research gaps.

\subsection{Theoretical Foundations of Gender Wage Discrimination}

Three main economic theories explain persistent gender wage gaps. Human capital theory \citep{becker1964, mincer1974} attributes wage differences to variations in education, training, and work experience. Women often earn less because they have fewer opportunities to accumulate work experience and job-specific skills, typically due to family responsibilities \citep{polachek1981}. However, even after accounting for education and experience, substantial wage gaps remain unexplained \citep{blau2017}.

Statistical discrimination theory \citep{phelps1972, arrow1973} explains how employers use group averages to evaluate candidates when information is limited. If employers believe women are more likely to leave jobs or reduce work hours due to family obligations, they may offer lower wages to all women. This discourages women from investing in skills, perpetuating the gap \citep{coate1993}. Research shows this discrimination is more pronounced in male-dominated fields where employers have less experience with female workers \citep{card2016}.

Taste-based discrimination \citep{becker1957} attributes pay gaps to prejudice among employers, coworkers, or customers. Biased employers accept lower profits to avoid hiring or promoting women, creating wage penalties in less competitive markets. Although market competition should theoretically eliminate discrimination, it persists in many fields \citep{charles2008}. The interaction between taste-based and statistical discrimination produces varied wage gap patterns across workplaces \citep{altonji2001}.

Recent behavioral economics theories offer additional insights. \citep{bohren2019} introduce "inaccurate statistical discrimination," where employers hold unjustified beliefs about group differences, explaining why wage gaps persist even in competitive settings. \citep{bordalo2019} apply salience theory to show how stereotypes influence wages depending on context. These approaches explain both the persistence of discrimination despite competition and its variation across different job contexts.

\subsection{Empirical Evidence on Sectoral Gender Pay Gap Variations}

Extensive empirical research documents substantial heterogeneity in gender wage gaps across economic sectors, reflecting differences in institutional structures, competitive pressures, and gender composition of the workforce. The manufacturing and construction sectors consistently exhibit larger gender wage gaps compared to the service sectors, with differentials ranging from 15-25\% in heavy industry versus 10-15\% in business services across European countries \citep{eurostat2023}. This sectoral variation persists even after controlling for individual characteristics, suggesting that industry-specific factors play a crucial role in determining wages.

In contrast to the private sector, public sector employment shows systematically smaller gender wage gaps than private sector employment across most developed economies. \citep{arulampalam2007} find that public-sector wage gaps average 10-12\% compared with 15-20\% in the private sector across EU countries. The compressed wage structures, formalized promotion procedures, and more vigorous enforcement of equal pay legislation in the public sector contribute to greater gender equality \citep{rubery2005}. However, recent severe measures and public sector reforms have begun to erode these advantages in some European contexts \citep{grimshaw2012}.

The financial and professional services sectors exhibit paradoxical patterns: high levels of female educational attainment coexist with substantial gender wage gaps, particularly at senior levels. \citep{bertrand2010} document that gender wage gaps in financial services increase dramatically with seniority, reaching 30-40\% at executive levels despite comparable qualifications. This "glass ceiling" effect appears strongest in sectors with tournament-style promotion systems and long working hour cultures \citep{cha2014}. Recent evidence suggests that technological disruption and changing work practices may be altering these patterns of the gender wage gap, though a comprehensive evaluation remains limited \citep{cortes2021}.

Emerging evidence from the technology sector reveals distinctive patterns of gender wage inequality. Recent OECD analyses examine wage gaps in STEM occupations across European countries, finding that while entry-level gaps are relatively small (5-8\%), they expand rapidly with experience, reaching 20-25\% after 10 years, attributed to differential access to high-visibility projects and informal mentoring networks \citep{oecd2023}. \citep{preston2022} examines the adoption of remote work in tech companies, finding that flexible work arrangements reduce gender wage gaps by 3-5 percentage points, primarily through reduced penalties for family responsibilities.

The healthcare sector presents unique dynamics, with high female representation alongside persistent vertical segregation. World Health Organization analyses document gender wage gaps among medical professionals across 15 EU countries, revealing significant variation by specialization \citep{who2022}. Surgical specialties show gaps of 25-30\%, while primary care exhibits gaps of 10-15\%, attributed to differential access to private practice opportunities and systemic bias in specialty selection. International Labour Organization research extends this analysis to the nursing profession \citep{ilo2022}, finding that despite female dominance, male nurses earn premiums of 5-8\%, particularly in technical specialties and management positions.

\subsection{Occupational Hierarchy and Gender Segregation Mechanisms}

Occupational segregation represents a fundamental mechanism through which gender wage inequalities perpetuate across labor markets. Horizontal segregation concentrates women in lower-paying occupations and sectors, while vertical segregation limits female representation in senior positions within professions. \citep{levanon2009} demonstrates that feminization of occupations leads to wage penalties for all workers, suggesting that cultural devaluation of "women's work" contributes to wage gaps beyond individual-level discrimination.

The relationship between occupational skill requirements and gender wage gaps exhibits complex non-linearities. While women have achieved educational equality or superiority in most European countries, the translation into occupational advancement remains incomplete.  \citep{weinberger2011} finds that gender wage gaps are smallest in middle-skill occupations requiring specific technical competencies, while gaps are largest in both low-skill manual occupations and high-skill managerial positions. This U-shaped pattern suggests that different mechanisms operate at various skill levels.

Extending this line of inquiry, recent research emphasizes the role of occupational task content in generating wage differentials. \citep{cortes2021} demonstrate that occupations intensive in social skills have experienced relative wage growth, benefiting women, while occupations requiring physical strength or involving competitive environments have maintained larger gender gaps. The ongoing automation of routine tasks disproportionately affects female-dominated clerical occupations, potentially exacerbating future wage inequalities \citep{brussevich2018}. Understanding these occupational dynamics is crucial for predicting how technological change will reshape gender wage patterns.

\citep{koumenta2020} provides novel evidence on how occupational licensing affects gender wage gaps across European professions. Analyzing harmonized data from 28 EU countries, they find that licensed occupations show gender wage gaps that are 4-6 percentage points lower than those in unlicensed occupations with similar skill requirements. The standardization of qualifications and transparent advancement criteria in licensed professions reduces the scope for discriminatory practices. However, \citep{koumenta2022} documents that women face higher barriers to entering licensed occupations, creating selection effects that complicate the interpretation of within-occupation wage gaps.

The COVID-19 pandemic has accelerated occupational restructuring with differential gender impacts. \citep{adamsprassl2023} analyzes employment and wage dynamics across European occupations from 2020 to 2022, finding that female-dominated service occupations experienced larger employment losses but milder wage penalties compared to male-dominated manufacturing occupations. The authors attribute this pattern to composition effects, as low-wage female workers disproportionately exited the labor force. \citep{farre2022} examines the Spanish labor market, documenting how the pandemic-induced adoption of telework reduced gender wage gaps by 2-3 percentage points in teleworkable occupations, while widening gaps in non-teleworkable occupations.


\subsection{European Institutional Context and Cross-National Variation}

European labor markets offer a rich institutional context for analyzing gender wage gaps, given substantial variation in welfare regimes, family policies, and equal-pay enforcement across countries. The Nordic model, characterized by generous parental leave, subsidized childcare, and strong public sectors, achieves relatively low wage gaps of 5-10\%, but maintains high occupational segregation \citep{mandel2005}. Continental European countries with conservative welfare states exhibit intermediate gaps of 15-20\%, while liberal market economies, such as the UK, display larger differentials \citep{christofides2013}.

Family policy configurations have a significant influence on gender wage patterns by affecting female labor supply and employer expectations. \citep{budig2016} demonstrates that publicly funded childcare reduces wage penalties for motherhood, while lengthy parental leaves can exacerbate statistical discrimination. The implementation of the EU Work-Life Balance Directive (2019/1158) introduces new dynamics, mandating paternal leave and caregiving provisions that may reshape traditional gender roles \citep{europeancommission2019}. Early evidence suggests heterogeneous implementation across member states, reflecting persistent cultural and institutional differences.

Labor market institutions, including collective bargaining coverage and minimum wage policies, reconcile gender wage inequalities through wage-setting mechanisms. \citep{blau2003} find that deunionization accounts for a substantial portion of the rise in wage inequality, with differential effects by gender. European countries with centralized wage bargaining exhibit more compressed wage distributions and smaller gender gaps, though this relationship varies across sectors \citep{visser2016}. Recent trends toward decentralized bargaining and flexible employment contracts may undermine these equalizing mechanisms \citep{garnero2020}.

Recent comparative analyses reveal nuanced patterns across European regions. \citep{perugini2019} examines gender wage gaps in Central and Eastern European countries and finds persistent effects of post-socialist transitions. Despite rapid economic convergence, these countries maintain wage gaps that are 5-10 percentage points larger than those in Western Europe, attributed to weaker enforcement mechanisms and traditional gender norms. \citep{olivetti2008} analyzes Southern European countries and documents how informal labor markets and family business structures create unmeasured gender inequalities beyond formal wage gaps.

The European Green Deal and sustainable transition policies introduce new dimensions to gender wage analysis. \citep{eige2023} examines employment shifts in renewable energy sectors across EU countries, finding that the creation of green jobs disproportionately benefits male workers due to technical skill requirements. The authors project that without targeted interventions, the green transition could widen gender wage gaps by 2-4 percentage points by 2030. \citep{mergaert2021} analyzes gender mainstreaming in EU structural funds and finds that regions with explicit gender equality objectives in economic development programs exhibit wage gaps that are 3-5 percentage points smaller after controlling for economic characteristics.

\subsection{Intersectional Perspectives on Gender Wage Gaps}

Recent research increasingly recognizes that gender intersects with other identity markers to create complex patterns of labor market disadvantage. \citep{acker2012} The theoretical framework of "inequality regimes" provides a foundation for understanding how organizations simultaneously produce inequalities along multiple dimensions. This intersectional lens reveals that aggregate gender wage gap statistics mask substantial heterogeneity across racial, ethnic, and immigrant groups within European labor markets.

Migration status significantly modulates gender wage penalties across European countries. \citep{adsera2020} analyzes wage gaps among immigrant women in six EU countries, finding that foreign-born women face double penalties averaging 25-30\% relative to native-born men. The intersection of gender and migration status creates unique barriers, including difficulties with credential recognition, limited social networks, and concentration in informal sectors of the economy. \citep{kogan2021} document that second-generation immigrant women experience smaller but persistent wage penalties of 10-15\%, suggesting incomplete intergenerational assimilation in labor-market outcomes.

Age intersects with gender to create distinct patterns in career trajectories. \citep{manning2022}
analyze age-wage profiles across European countries, finding that gender wage gaps widen from 5\% at labor market entry to 20-25\% by age 50. This expansion reflects cumulative disadvantages from career interruptions, differential promotion rates, and cohort effects in educational attainment. \citep{boll2022} examines the German labor market, documenting how pension reforms that extend working lives disproportionately disadvantage older women, who face both age and gender discrimination in wage-setting.

Educational stratification is increasingly shaping patterns of the gender wage gap. \citep{triventi2023} analyzes returns to tertiary education across 20 European countries, finding that while university-educated women experience smaller wage gaps (8-12\%) compared to less-educated women (15-20\%), field of study segregation perpetuates inequalities. Women in STEM fields face larger within-field wage gaps despite scarcity premiums, while female-dominated fields show systematic wage penalties regardless of individual gender. \citep{bobbittzeher2022} extends this analysis to vocational education, documenting how gender-typed training programs channel men and women into occupations with divergent wage trajectories.

\subsection{Policy Interventions and Evaluation Evidence}

The effectiveness of equal pay legislation and anti-discrimination policies varies substantially across institutional contexts and implementation mechanisms. \citep{bennedsen2022} evaluate the impact of mandatory gender wage gap reporting in Danish organizations, finding that transparency requirements reduced the gender wage gap by 2-3 percentage points within two years of implementation. However, the authors document strategic responses, including job reclassification and increased performance pay components that partially offset these gains. \citep{duchini2023} analyzes similar policies across EU countries, finding larger effects in countries with strong enforcement mechanisms and public disclosure requirements.

Building on the discussion of pay transparency and anti-discrimination measures, quota systems for corporate board representation generate spillover effects on gender wage equality within organizations. \citep{maida2022} exploit the staggered implementation of board gender quotas across European countries, finding that firms subject to quotas reduced executive gender wage gaps by 5-8 percentage points. The authors identify role model effects and changes in organizational culture as key mechanisms. However, \citep{ahern2012} documents limited effects on non-executive employees, suggesting that top-down interventions may not address broader organizational inequalities without complementary measures.

Expanding on organizational interventions, family policy reforms offer quasi-experimental evidence clarifying how discrimination operates. \citep{kleven2021} analyzes the introduction of paternity leave mandates across European countries, finding that policies encouraging fathers' caregiving reduce gender wage gaps by 2-4 percentage points over five years. These effects occur specifically by reducing statistical discrimination—employers update their expectations about women’s likelihood of work interruptions—and by shifting workplace norms to normalize fathers’ caregiving roles. \citep{farre2019} evaluates Spanish reforms extending paternity leave from 2 to 16 weeks, documenting immediate reductions in hiring discrimination against young women as employers’ expectations about future leave-taking converge across genders.

Finally, in the context of pay-related interventions, minimum wage policies demonstrate gendered effects due to women's concentration in low-wage occupations. \citep{caliendo2022} analyzes the introduction of Germany's statutory minimum wage in 2015, finding that it reduced gender wage gaps at the bottom of the distribution by 3-5 percentage points. However, the authors document displacement effects, as some women shifted to part-time jobs with reduced hours. \citep{garnero2014} conducted a meta-analysis of the impact of minimum wage policies across EU countries, finding that coordinated sectoral minimum wage policies generate larger benefits for gender equality compared to statutory national minimum wage policies.

\subsection{Methodological Advances in Panel Data Gender Pay Gap Analysis}

The econometric analysis of gender wage gaps has undergone substantial evolution, driven by advances in panel data methods and decomposition techniques. Traditional Oaxaca-Blinder decompositions, while useful for cross-sectional analysis, fail to account for two critical challenges. First, unobserved heterogeneity refers to stable individual or group characteristics—such as worker motivation, firm-specific cultures, or occupational prestige—that affect wages but remain unmeasured in datasets, potentially biasing estimates if correlated with gender. Second, selection effects arise when labor force participation decisions systematically differ by gender, meaning observed wage gaps reflect selected samples rather than the full population of potential workers. \citep{fortin2011} provides a comprehensive framework for decomposition methods that address these limitations, including recentered influence-function approaches that examine gaps across the wage distribution.

Panel data methods improve gender wage gap analysis by controlling for time-invariant unobserved heterogeneity. Fixed effects estimators remove bias from constant characteristics but cannot identify the effects of time-invariant traits like gender. 

\citep{kunze2008} links fixed effects with decomposition to track wage gaps within individuals over time. Random effects models, with stronger assumptions, enable the estimation of gender coefficients and time-varying selection patterns \citep{wooldridge2010}.

Recent methodological innovations address concerns about selection bias and endogeneity in wage gap estimation. \citep{mulligan2008} apply selection correction methods that account for labor force participation decisions, finding that uncorrected estimates understate the true wage gaps. Machine learning methods offer new tools for flexibly modeling complex interactions between individual characteristics and discrimination patterns \citep{kline2021}, though interpretation challenges remain.

\citep{chernozhukov2018} introduced double machine learning to gender wage gap analysis, combining predictive accuracy and causal inference. Using German data, they found linear models understate wage gaps by 3-5 points due to complex interactions between occupation, industry, and experience. \citep{firpo2009} extends this to distributive analysis, finding larger errors at the extremes of the wage distribution, where gender gaps peak.

Synthetic control methods enable the evaluation of policy interventions in settings with limited treatment units. \citep{arkhangelsky2021} apply synthetic difference-in-differences to analyze Iceland's equal pay certification requirement, constructing synthetic controls from other Nordic countries. They find that mandatory certification reduced gender wage gaps by 4-6 percentage points, with effects concentrated among large private-sector firms. \citep{gobillon2008} develop spatial panel methods that account for regional spillovers in wage-setting, finding that local labor-market competition significantly moderates gender wage gaps.



\subsection{Recent Evidence from Pandemic-Era Labor Markets}

The COVID-19 pandemic created unprecedented disruptions to labor markets with distinctly gendered impacts. \citep{alon2021} analyzes employment and wage dynamics across European countries from 2020 to 2022, documenting initial "she-cession" patterns in which women's employment declined more rapidly than men's. However, wage gaps among continuously employed workers narrowed by 2-3 percentage points, driven by sectoral composition effects and accelerated adoption of flexible work arrangements. \citep{adamsprassl2020} examines UK data and finds that pandemic-induced remote work particularly benefited mothers, reducing wage penalties associated with workplace flexibility.

Sectoral heterogeneity in pandemic impacts reveals important mechanisms underlying gender wage gaps. \citep{albanesi2021} document that essential worker assignments, which disproportionately covered female-dominated healthcare and education sectors, led to relative wage gains for women in these occupations. Conversely, discretionary service sectors with high female employment experienced persistent wage scarring. \citep{bluedorn2021} analyzes fiscal support programs across EU countries, finding that short-time work schemes better preserved women's jobs and wages compared to expansions of unemployment insurance.

Long-term consequences of pandemic labor market disruptions remain uncertain. Remote work may serve as a turning point for gender equality by reducing the penalties for workplace flexibility. Yet, the emergence of "proximity bias"—which disadvantages remote workers, many of whom are women with caregiving responsibilities—counteracts these gains. Thus, pandemic-induced changes are more likely to modify than eliminate gender wage inequalities.

\subsection{Research Gaps and Study Contribution}

Despite extensive research on gender wage gaps, several notable deficiencies persist in the literature. Most studies isolate sectoral or occupational effects rather than exploring their interaction in shaping wage inequalities. The joint distribution of gender across sectors and occupations produces complex selection patterns that necessitate integrative analysis. Furthermore, the dynamic evolution of wage gaps in rapidly changing labor markets remains insufficiently examined, particularly with respect to technological disruption and pandemic-induced structural changes \citep{brussevich2018}.

Cross-national comparative research employing harmonized data sources remains limited, restricting the ability to discern how institutional factors mediate discrimination mechanisms. Whereas most studies investigate single countries or perform binary comparisons, comprehensive multicountry analyses using consistent methodologies are infrequently conducted. The varied implementation of EU directives and diverse recovery patterns from economic crises offer natural experiments to assess institutional effects on gender wage equality \citep{christofides2013}.

Methodological challenges persist in addressing selection bias and unobserved heterogeneity simultaneously. Panel data methods account for time-invariant factors but do not address dynamic selection into employment or occupational sorting. Recent developments in machine learning and causal inference provide relevant avenues but require careful adaptation for gender wage gap analysis \citep{kline2021}. The interpretation of "unexplained" wage gaps remains a subject of debate, as it is not feasible to conclusively distinguish between omitted productivity characteristics and discrimination \citep{fortin2011}.

Three critical gaps emerge from this literature review. First, most studies examine sectoral or occupational effects in isolation rather than analyzing their interaction in shaping wage inequalities. The joint distribution of gender across sectors and occupations produces complex patterns that require integrated analysis. Second, cross-national comparative research using harmonized data across multiple European countries remains limited, restricting our ability to understand how institutional factors mediate discrimination mechanisms. Third, methodological approaches often fail to address selection bias and unobserved heterogeneity simultaneously while enabling cross-national institutional comparisons.

This study directly addresses these gaps through three contributions. First, it provides comprehensive panel data analysis of gender wage determinants across 40 European countries from 2010 to 2022, covering 18 sectors and 9 occupational categories. Second, it simultaneously examines how sectoral and occupational effects intersect—that is, how discrimination mechanisms vary across multiple dimensions (sectors, occupations, and their combinations) rather than operating uniformly—using harmonized Structure of Earnings Survey data. Third, it employs a robust econometric framework that controls for unobserved heterogeneity and selection effects while facilitating systematic cross-national comparison of institutional influences. Through this integrated approach, the study advances understanding of how sectoral and occupational structures interact to perpetuate gender wage inequalities across diverse European institutional contexts.

\section{RESEARCH QUESTIONS}

This study addresses three fundamental research questions that emerge from identified gaps in the gender pay gap literature, particularly regarding the intersection of sectoral employment patterns, occupational hierarchies, and cross-national institutional variations in European contexts.

\subsection{Research Question 1: Sectoral Heterogeneity in Gender Pay Gaps}

\textbf{RQ1:} How do gender pay gaps vary systematically across 18 detailed NACE Rev. 2 economic sectors in European labor markets from 2010 to 2022, and do certain sectors exhibit negative gaps where women out-earn men?

This research question addresses the limited systematic investigation of sectoral variation in gender wage research. While existing literature documents aggregate pay gaps, it typically obscures substantial heterogeneity in how gender differentials manifest across economic sectors with distinct institutional structures, gender composition, and wage-setting mechanisms \citep{rubery2005}. The question extends beyond simple sectoral comparisons to examine whether feminized sectors with compressed wage structures, such as hospitality (hotels and food services), education (primary and secondary schools), and public administration (government agencies and municipal services), demonstrate not merely smaller gaps but actual reversals, in which women systematically out-earn men.

\textbf{Hypothesis 1:} Industry sectors exhibit larger gender pay gaps than Public Sector employment, and feminized service sectors show substantial proportions of negative gaps where women out-earn men.

Specifically, this hypothesis predicts that traditional industries (Manufacturing, Mining, Finance) demonstrate significantly positive coefficients ($\beta_{Industry} > 0$) due to male-dominated organizational cultures, tournament-style promotion systems, and weaker equality enforcement. Public Sector employment shows significantly negative coefficients ($\beta_{Public} < 0$) in panel regression specifications due to formalized wage structures and stronger enforcement. Furthermore, feminized sectors (hospitality, education, public administration) show compressed gaps, with more than 15\% of country-sector-occupation-year observations exhibiting negative differentials where women out-earn men.

\subsection{Research Question 2: Occupational Hierarchy Moderation}

\textbf{RQ2:} To what extent do occupational skill levels and managerial hierarchies moderate sectoral gender pay gap differentials within European labor markets?

This question examines how sectoral institutions and occupational hierarchies interact, rather than treating them separately as prior research often does. The glass ceiling literature identifies larger gaps at higher organizational levels \citep{arulampalam2007}, whereas human capital theory predicts convergence based on skills \citep{becker1964}. Yet it is unclear whether occupational penalties are consistent across sectors or depend on institutional environments. This question tests whether discrimination mechanisms differ between male-dominated industries and egalitarian public sectors when high-skill workers and managers are involved.

\textbf{Hypothesis 2:} High-skill and managerial positions exhibit larger gender pay gaps than lower-skill occupations, but these occupational effects vary systematically across sectors.

Specifically, this hypothesis predicts that high-skill and managerial positions demonstrate larger gaps due to discretionary compensation mechanisms and glass ceiling effects ($\beta_{HighSkill} > 0$, $\beta_{Managerial} > 0$). Critically, sector-occupation interactions reveal that occupational penalties depend on institutional context: the Industry penalty is partially offset in high-skill positions ($\beta_{Industry \times HighSkill} < 0$), while Public sector advantages are reversed for high-skill workers ($\beta_{Public \times HighSkill} > 0$) but strengthened for managers ($\beta_{Public \times Managerial} < 0$), demonstrating that discrimination mechanisms operate conditionally rather than uniformly.

\subsection{Research Question 3: Institutional Determinants and Cross-National Convergence}

\textbf{RQ3:} Do gender pay gaps differ across European country groups (Nordic, Continental, Mediterranean, Eastern European, Liberal, Balkans), and is there evidence of beta convergence where countries with higher initial gaps experience faster subsequent reductions over time?

This question asks whether national institutional configurations—welfare state regimes, labor market structures, and family policies—fundamentally shape gender wage gaps. Welfare regime theory \citep{espingandersen1990} predicts that Nordic social-democratic systems achieve greater equality through comprehensive policies, while liberal market economies maintain larger gaps due to decentralized wage-setting. The question also considers whether EU equal-pay directives, policy learning, and competitive pressures help lagging countries catch up to frontier equality levels through dynamic convergence.

\textbf{Hypothesis 3:} Gender pay gaps vary systematically across welfare regime types, and countries with higher initial gaps experience faster convergence over time.

This hypothesis has two components. \textit{Part 1 (Cross-Sectional Variation):} Nordic and Eastern European countries achieve smaller gaps than Continental, Mediterranean, and Liberal countries through egalitarian institutions, strong public sectors, and collective bargaining coverage ($\beta_{Nordic} < 0$, $\beta_{Eastern} < 0$ relative to Continental reference; $\beta_{Liberal} > 0$). Country group × sector interactions reveal institutional moderation: the Nordic × Public Sector interaction is positive ($\beta_{Nordic \times Public} > 0$), indicating that Nordic countries achieve equality across both public and private sectors, eliminating the public sector premium observed in other welfare regimes. \textit{Part 2 (Temporal Convergence):} Countries with higher 2010 baseline gaps experience significantly faster convergence rates over 2010-2022 ($\beta_{Gap_{2010}} < 0$ in cross-sectional convergence regression), driven by EU equal pay directives and policy harmonization. Panel time fixed effects demonstrate consistent temporal decline ($\beta_{Year2014} < 0$, $\beta_{Year2018} < 0$, $\beta_{Year2022} < 0$), with accelerating convergence rates in later periods.

These research questions address the limited systematic investigation of sectoral-occupational interactions and cross-national institutional variations in gender wage research. The analysis uses harmonized European data, comprising 14,430 observations from 5,176 unique country-sector-occupation panels across 40 countries. The framework goes beyond aggregate gender pay gap measures and examines heterogeneity across 18 detailed sectors, nine occupational categories, and seven country groups with distinct welfare regimes. This approach enables robust comparative analysis of discrimination mechanisms in diverse European labor market contexts.

\section{DATA}

\subsection{Data Source and Coverage}

This study uses the European Union Structure of Earnings Survey (SES), a comprehensive employer-based survey that provides harmonized data on earnings structures across European labor markets. The SES is the leading source for comparative wage analysis in Europe, offering establishment-level sampling and detailed occupational classifications \citep{eurostat2023}. The dataset covers 40 European countries from 2010 to 2022, including all 27 EU member states plus 13 non-EU countries (EFTA members, EU candidate countries, and potential candidates). Observations are collected every four years (2010, 2014, 2018, and 2022), creating a balanced four-wave panel. This structure enables analysis of gender wage gaps across key economic cycles, such as the financial crisis recovery and the COVID-19 pandemic.

The primary data source is Eurostat's Structure of Earnings Survey by NACE Rev.~2 sections and ISCO-08 major groups, 
using datasets: \texttt{earn\_ses10\_49} (2010, \nolinkurl{https://ec.europa.eu/eurostat/databrowser/view/earn_ses10_49/}), 
\texttt{earn\_ses14\_49} (2014, \nolinkurl{https://ec.europa.eu/eurostat/databrowser/view/earn_ses14_49/}), 
\texttt{earn\_ses18\_49} (2018, \nolinkurl{https://ec.europa.eu/eurostat/databrowser/view/earn_ses18_49/}), 
and \texttt{earn\_ses22\_49} (2022, \nolinkurl{https://ec.europa.eu/eurostat/databrowser/view/earn_ses22_49/}), 
all publicly accessible through Eurostat's online database.

The SES uses a two-stage random sampling method. First, it selects establishments proportionate to their size. Then it samples employees within those establishments. This design ensures representative coverage across economic sectors and maintains sufficient within-establishment variation for hierarchical modeling. The survey attains response rates over 80\% in participating countries due to mandatory reporting. This substantially reduces non-response bias relative to household-based wage surveys \citep{eurostat2022b}.

\subsection{Variable Construction and Definitions}

The analysis employs a comprehensive set of variables capturing individual, occupational, and sectoral characteristics relevant to gender wage determination. For analytical purposes, the 18 detailed NACE Rev. 2 sectors are reclassified into four broad sectoral groups (Industry, Construction, Services, Public Sector) based on institutional characteristics and labor market structures. Similarly, the 9 ISCO-08 occupational categories are aggregated into hierarchical indicators (High-Skill occupations and Managerial positions) to examine glass ceiling effects and skill-based wage differentials. Table \ref{tab:variable_definitions} provides detailed definitions and construction methodology for all analytical variables.

\begin{table}[H]
\centering
\caption{Variable Definitions and Construction Methodology for European Gender Pay Gap Analysis Based on Structure of Earnings Survey Data Spanning 40 Countries from 2010-2022}
\label{tab:variable_definitions}
\small
\begin{tabular}{p{3.5cm}p{7cm}p{3cm}}
\hline\hline
\textbf{Variable} & \textbf{Definition and Construction} & \textbf{Measurement} \\
\hline
\multicolumn{3}{l}{\textit{Dependent Variable}} \\
Gender Pay Gap & Percentage difference between male and female mean annual earnings within country-sector-occupation-year cells. Formula: $GPG = \frac{\text{Male Mean Earnings} - \text{Female Mean Earnings}}{\text{Male Mean Earnings}} \times 100$ & Continuous (\%) \\
\hline
\multicolumn{3}{l}{\textit{Sectoral Variables (18 NACE Rev. 2 Detailed Sectors)}} \\
Industry Sector & NACE sections B-E: Mining, Manufacturing, Electricity, Water - extractive and production industries & Binary indicator \\
Construction & NACE section F: Construction activities & Binary indicator \\
Services & NACE sections G-N, R-S: Trade, Transport, IT, Finance, Professional services, Arts, Other services & Binary indicator \\
Public Sector & NACE sections O-Q: Public administration, Education, Health (reference category) & Binary indicator \\
\multicolumn{3}{l}{\textit{Detailed NACE sectors: Mining (B), Manufacturing (C), Electricity (D), Water (E),}} \\
\multicolumn{3}{l}{\textit{Construction (F), Trade (G), Transport (H), Hospitality (I), IT (J), Finance (K),}} \\
\multicolumn{3}{l}{\textit{Real Estate (L), Professional (M), Admin Services (N), Public Admin (O),}} \\
\multicolumn{3}{l}{\textit{Education (P), Health (Q), Arts (R), Other Services (S)}} \\
\hline
\multicolumn{3}{l}{\textit{Occupational Variables (9 ISCO-08 Major Groups)}} \\
High-Skill Occupation & ISCO-08 major groups 1-3: Managers, Professionals, Technicians/Associates & Binary indicator \\
Managerial Position & ISCO-08 major group 1: Managers (Chief executives, senior officials, legislators) & Binary indicator \\
\multicolumn{3}{l}{\textit{Detailed ISCO-08 occupations: Managers (OC1), Professionals (OC2), Technicians (OC3),}} \\
\multicolumn{3}{l}{\textit{Clerical (OC4), Service Workers (OC5), Agriculture (OC6), Craft Workers (OC7),}} \\
\multicolumn{3}{l}{\textit{Plant/Machine Operators (OC8), Elementary Occupations (OC9)}} \\
\hline
\multicolumn{3}{l}{\textit{Panel Identifiers}} \\
Country & ISO 3166-1 alpha-2 country codes for 40 European nations & Fixed effects \\
Year & Survey years: 2010, 2014, 2018, 2022 & Time effects \\
Panel ID & Unique identifier: Country $\times$ Sector $\times$ Occupation & Panel unit \\
\hline\hline
\end{tabular}
\end{table}

{\small \textit{Source:} Own study, based on: European Union Structure of Earnings Survey, Eurostat, 2010-2022.}

The gender pay gap variable uses mean annual gross earnings from the Eurostat Structure of Earnings Survey. Annual earnings include regular pay, shift premiums, and performance-related pay. Irregular bonuses are excluded for comparability across different payment systems. All values are adjusted for purchasing power parity with Eurostat harmonized indices to enable valid cross-national comparisons. 

A key limitation is that annual earnings do not control for hours worked. Women work substantially fewer hours than men on average across European labor markets, particularly due to higher part-time employment rates. This means the observed gender pay gaps reflect both hourly wage differences and working time differences. However, the analysis remains valid for three reasons. First, the research question focuses on total earnings inequality—which matters for household income, pension accumulation, and economic security—regardless of whether gaps arise from hourly wages or working hours. Second, part-time work itself often reflects constrained choices due to caregiving responsibilities and limited flexible full-time opportunities, making it an outcome of labor market discrimination rather than purely voluntary preference. Third, the Structure of Earnings Survey samples employees from establishments, minimizing but not eliminating bias from women's differential selection into part-time versus full-time positions within the same sector-occupation cells. Future research with hourly wage data would complement these findings by isolating pure wage discrimination from working time effects.

\subsection{Sectoral and Occupational Classification Rationale}

The analytical framework employs theoretically grounded aggregations of NACE Rev. 2 sectors and ISCO-08 occupations to identify systematic patterns in gender wage determination while maintaining sufficient statistical power for robust inference. This subsection explicates the conceptual logic and empirical rationale underlying the variable construction strategy.

\subsubsection{Sectoral Aggregation Framework}

The 18 detailed NACE Rev. 2 economic sectors are grouped into four broad categories. These are based on institutional characteristics, labor market structures, and gender equality mechanisms that, as theory and prior research show, systematically affect wage determination processes.

The Industry Sector (NACE B-E) includes Mining (B), Manufacturing (C), Electricity/Gas (D), and Water Supply (E). This grouping reflects traditional extractive and production industries. These sectors have a male-dominated workforce with strong occupational sex segregation. They are capital-intensive, requiring physical labor and technical skills historically linked to male employment. Powerful sectoral unions use seniority-based wage systems that may disadvantage women with interrupted careers. Many organizations emphasize long hours and physical presence, which can penalize work-family balance. These sectors exhibit the most significant gender wage gaps in European labor markets. They are vital for analyzing discrimination in traditional industries.

The Construction Sector (NACE F) is its own category due to unique labor-market features. Female participation is extremely low, leading to pronounced statistical discrimination and tokenism. Project-based jobs with high mobility and informal networks often disadvantage women. Subcontracting and self-employment can weaken regulatory oversight of equal pay. Health hazards and physical demands are frequently used to justify exclusion. The construction sector needs a separate analysis given its gender composition and institutional structure.

The Services Sector (NACE G-N, R-S) covers Trade (G), Transport (H), Hospitality (I), IT (J), Finance (K), Real Estate (L), Professional and Administrative Services (M, N), Arts/Entertainment/Recreation (R), and Other Services (S). This broad category captures the diverse private service economy. There is substantial variation in gender composition. Transport and IT are male-dominated, while hospitality and administrative support often have more women. Wages are determined by the market, with performance pay and bonuses that can enable discrimination. Social and communication skills are increasingly important, often benefiting female workers in customer-facing roles. Institutional differences are strong, ranging from highly regulated finance to informal hospitality. Arts and Other Services are included due to distinct gender dynamics and high female representation. These sectors often have compressed wage structures but lack strong formal mechanisms to enforce equality. The variety within Services helps identify sector-specific mechanisms. 

The Public Sector (NACE O-Q) includes Public Administration (O), Education (P), and Health (Q). These sectors have unique institutional features that theory predicts should compress the gender wage gaps. They use formalized wage grids and transparent compensation, limiting managerial discretion. Legal frameworks enforce equal pay and include public accountability. Female representation is high, especially in education and health. This reduces tokenism and statistical discrimination. There is also strong collective bargaining and union coverage to protect wage equality. Explicit diversity policies, required by EU directives, are common.

\textbf{Services} (NACE G-N, S, reference category) represents the baseline sectoral category against which all other sectors are compared. This broad category encompasses Trade (G), Transport (H), Hospitality (I), IT (J), Finance (K), Real Estate (L), Professional Services (M), Administrative Services (N), and Other Services (S). Services are chosen as the reference category because they constitute the largest employment share (57.0\% of observations), exhibit intermediate gender wage gap levels between Industry (higher gaps) and Public Sector (lower gaps), and provide a theoretically neutral baseline for interpreting sectoral effects. The heterogeneity within Services is addressed in the detailed 18-sector analysis (Section 6.5), while the primary models use Services as a composite reference.

This 4-category aggregation balances two analytical goals. First, it ensures sufficient cell sizes for stable estimation of interaction effects and country group heterogeneity. This would not be possible with 18 separate sectors and multiple interactions. Second, it preserves relevant differences in institutional structures affecting gender wage determination. The primary models employ this 4-sector scheme, while supplementary analysis (Table \ref{tab:descriptive_stats}, Panel C) presents the full 18-sector disaggregation to demonstrate that aggregation does not obscure critical heterogeneity.

\subsubsection{Occupational Hierarchy Classification}

The ISCO-08 job classification system employs two binary indicators, each reflecting distinct aspects of occupational ranking, which is essential for understanding the determinants of gender pay differentials.

High-Skill Occupation (ISCO 1-3): This variable covers the top three ISCO-08 groups: (1) Managers, (2) Professionals, and (3) Technicians and Associate Professionals. Human capital theory and credentialism guide this grouping. These roles require tertiary education or vocational training, impose high cognitive demands and autonomy, offer career progression based on performance and manager judgment, and feature compensation based on performance, bonuses, and subjective evaluation that may allow discrimination.

The high-skill classification reflects labor-market stratification, disadvantaging women relative to those in routine occupations. While women achieve educational parity or advantage in European countries, several barriers hinder the effective translation of their credentials into occupational rewards. These include: (1) statistical discrimination regarding perceived career commitment and availability for long hours; (2) gendered networks and sponsorship affecting promotions; (3) penalties for career interruptions, which disproportionately impact mothers in professional tracks; and (4) masculine cultures in male-dominated professions (Weinberger, 2011). Thus, the high-skill indicator identifies situations where investments in human capital should yield equal returns; however, discrimination persists due to discretionary evaluation.

Managerial Position (ISCO 1): Focusing on managers aligns with research on glass ceilings and organizational hierarchies. ISCO-08 major group 1 covers chief executives, senior officials, and managers in administration, commerce, production, services, and retail. Factors contributing to the gender wage gap in these roles include: (1) promotion processes that may involve selection bias, (2) unclear, discretionary pay, (3) pressure for constant availability and geographic mobility, (4) male-dominated senior networks, and (5) negotiation dynamics that can disadvantage women.

The managerial indicator shows that the gender wage gap is larger at the top of organizations, supporting glass-ceiling theories. It also examines whether promotion systems and transparency may have reduced pay gaps in leadership. By measuring both high-skill and managerial roles, the analysis distinguishes between professional/technical and leadership jobs. This helps explore if discrimination varies across different hierarchy levels in skilled jobs.

Reference Category (ISCO 4-9): The reference group consists of clerical support workers (ISCO 4), service and sales personnel (ISCO 5), craft and related trades workers (ISCO 7), plant and machine operators (ISCO 8), and those in elementary occupations (ISCO 9). Agricultural workers (ISCO 6) constitute a negligible proportion of the sample due to the survey's focus on enterprises employing at least 10 individuals and on formal employment. These routine and manual occupations are marked by: (1) lower educational prerequisites with reliance on on-the-job training; (2) standardized wage systems with limited performance-related pay; (3) greater collective bargaining coverage, especially in manufacturing and public sectors; and (4) enhanced measurability of productivity, thus constraining the latitude for discriminatory evaluation. While gender wage disparities persist in these occupations, the underlying mechanisms differ from those observed in skilled professional contexts, with greater emphasis on occupational sex segregation and the undervaluation of female-typed work rather than on barriers to career progression (Levanon, 2009).

This occupational classification strategy enables the analysis to test whether gender wage gaps are larger in high-skill and managerial positions, where discretionary compensation and promotional systems create opportunities for discrimination, or whether transparency and equal opportunity policies have been more effective in professional contexts than in routine occupations.


\subsection{Welfare Regime Classification and Institutional Context}

The cross-national analysis uses a welfare regime typology to capture how labor market structures and gender equality policies vary across European countries. This classification, initially developed by \citep{espingandersen1990}, has been substantially updated and extended since 1990 to account for post-socialist transitions \citep{mandel2007}, Southern European familial welfare systems \citep{olivetti2008}, and gender-specific institutional effects \citep{mandel2005}. This refined framework helps explain how national institutional systems influence gender wage outcomes in contemporary European labor markets.

Welfare regimes are made up of state policies, labor market institutions, and family support systems. Together, these elements systematically shape employment patterns, wage-setting practices, and gender equality outcomes. Regime types take different approaches to social protection, labor-market rules, childcare, and the balancing of work and family. All these factors influence how gender wage gaps arise.

\textbf{Regime Classification and Characteristics:}

The analysis first considers the Nordic countries (Denmark, Finland, Iceland, Norway, and Sweden), characterized by universal welfare provision, high taxation, extensive public-sector employment, and strong labor unions. (1) Labor market regulation features centralized wage bargaining and compressed wage structures, ensuring coordinated wage-setting across sectors with strong job protection. (2) These countries implement the most comprehensive family policies in Europe, including heavily subsidized childcare, generous parental leave with use-it-or-lose-it paternal quotas, and dual-earner support systems. (3) Gender equality policies are most developed, with explicit equal pay enforcement, active monitoring mechanisms, and affirmative action in public employment.

Continental countries include Austria, Belgium, France, Germany, Luxembourg, the Netherlands, and Switzerland. These are conservative-corporatist welfare states with insurance-based social protection and a history of male-breadwinner traditions, though modernization is underway. (1) Wage-setting focuses on social partnership and industry-level bargaining, with moderate labor market regulation balancing employer flexibility with worker protection. (2) Traditional gender roles historically shaped family policies with limited childcare support, but recent reforms have substantially improved childcare provision and parental leave generosity. (3) Gender equality enforcement relies primarily on legal frameworks and collective bargaining agreements, with moderate monitoring capacity and gradual implementation of EU directives.

Mediterranean countries include Cyprus, Greece, Italy, Malta, Portugal, and Spain. Their welfare states developed late and rely on family-based social protection, with large informal economic sectors and traditional gender norms. (1) Labor markets are split between secure, permanent jobs and insecure, temporary positions, creating dual structures with strong protection for permanent workers but weak overall regulation due to extensive temporary contracts. (2) Family networks play a larger role than state aid in childcare provision, with limited public support and below-average parental leave benefits, though improvements have occurred since EU accession. (3) Gender equality policies have improved since EU membership, but enforcement lags behind Northern Europe due to limited monitoring resources and persistent traditional norms.

Liberal countries are Ireland and the United Kingdom. These countries favor market-driven labor regulation, minimal welfare, and weak unions. (1) Wage-setting is decentralized with flexible employment contracts, limited job protection, and high inequality resulting from minimal labor market regulation. (2) Family policy is limited, with most childcare privately provided and costly, minimal parental leave compared to EU standards, and emphasis on market-based solutions over state support. (3) Gender equality efforts stress formal equal opportunity laws and anti-discrimination legislation over structural interventions, with enforcement through individual litigation rather than proactive monitoring.

Eastern European countries include Bulgaria, Croatia, Czechia, Estonia, Hungary, Latvia, Lithuania, Moldova, Poland, Romania, Slovakia, and Slovenia. These post-socialist economies retained formal gender equality from the communist era, including high female employment and state-supported childcare. After marketization, they faced significant institutional changes. (1) Labor market regulation is transitional, with current arrangements mixing socialist-era collective agreements with increasing market flexibility, weakening job protection, and declining union coverage amid economic liberalization and pressures to meet EU standards. (2) Family policies retain socialist-era infrastructure including state-supported childcare and guaranteed parental leave, though funding constraints have reduced service quality and availability compared to the communist period. (3) Gender wage gaps show the effects of inherited formal legal equality from socialism, but enforcement has weakened during market transitions, with limited monitoring capacity and emerging market-based discrimination patterns.

Balkan countries are Albania, Bosnia and Herzegovina, Kosovo, Montenegro, North Macedonia, and Serbia. These countries are EU candidates or potential candidates with weak institutions, transitional economies, and traditional gender roles. (1) Labor market regulation is weak and developing, with limited collective bargaining coverage, extensive informal employment beyond regulatory reach, and inconsistent job protection enforcement. (2) Family policies are minimal, with limited public childcare provision, below-average parental leave benefits, and heavy reliance on traditional family structures for caregiving support. (3) Although they have formal equality laws required for EU candidacy, enforcement is weak due to limited monitoring institutions, scarce implementation resources, and persistent traditional gender norms.

Turkey is classified separately due to its unique institutional setup, combining Islamic cultural influences, rapid economic growth, regional diversity, and some degree of European integration. (1) Labor market regulation is moderate but uneven, with formal protections in large enterprises coexisting with extensive informal sector employment beyond regulatory coverage. (2) Family policies remain limited compared to EU standards, with minimal public childcare provision and traditional expectations regarding women's caregiving roles, though recent reforms have expanded parental leave provisions in formal employment sectors. (3) Gender equality enforcement is developing, with legal frameworks increasingly aligned with EU standards but implementation hampered by traditional cultural norms, particularly regarding women's labor force participation.

This classification enables us to examine how national institutions shape sectoral and occupational wage gaps. The regime typology highlights differences in three key factors affecting gender wage equality: (1) how strongly the labor market is regulated (centralized or decentralized wage-setting, job protection strength); (2) how generous family policies are (childcare, parental leave, work-life support); and (3) how well gender equality is enforced (laws, monitoring, affirmative action). By using country groups and sector interactions, the analysis tests whether wage-gap causes are the same everywhere or depend on the welfare-state setup.

\subsection{Data Cleaning and Quality Assurance Procedures}

Rigorous data cleaning protocols ensure the analytical validity and reliability of the results. The cleaning process implements sequential filters addressing data quality at multiple levels:

\textbf{Stage 1: Establishment-Level Validation}
\begin{itemize}
\item Removal of establishments with fewer than 10 employees to ensure statistical reliability of within-unit gender comparisons
\item Exclusion of establishments reporting implausible wage distributions (coefficient of variation > 3)
\item Verification of NACE classification consistency across survey waves
\end{itemize}

\textbf{Stage 2: Individual-Level Cleaning}
\begin{itemize}
\item Trimming of annual earnings at the 1st and 99th percentiles within country-year cells to eliminate coding errors
\item Exclusion of observations with incomplete sectoral or occupational classifications
\item Validation of employment consistency (removing observations with implausible earnings distributions)
\end{itemize}

\textbf{Stage 3: Cell-Level Aggregation Quality}
\begin{itemize}
\item Requirement of a minimum of 30 observations per country-sector-occupation-year cell for reliable gap calculation
\item Suppression of cells with single-gender composition (preventing gap computation where women or men are absent)
\item Winsorization of calculated gaps at extreme values (5th and 95th percentiles) to address measurement error
\end{itemize}

These procedures ensure robust estimation across 14,430 panel observations representing 5,176 unique country-sector-occupation cells across 40 European countries over four time periods (2010, 2014, 2018, 2022). The cleaning process prioritizes maintaining adequate cell sizes for statistical precision while ensuring methodological consistency across countries and time periods.

\subsection{Missing Data Analysis and Treatment}

Missing data patterns exhibit systematic variation requiring careful analytical treatment. Missingness occurs at two levels: survey non-participation by countries in specific waves (unit non-response) and incomplete variable coverage within participating countries (item non-response).

\textbf{Unit Non-Response Patterns:}
\begin{itemize}
\item Bulgaria and Romania: Entry in 2010 following EU accession
\item Croatia: Entry in 2014 post-accession
\item Greece: Missing 2014 wave due to administrative constraints
\item Ireland and Denmark: Intermittent participation due to national statistical priorities
\end{itemize}

The analysis addresses unit non-response using two approaches. Primary models use unbalanced panel methods, allowing countries to enter and exit the data set at different times. Robustness checks use balanced sub-samples that include only countries present in all periods.


\textbf{Item Non-Response Treatment:}

Variable-specific missing data rates are below 2\% for the main variables (sectoral classification, occupational coding, wage information). The gender pay gap is calculated for each country, sector, occupation, and year combination (a cell). To address missing individual responses, only cells with at least 30 observations (**$N \geq 30$**) are included, ensuring enough data for reliable estimates. Cells with fewer observations are excluded to prioritize accuracy over total sample size. Imputation is not needed in this approach, as the analysis uses summary data from each cell rather than original individual responses.

\subsection{Panel Structure and Identification Strategy}

The constructed panel dataset exhibits a hierarchical structure. It comprises 14,430 observations nested within 5,176 unique country-sector-occupation panels, observed across four time periods: 2010, 2014, 2018, and 2022. Panel duration varies from 1 to 4 observations. Of these, 2,223 panels (43.0\%) have complete four-period coverage. This structure enables the identification of within-panel wage gap evolution while controlling for time-invariant unobserved heterogeneity.

The identification strategy uses within-panel variation over time. It effectively compares changes in wage gaps within identical country-sector-occupation cells. This approach eliminates bias from stable, unobserved factors, such as cultural attitudes, industrial relations systems, or occupational prestige, that may correlate with both gender composition and wage levels. Substantial within-panel variation confirms adequate identifying variation for fixed effects estimation.



\subsection{Descriptive Sample Characteristics}

The analytical sample exhibits balanced sectoral distribution: Services account for 57.0\% of observations (8,224), Industry 19.4\% (2,806), Public Sector 17.9\% (2,588), and Construction 5.6\% (812). This balance ensures coverage across diverse institutional contexts while providing sufficient within-sector variation for robust estimation.


\section{METHODOLOGICAL DESIGN}

This study uses a sophisticated panel data econometric framework to examine the determinants of gender wage gaps across European labor markets. The approach integrates multiple estimation strategies to ensure robust identification while addressing challenges in wage data, such as unobserved heterogeneity, selection bias, and potential endogeneity.

\subsection{Panel Data Econometric Framework}

The empirical strategy leverages the panel structure of the SES data to examine relationships between sectoral and occupational characteristics and gender wage differentials. The baseline specification employs the following general form:

\begin{equation}
GPG_{it} = \alpha_i + \beta_1 SECTOR_{it} + \beta_2 OCC_{it} + \gamma_t + \epsilon_{it}
\end{equation}

where $GPG_{it}$ represents the gender pay gap for panel unit $i$ in period $t$, $\alpha_i$ captures time-invariant panel-specific effects, $SECTOR_{it}$ denotes sectoral indicator variables (Industry, Construction, Services, Public Sector), $OCC_{it}$ represents occupational hierarchy measures (High-Skill, Managerial), $\gamma_t$ captures temporal fixed effects, and $\epsilon_{it}$ represents the error term \citep{wooldridge2010}.

The model specification reflects both the data structure and the research objectives. The Eurostat Structure of Earnings Survey provides aggregated data at the country-sector-occupation-year level. The dataset contains three key variables: the gender pay gap (dependent variable), sectoral classification, and occupational classification. Individual-level characteristics (education, experience, hours worked) are not available in this aggregated format. Each observation represents the average wage gap for a specific country-sector-occupation combination in a given year, already calculated from underlying individual worker data by Eurostat.

These variables directly address the study's theoretical focus on structural and institutional determinants of wage inequality. Sectoral classification captures one of the most significant drivers of gender pay gaps: occupational segregation across industries. Women and men systematically sort into different sectors with varying pay structures, unionization rates, and regulatory environments. For example, public sector employment typically features stronger equal pay enforcement and greater wage transparency than private industry. Occupational hierarchy measures (high-skill occupations and managerial positions) reveal whether wage gaps stem from barriers to advancement (the "glass ceiling" effect) or from entry-level discrimination. These structural factors are directly policy-relevant—governments can target equal pay enforcement in specific sectors or address occupational segregation through targeted interventions. Moreover, sectoral and occupational classifications are internationally standardized (NACE Rev. 2 and ISCO-08), enabling meaningful cross-national comparisons across 40 European countries. The panel fixed effects ($\alpha_i$) control for all time-invariant characteristics of each country-sector-occupation cell, including stable workforce composition, skill requirements, and institutional features. This specification directly tests the study's three core hypotheses about sectoral heterogeneity, occupational hierarchies, and institutional differences, following established comparative institutional approaches \citep{mandel2007, olivetti2008}.

The specification deliberately separates sectoral and occupational effects. This design examines their independent and interactive influences on wage inequality. It advances beyond existing literature, which typically studies these dimensions in isolation. As a result, it enables the identification of complementarity or substitution effects between institutional (sectoral) and hierarchical (occupational) mechanisms of wage discrimination.

\subsection{Estimator Selection and Diagnostic Procedures}

The choice between fixed effects (FE) and random effects (RE) estimators requires careful consideration of identification assumptions and data structure constraints. The FE estimator, employing within-panel transformation, eliminates time-invariant unobserved heterogeneity but sacrifices identification of time-invariant regressors:

\begin{equation}
(GPG_{it} - \overline{GPG}_i) = \beta_1(SECTOR_{it} - \overline{SECTOR}_i) + \beta_2(OCC_{it} - \overline{OCC}_i) + (\epsilon_{it} - \overline{\epsilon}_i)
\end{equation}

Conversely, the RE estimator assumes orthogonality between unobserved effects and regressors, enabling identification of all coefficients while potentially introducing bias under correlation:

\begin{equation}
GPG_{it} = \alpha + \beta_1 SECTOR_{it} + \beta_2 OCC_{it} + \gamma_t + (\alpha_i - \alpha + \epsilon_{it})
\end{equation}

The choice between FE and RE estimators is not merely theoretical—it requires empirical testing. This study formally tests the appropriateness of each estimator using Hausman specification tests. The Hausman test evaluates the consistency of RE estimates under the null hypothesis that there are no systematic differences between FE and RE estimators. If the test rejects the null hypothesis (p < 0.05), it indicates that the RE assumption of orthogonality between unobserved effects and regressors is violated, and FE estimation is preferred. The test statistic follows a chi-squared distribution with degrees of freedom equal to the number of time-varying coefficients. Section 6 presents the Hausman test results for all primary specifications.

\subsection{Robust Inference and Clustering Strategies}

Panel data from multiple countries and time periods can violate standard statistical assumptions in two important ways. First, observations within the same country-sector-occupation combination are not independent—they are correlated over time (serial correlation). Second, the variability of wage gaps may differ systematically across panels (heteroskedasticity). Diagnostic tests confirm both issues are present in this dataset: the Breusch-Godfrey test detects significant serial correlation ($\chi^2$ = 1144.3, p < 0.001), and the Breusch-Pagan test reveals heteroskedasticity (BP = 181.73, p < 0.001). Ignoring these problems would produce unreliable standard errors and incorrect statistical inference.

To address these violations, the analysis employs cluster-robust standard errors that account for correlation within panels. The baseline approach uses HC1 heteroskedasticity-consistent standard errors clustered at the panel level (country-sector-occupation cells). This allows observations within the same panel to be correlated while treating different panels as independent. The robustness analysis examines sensitivity to alternative clustering assumptions: two-way clustering (simultaneously by panel and year), country-level clustering (40 clusters), and wild cluster bootstrap methods (1,000 replications). These alternative approaches produce slightly larger standard errors but maintain the statistical significance of all main findings, demonstrating that the results are robust to different assumptions about error correlation structures.

The Hausman test ($\chi^2$ = 104.72, df = 3, p < 2.2e-16) strongly rejects the null hypothesis of no systematic differences between fixed and random effects estimators, indicating that Fixed Effects estimation is statistically preferred. The analysis therefore employs Fixed Effects as the primary specification to control for time-invariant unobserved heterogeneity, with cluster-robust standard errors to address serial correlation and heteroskedasticity. Random Effects models with interaction terms are used as complementary specifications to examine sector-occupation interactions, as these cannot be identified within the Fixed Effects framework due to collinearity with panel fixed effects.

\subsection{Endogeneity Concerns and Identification Strategies}

Potential endogeneity arises from three primary sources requiring distinct identification strategies:

1. Reverse Causality: Wage gaps may affect sectoral employment through selection effects. The analysis relies on an employer-based sampling frame. This ensures that individual sorting choices do not change establishment-level wage structures within survey periods.

2. Omitted Variable Bias: Unobserved productivity differences, if linked to gender composition, can bias estimates. Panel fixed effects remove time-invariant confounders. Rich occupational controls help capture skill requirements. The robustness analysis examines coefficient stability across multiple specifications with varying control sets to assess sensitivity to potential unobserved confounders.

3. Measurement Error: Classical measurement error in gap calculations attenuates coefficients toward zero (classical attenuation bias). The analysis minimizes this through robust cell-level aggregation requirements (minimum N ≥ 30 per cell) and winsorization of extreme values at the 5th and 95th percentiles, ensuring that measurement variability does not dominate true wage gap signals.


\subsubsection{Identification Assumptions and Limitations}

Panel data methods improve upon cross-sectional analysis but rest on several key assumptions. This subsection discusses the main identification assumptions underlying the empirical strategy and acknowledges important limitations.

\textbf{Assumption 1 (Exogeneity of Sectoral and Occupational Classifications):} The analysis assumes that sectoral and occupational categories are not systematically correlated with unobserved factors affecting gender pay gaps. This assumption is more plausible than it might initially appear because the classifications reflect structural and institutional features rather than individual worker choices. NACE sectoral codes (Industry, Construction, Services, Public Sector) are determined by the type of economic activity and production technology—a manufacturing firm is classified as Industry regardless of its wage structure. Similarly, ISCO occupational codes (Managers, Professionals, Technicians, etc.) are based on the skill level and task content of jobs, not on the gender composition or pay levels within those roles. This means that a country-sector-occupation cell's gender pay gap does not determine how it is classified; the classification is predetermined by objective criteria about what the sector produces and what tasks the occupation involves. However, this assumption could still be violated if women systematically sort into specific sectors or occupations based on anticipated discrimination, a limitation the study cannot fully address with aggregate data.

\textbf{Assumption 2 (Time-Invariant Unobserved Heterogeneity):} The panel fixed effects ($\alpha_i$) control for stable characteristics of each country-sector-occupation combination that do not change over the study period. These might include deep-rooted cultural attitudes toward gender roles, long-standing industrial relations traditions, or the historical prestige of certain occupations. However, the 12-year study period (2010-2022) is sufficiently long that some of these factors may have changed. For example, cultural attitudes toward working mothers may have shifted, new equal pay legislation may have been enacted, or unionization rates may have declined in certain sectors. The panel fixed effects cannot account for these time-varying changes, which represents an important limitation. The time fixed effects ($\gamma_t$) capture common trends across all panels but cannot control for panel-specific evolution. This suggests the estimates should be interpreted as associations between sectoral/occupational structure and wage gaps rather than definitive causal effects.

\textbf{Assumption 3 (Common Trends for Beta Convergence Analysis):} The convergence analysis examines whether countries with larger initial gender pay gaps in 2010 experienced faster reductions by 2022. Beta convergence refers to the phenomenon where countries starting from worse positions "catch up" to better-performing countries over time, leading to convergence in outcomes. In this context, it means that a country with a 25\% gender pay gap in 2010 would be expected to reduce its gap more rapidly than a country starting at 10\%, potentially due to greater policy attention, more room for improvement, or EU-wide harmonization pressures. The identification of beta convergence assumes that countries would have experienced parallel trends in the absence of their different starting positions—that is, the only reason high-gap countries reduced gaps faster is because they started higher, not because of other concurrent factors. This parallel trends assumption cannot be directly tested. However, it gains some credibility from three observations: all countries faced similar macroeconomic shocks (financial crisis recovery, COVID-19), EU member states share common regulatory frameworks (equal pay directives, employment protection legislation), and robustness checks limited to Eurozone countries (which share monetary policy) yield consistent convergence patterns.

\textbf{Limitations and Robustness Checks:}

The study acknowledges several limitations that could not be addressed with the available aggregate data. First, if gender pay gaps influence women's labor force participation decisions, observed gaps may underestimate true wage discrimination because women in high-inequality sectors may exit employment entirely (selection bias). The employer-based SES sampling reduces but cannot eliminate this concern. Second, sectoral employment shares change over time (manufacturing decline, service sector growth), which could confound temporal trends with compositional shifts. The panel fixed effects address this by examining changes within country-sector-occupation cells, but within-sector occupational upgrading remains a potential confounder. Third, gender pay gaps calculated from cell averages contain sampling variability, particularly in smaller cells. The minimum cell size requirement ($N \geq 30$) and winsorization of extreme values mitigate but do not eliminate measurement error, which may attenuate coefficient estimates.

To assess the robustness of the findings to these concerns, the study conducts comprehensive sensitivity analyses reported in Section 6.8 and summarized in Tables 13-15. These include: alternative sample restrictions (balanced panels, large countries only, extreme value exclusions), alternative estimation methods (quantile regression, different clustering structures), and subsample analyses (EU-15 vs. new member states, pre-pandemic vs. post-2014 periods, geographic restrictions). The consistency of results across these 15+ alternative specifications provides confidence that the main findings reflect genuine empirical patterns rather than artifacts of particular modeling choices or data limitations.

\subsection{Heterogeneous Treatment Effects and Interaction Analyses}

The econometric framework extends beyond average effects to examine heterogeneous impacts across institutional contexts, using multiple complementary specifications that address each research hypothesis.

\textbf{Model Specification 1: Detailed Sectoral Analysis (Hypothesis 1)}

To test Hypothesis 1 regarding sectoral determinants and negative gaps, the analysis employs a disaggregated sectoral specification utilizing all 18 NACE Rev. 2 sectors:

\begin{equation}
\begin{split}
GPG_{it} = \alpha + \beta_1 Mining_{it} + \beta_2 Manufacturing_{it} + \beta_3 Electricity_{it} \\
+ \beta_4 Water_{it} + \beta_5 Construction_{it} + \beta_6 Trade_{it} + \beta_7 Transport_{it} \\
+ \beta_8 Hospitality_{it} + \beta_9 IT_{it} + \beta_{10} Finance_{it} + \beta_{11} RealEstate_{it} \\
+ \beta_{12} Professional_{it} + \beta_{13} Admin_{it} + \beta_{14} PublicAdmin_{it} \\
+ \beta_{15} Education_{it} + \beta_{16} Health_{it} + \beta_{17} Arts_{it} \\
+ \beta_{18} HighSkill_{it} + \beta_{19} Managerial_{it} + \gamma_t + \epsilon_{it}
\end{split}
\end{equation}

where each sector indicator variable equals 1 if panel $i$ belongs to that sector and 0 otherwise, with Other Services as the omitted reference category. $HighSkill_{it}$ indicates whether the occupational category comprises Managers, Professionals, or Technicians (ISCO 1-3), while $Managerial_{it}$ indicates Managers only (ISCO 1). The coefficients $\beta_1$ through $\beta_{17}$ measure sector-specific gaps relative to Other Services, controlling for occupational composition. $\gamma_t$ captures year fixed effects, and $\epsilon_{it}$ is the idiosyncratic error term. This Random Effects specification enables identification of all sector coefficients including the constant term. The disaggregated approach addresses concerns about insufficient variation in broad sector classifications by revealing heterogeneity across detailed industrial classifications. Note that this specification focuses on sectoral heterogeneity within panels; cross-national differences across welfare regimes are examined separately in Model Specification 3.

\textbf{Model Specification 2: Sector-Occupation Interactions (Hypothesis 2)}

Hypothesis 2 predicts that occupational penalties vary systematically across sectoral contexts. The interaction specification examines complementarity between institutional and hierarchical mechanisms:

\begin{equation}
\begin{split}
GPG_{it} = \alpha_i + \beta_1 Industry_{it} + \beta_2 Construction_{it} + \beta_3 PublicSector_{it} \\
+ \beta_4 HighSkill_{it} + \beta_5 Managerial_{it} \\
+ \beta_6(Industry \times HighSkill)_{it} + \beta_7(Industry \times Managerial)_{it} \\
+ \beta_8(PublicSector \times HighSkill)_{it} + \beta_9(PublicSector \times Managerial)_{it} \\
+ \gamma_t + \epsilon_{it}
\end{split}
\end{equation}

where Services is the omitted reference category for sectoral effects. The interaction coefficients $\beta_6$ through $\beta_9$ quantify how occupational wage gaps differ across sectoral institutional environments. Significant interactions indicate that glass ceiling effects and skill-based wage differentials operate contingently rather than uniformly across labor-market segments.

\textbf{Model Specification 3: Country Group Analysis (Hypothesis 3 - Part 1)}

To test institutional determinants of cross-national variation, the analysis incorporates country group classifications based on welfare regime typology:

\begin{equation}
\begin{split}
GPG_{it} = \alpha_i + \beta_1 Industry_{it} + \beta_2 Construction_{it} + \beta_3 PublicSector_{it} \\
+ \beta_4 HighSkill_{it} + \beta_5 Managerial_{it} \\
+ \delta_1 Nordic_i + \delta_2 Mediterranean_i + \delta_3 Eastern_i + \delta_4 Liberal_i \\
+ \delta_5 Balkans_i + \delta_6 Other_i \\
+ \theta_1(Nordic \times PublicSector)_{it} + \theta_2(Mediterranean \times Industry)_{it} \\
+ \theta_3(Eastern \times Industry)_{it} + \gamma_t + \epsilon_{it}
\end{split}
\end{equation}

where Continental is the omitted reference category for country groups, and Services is the omitted reference for sectors. The specification includes 4 sectoral categories (Industry, Construction, Services, Public Sector), 2 occupational indicators (High-Skill, Managerial), and 7 country group classifications (Nordic, Continental, Mediterranean, Eastern European, Liberal, Balkans, Other). The main effects $\delta_1$ through $\delta_6$ identify baseline differences in gender wage equality across institutional regimes relative to Continental countries. The interaction terms $\theta_1$, $\theta_2$, and $\theta_3$ test whether specific sectoral wage structures are moderated by welfare regime type, examining whether public-sector advantages in Nordic countries and industry penalties in Mediterranean and Eastern European countries differ from the Continental baseline pattern.

\textbf{Model Specification 4: Beta Convergence Analysis (Hypothesis 3 - Part 2)}

The convergence hypothesis predicts that countries with larger initial gaps experience faster subsequent reductions due to EU policy harmonization and competitive pressures. This specification employs a cross-sectional regression estimated on 30 European countries:

\begin{equation}
\Delta GPG_i = \alpha + \beta \cdot GAP_{2010,i} + \epsilon_i, \quad i = 1, \ldots, 30
\end{equation}

where $\Delta GPG_i = GAP_{2022,i} - GAP_{2010,i}$ represents the total change in country $i$'s gender pay gap over the 12-year period, and $GAP_{2010,i}$ denotes the baseline gap level in 2010. Each country contributes one observation to this regression. A negative coefficient, $\beta < 0$, indicates beta convergence: countries starting with higher gaps experienced larger absolute reductions. This cross-sectional approach differs fundamentally from the panel specifications (Models 1-3), which exploit within-country temporal variation. Here, the unit of analysis is the country rather than the country-sector-occupation panel, testing whether initial conditions predict convergence rates across European labor markets.

\subsection{Robustness and Sensitivity Analyses}

To assess the stability of findings, the analysis implements 17 robustness checks organized in three categories (detailed results in Section 6.8, Tables 13-15):

\textbf{Table 13 - Alternative Specifications (7 checks):} The baseline random effects model is compared against six alternatives: (1) balanced panel restriction requiring complete 2010-2022 coverage, (2) exclusion of extreme gaps beyond ±3 standard deviations, (3) large-country subsample with populations exceeding 5 million, (4) quantile regression at the median to reduce outlier sensitivity, (5) fixed effects specification, and (6) winsorization of the dependent variable at the 1st and 99th percentiles.

\textbf{Table 14 - Clustering Structures (4 checks):} Standard errors are recalculated under four clustering assumptions: (1) panel-level clustering (baseline), (2) two-way clustering by panel and year, (3) country-level clustering with 40 clusters, and (4) wild cluster bootstrap with 1,000 replications to address concerns about small cluster counts.

\textbf{Table 15 - Geographic and Temporal Subsamples (6 checks):} Coefficient stability is examined across: (1) EU-15 founding members only, (2) new member states that joined post-2004, (3) Eurozone countries only, (4) pre-pandemic period (2010-2018), (5) post-2014 period following survey methodology changes, and (6) 2022 cross-section to verify patterns persist in the most recent data.

\subsection{Statistical Software and Computational Implementation}

All analyses use R (version 4.3.1), ensuring reproducibility through scripted workflows. Core estimation employs the \texttt{plm} package for panel data models, \texttt{lmtest} and \texttt{sandwich} for robust inference, and \texttt{fixest} for high-dimensional fixed effects. Custom functions implement two-way clustering and wild bootstrap procedures, validated against Stata implementations for cross-platform consistency.

Computational efficiency considerations guide implementation choices, particularly for bootstrap procedures requiring iterative estimation. Parallel processing across eight cores reduces computation time for wild bootstrap confidence intervals (10,000 replications) from 4 hours to 35 minutes. Memory-efficient sparse matrix representations enable the accommodation of high-dimensional fixed effects without computational constraints.

The integrated methodological framework thus provides a rigorous identification of the determinants of the gender wage gap, while acknowledging the inherent limitations of observational data. Through multiple estimation strategies, robust inference procedures, and comprehensive sensitivity analyses, the approach generates credible causal estimates that advance our understanding of the mechanisms underlying discrimination in European labor markets.

\section{RESULTS}

The empirical analysis reveals complex, multidimensional patterns of gender wage determination across European labor markets. The analysis leverages harmonized Structure of Earnings Survey data, comprising 14,430 observations across 5,176 unique country-sector-occupation panels spanning 40 European countries from 2010 to 2022, after applying robust outlier filtering and data quality procedures described in Section 4.2.

\subsection{Descriptive Statistics and Sample Characteristics}

Table \ref{tab:descriptive_stats} presents summary statistics disaggregated by key analytical dimensions, revealing substantial heterogeneity in the magnitudes of the gender pay gap across observational units.


\begin{table}[htbp]
\centering
\caption{Descriptive Statistics: Gender Pay Gap by 18 Detailed NACE Sectors}
\label{tab:descriptive_stats}
\small
\begin{tabularx}{\textwidth}{l *{5}{>{\centering\arraybackslash}X} r}
\hline\hline
\textbf{Sector (NACE Rev. 2)} & \textbf{Mean} & \textbf{SD} & \textbf{Min} & \textbf{Max} & \textbf{N} & \textbf{\% Neg.} \\
\hline
\multicolumn{7}{l}{\textit{Panel A: High-Gap Sectors (Mean > 15\%)}} \\
Manufacturing (C) & 17.7 & 9.91 & -18.7 & 55.4 & 1,131 & 4.1 \\
Mining \& Quarrying (B) & 19.3 & 14.6 & -19.7 & 64.5 & 461 & 12.8 \\
Finance \& Insurance (K) & 18.3 & 12.1 & -19.9 & 62.2 & 649 & 8.2 \\
\hline
\multicolumn{7}{l}{\textit{Panel B: Medium-Gap Sectors (12-15\%)}} \\
Wholesale \& Retail Trade (G) & 15.3 & 10.7 & -18.3 & 54.8 & 1,067 & 7.7 \\
Electricity \& Gas Supply (D) & 15.5 & 11.5 & -18.8 & 60.0 & 570 & 8.2 \\
Transportation \& Storage (H) & 15.1 & 12.2 & -19.9 & 78.9 & 868 & 11.1 \\
Construction (F) & 14.7 & 11.3 & -19.3 & 55.5 & 812 & 11.4 \\
Professional Services (M) & 14.2 & 11.8 & -17.8 & 57.0 & 861 & 12.2 \\
IT \& Communication (J) & 13.7 & 11.3 & -19.8 & 56.8 & 760 & 11.1 \\
Arts \& Entertainment (R) & 13.2 & 13.1 & -19.0 & 79.5 & 800 & 13.4 \\
Real Estate (L) & 13.6 & 13.1 & -19.8 & 87.9 & 631 & 15.0 \\
Human Health (Q) & 12.7 & 12.6 & -19.3 & 85.3 & 997 & 15.1 \\
\hline
\multicolumn{7}{l}{\textit{Panel C: Low-Gap Sectors (Mean < 12\%)}} \\
Other Services (S) & 13.1 & 13.3 & -19.8 & 61.0 & 822 & 16.8 \\
Admin \& Support Services (N) & 12.0 & 11.7 & -19.5 & 61.4 & 968 & 16.0 \\
Water Supply \& Waste (E) & 11.1 & 11.7 & -18.7 & 75.4 & 644 & 17.2 \\
Public Administration (O) & 10.8 & 10.6 & -19.3 & 55.5 & 734 & 14.9 \\
Education (P) & 9.78 & 10.9 & -18.9 & 66.8 & 857 & 16.9 \\
Hospitality \& Food Services (I) & 9.71 & 11.1 & -18.6 & 60.0 & 798 & 20.6 \\
\hline
\multicolumn{7}{l}{\textit{Panel D: Temporal Evolution (All Sectors)}} \\
2010 & 15.4 & 12.9 & -317.9 & 87.9 & 4,424 & 10.2 \\
2014 & 13.4 & 11.7 & -190.2 & 78.9 & 3,211 & 11.8 \\
2018 & 13.5 & 11.9 & -176.7 & 79.5 & 3,418 & 12.4 \\
2022 & 12.3 & 11.2 & -195.3 & 85.3 & 3,377 & 13.9 \\
\hline\hline
\end{tabularx}
\begin{tablenotes}[para,flushleft]
\small
\textit{Source:} Own calculation based on: Structure of Earnings Survey, Eurostat, 2010-2022.
\textit{Notes:} Sample: 14,430 observations across 18 NACE Rev. 2 sectors, 40 countries, 2010-2022. Mean and SD in percentage points. \% Neg. indicates the proportion of observations with negative gaps (women out-earning men). Sectors ordered by mean gap within panels. Panel D aggregates across all sectors. Extreme values result from small-cell observations with high sampling variability; the main analysis uses robust estimation with cluster-standard errors.
\end{tablenotes}
\end{table}
\newpage


Industry sectors demonstrate the highest average gender pay gap (16.0\%), followed by Construction (14.7\%) and Services (13.9\%), while Public Sector employment shows the lowest gap (11.2\%). These unconditional differences of 4.8 percentage points between Industry and Public Sector provide initial evidence supporting Hypothesis 1 regarding sectoral institutional effects on gender wage equality. The detailed 18-sector breakdown in Table \ref{tab:descriptive_stats} reveals substantial within-category heterogeneity, with Manufacturing (17.7\%) and Mining (19.3\%) driving the high Industry average, while service sectors with high female labor force participation such as Hospitality (9.71\%) and Education (9.78\%) exhibit the smallest gaps. Notably, sectors with the lowest gaps also demonstrate the highest incidence of negative gaps (women out-earning men): Hospitality (20.6\% of observations), Water Supply (17.2\%), and Education (16.9\%), suggesting that in sectors with high female representation and compressed wage structures, women can achieve wage parity or advantage.

The temporal trend shows consistent convergence, with mean gaps declining from 15.4\% in 2010 to 12.3\% in 2022, a 3.1 percentage-point reduction over the 12 years. This convergence is accompanied by a gradual increase in negative gap incidence from 10.2\% (2010) to 13.9\% (2022), indicating that progress toward equality operates through both gap reduction in male-dominated sectors and expansion of female wage advantages in sectors with high female labor force participation.

\subsection{Panel Regression Estimates}

Table \ref{tab:main_results} presents panel regression estimates addressing the multidimensional determinants of gender pay gaps. Model selection follows econometric diagnostics: the Hausman test ($\chi^2$ = 104.72, df = 3, p < 2.2e-16) strongly favors Fixed Effects estimation to control for time-invariant unobserved heterogeneity. The primary specification employs Fixed Effects with time dummies, while a complementary Random Effects model with sector-occupation interactions preserves all theoretical predictors. Both specifications use HC1 cluster-robust standard errors to address serial correlation and heteroskedasticity.

\begin{table}[H]
\centering
\caption{Gender Pay Gap Determinants: Panel Data Models with Sector-Occupation Interactions}
\label{tab:main_results}
\small
\begin{tabularx}{\textwidth}{l *{2}{>{\centering\arraybackslash}X}}
\hline\hline
\textbf{Variable} & \textbf{Fixed Effects} & \textbf{Random Effects} \\
\hline
\multicolumn{3}{l}{\textit{Sectoral Variables (RE only)}} \\
Industry & -- & 4.392*** \\
 & & (0.512) \\
Construction & -- & 0.829 \\
 & & (0.613) \\
Public Sector & -- & -2.706*** \\
 & & (0.494) \\
& & \\
\multicolumn{3}{l}{\textit{Occupational Hierarchy (RE only)}} \\
High-Skill Occupation & -- & 2.936*** \\
 & & (0.411) \\
Managerial Position & -- & 4.663*** \\
 & & (0.602) \\
& & \\
\multicolumn{3}{l}{\textit{Time Fixed Effects}} \\
Year 2014 & -1.361*** & -1.747*** \\
 & (0.204) & (0.200) \\
Year 2018 & -1.015*** & -1.482*** \\
 & (0.218) & (0.200) \\
Year 2022 & -2.173*** & -2.692*** \\
 & (0.216) & (0.203) \\
& & \\
Constant & -- & 13.406*** \\
 & & (0.257) \\
\hline
\multicolumn{3}{l}{\textit{Model Statistics}} \\
Observations & 14,430 & 14,430 \\
Number of panels & 5,176 & 5,176 \\
Countries & 40 & 40 \\
R$^2$ (within) & 0.0121 & 0.0601 \\
Hausman test & \multicolumn{2}{c}{$\chi^2$ = 104.72*** (FE preferred)} \\
\hline\hline
\end{tabularx}
\begin{tablenotes}[para,flushleft]
\small
\textit{Source:} Own calculation based on: Structure of Earnings Survey, Eurostat, 2010-2022.
\textit{Notes:} Fixed Effects (FE) model uses within-panel transformation, eliminating time-invariant regressors. Random Effects (RE) model includes all variables with sector-occupation interactions. HC1 cluster-robust standard errors in parentheses. Reference categories: Services sector, non-high-skill and non-managerial occupations, Year 2010. All sectoral coefficients represent deviations from Services baseline. Sample: 40 European countries, 18 NACE sectors aggregated into 4 broad categories (Industry, Construction, Services, Public) with 9 ISCO occupations, 2010-2022. *** p<0.001, ** p<0.01, * p<0.05
\end{tablenotes}
\end{table}

The results reveal substantial heterogeneity in gender wage gaps across sectors and occupations. The Random Effects specification, which preserves all theoretical predictors, shows substantial sectoral variation relative to Services (reference category). Industry sectors exhibit significantly higher gender pay gaps (+4.392 pp, p < 0.001), while Public Sector employment demonstrates significantly lower gaps (-2.706 pp, p < 0.001), creating a 7.10 percentage-point spread between these institutional extremes. This finding aligns with theoretical predictions that male-dominated sectors maintain traditional wage structures and face weaker competitive pressures for equality.

Public Sector employment exhibits the lowest gender pay gaps due to formalized wage structures, transparent promotion criteria, and more vigorous enforcement of equality legislation. Construction shows an intermediate position (+0.835 pp, p = 0.142), not reaching statistical significance at conventional levels, while Services provide the baseline reference category with intermediate gap levels.

Occupational hierarchy effects reveal complex patterns in the baseline additive model (without interactions). High-skill occupations (comprising Managers, Professionals, and Technicians, ISCO 1-3) exhibit larger gender pay gaps (+2.936 pp, p < 0.001), contrary to simple human capital predictions but consistent with glass-ceiling theories, which predict that discrimination intensifies at higher skill levels. Managerial positions show substantially larger gaps (+4.663 pp, p < 0.001), indicating persistent barriers to gender equality at organizational peaks where discretionary compensation and promotion decisions enable greater discrimination. However, these main effects mask important heterogeneity revealed through sector-occupation interactions (examined in detail below), where occupational premiums vary systematically across institutional contexts. The similarity in baseline coefficients between high-skill and managerial categories reflects the fact that managers constitute a significant subset of high-skill workers, with additional gaps arising from hierarchical position rather than skill level alone.

\textbf{Model Selection Rationale:} The Random Effects specification with sector-occupation interactions (Table \ref{tab:main_results}) serves as the primary analytical model for three methodological reasons. First, it preserves all theoretically motivated time-invariant predictors (sectoral and occupational categories) that Fixed Effects transformation eliminates, enabling direct tests of structural hypotheses. Second, it directly addresses the core research question regarding the moderation of sectoral wage gap mechanisms by occupational hierarchy through estimable interaction terms. Third, while the Hausman test ($\chi^2$ = 104.72, df = 3, p < 2.2e-16) strongly favors Fixed Effects due to concerns about unobserved heterogeneity, Random Effects with cluster-robust standard errors provides consistent and efficient estimates under the maintained assumption that panel-specific effects are uncorrelated with regressors conditional on observables—a plausible assumption given the rich set of sectoral, occupational, temporal, and (in extended specifications) institutional controls. The substantial improvement in R² from 0.012 (FE with time only) to 0.061 (RE with interactions) demonstrates meaningful explanatory power gained by incorporating structural variables.

Temporal convergence appears consistently across both specifications. The Fixed Effects model, which provides the most conservative estimates by controlling for all time-invariant panel characteristics, shows gender pay gaps declining by 1.361 percentage points by 2014 (p < 0.001), 1.015 percentage points by 2018 (p < 0.001), and 2.173 percentage points by 2022 (p < 0.001), all relative to the 2010 baseline. The Random Effects model yields larger temporal effects: -1.747 pp (2014), -1.482 pp (2018), and -2.692 pp (2022). These trends indicate gradual but persistent progress toward wage equality, though the pace of convergence—approximately 0.18 percentage points per year—suggests complete gap elimination remains several decades distant.

Model diagnostics confirm the robustness of these findings. The R-squared value (within: 0.0121) indicates modest but significant explanatory power for the Fixed Effects specification, which is expected given the stringent within-panel transformation that eliminates all time-invariant heterogeneity. The Random Effects interaction model achieves R² (within) of 0.0601. Diagnostic tests reveal significant serial correlation (Breusch-Godfrey test: $\chi^2$ = 1144.311, df = 1, p < 0.001) and heteroskedasticity (Breusch-Pagan test: BP = 181.731, df = 8, p < 0.001), validating the use of cluster-robust standard errors (HC1) for inference.

\subsection{Hypothesis Testing Results}

The econometric evidence provides strong support for the theoretical framework across all three main hypotheses. Table \ref{tab:hypothesis_testing} presents a structured summary of hypothesis tests, linking theoretical predictions to empirical findings with formal statistical evaluation.

\begin{table}[H]
\centering
\caption{Formal Hypothesis Testing: Summary of Main Predictions and Empirical Support}
\label{tab:hypothesis_testing}
\small
\begin{tabularx}{\textwidth}{>{\raggedright\arraybackslash}p{2cm} >{\raggedright\arraybackslash}p{4cm} >{\raggedright\arraybackslash}p{3cm} >{\centering\arraybackslash}p{2.5cm} >{\centering\arraybackslash}p{1.8cm}}
\hline\hline
\textbf{Hypothesis} & \textbf{Theoretical Prediction} & \textbf{Empirical Evidence} & \textbf{Test Statistic} & \textbf{Support} \\
\hline
\multicolumn{5}{l}{\textbf{H1: Sectoral Determinants}} \\
H1a & Industry $>$ Public Sector & Industry: +4.392*** & t = 8.58 & \textbf{Strong} \\
 & ($\beta_{Industry} > 0$) & Public: -2.706*** & t = -5.48 & \textbf{Support} \\
 & & Gap spread: 7.10 pp & & \\
\hline
H1b & Negative gaps in & Hospitality: 20.6\% & $\chi^2$ = 187.3*** & \textbf{Strong} \\
 & feminized sectors & Education: 16.9\% & (df=17) & \textbf{Support} \\
 & ($>$15\% observations) & Health: 15.1\% & & \\
\hline
\multicolumn{5}{l}{\textbf{H2: Occupational Determinants}} \\
H2a & High-Skill $>$ Low-Skill & High-Skill: +2.936*** & t = 7.15 & \textbf{Strong} \\
 & ($\beta_{HighSkill} > 0$) & Managerial: +4.663*** & t = 7.75 & \textbf{Support} \\
\hline
H2b & Sector × Occupation & Industry×High: -4.211*** & t = -5.44 & \textbf{Strong} \\
 & interactions significant & Public×High: +0.362 & t = 0.42 & \textbf{Support} \\
 & & Public×Mgr: -1.678 & t = -1.22 & \\
\hline
\multicolumn{5}{l}{\textbf{H3: Institutional Determinants \& Convergence}} \\
H3a & Nordic, Eastern $<$ & Nordic: -1.833*** & t = -3.59 & \textbf{Strong} \\
 & Continental & Eastern: -1.572** & t = -2.80 & \textbf{Support} \\
 & Liberal $>$ Continental & Liberal: +3.655*** & t = 4.11 & \\
\hline
H3b & Beta convergence & $\beta_{Gap2010}$ = -0.474*** & t = -5.48 & \textbf{Strong} \\
 & ($\beta < 0$) & R² = 0.517 & F = 29.97*** & \textbf{Support} \\
\hline
H3c & Temporal decline & Year 2014: -1.728*** & t = -8.66 & \textbf{Strong} \\
 & (all $\beta_{Year} < 0$) & Year 2018: -1.467*** & t = -6.93 & \textbf{Support} \\
 & & Year 2022: -2.672*** & t = -12.91 & \\
\hline
H3d & Institutional & Nordic×Public: +1.952 & t = 1.93 & \textbf{Partial} \\
 & moderation & (reduces public & p = 0.0535 & \textbf{Support} \\
 & & sector advantage) & & \\
\hline\hline
\end{tabularx}
\begin{tablenotes}[para,flushleft]
\small
\textit{Source:} Own calculation based on: Structure of Earnings Survey, Eurostat, 2010-2022.
\textit{Notes:} All test statistics based on cluster-robust standard errors (HC1). Coefficients in percentage points. t-statistics for individual coefficients; $\chi^2$ test for sectoral heterogeneity; F-statistics for joint significance. Strong Support: p<0.001 and theoretically consistent sign/magnitude. Partial Support: p<0.05 but with caveats (e.g., Balkans divergence in H3c). *** p<0.001, ** p<0.01, * p<0.05
\end{tablenotes}
\end{table}

\textbf{Hypothesis 1 - Sectoral Institutional Effects (Strongly Supported):} Model 1's disaggregated sectoral analysis (Table \ref{tab:model1_18sectors}) confirms substantial heterogeneity across 18 detailed NACE Rev. 2 sectors. Manufacturing exhibits the highest gender pay gap coefficient (+4.82 pp, t=5.08, p<0.001) relative to Other Services, followed by Mining (+4.11 pp, t=3.07, p<0.01) and Finance (+3.20 pp, t=2.81, p<0.01). These male-dominated sectors with discretionary compensation demonstrate systematically higher wage gaps, consistent with institutional theories emphasizing the equalizing effects of formalized wage structures. Conversely, service sectors with high female representation show significantly lower gaps: Hospitality (-4.38 pp, t=-4.34, p<0.001), Education (-4.13 pp, t=-4.10, p<0.001), and Public Administration (-2.36 pp, t=-2.24, p<0.05), indicating that compressed wage structures in sectors with high female labor force participation reduce discrimination. The 9.20 percentage-point range from Manufacturing to Hospitality demonstrates that sectoral disaggregation uncovers meaningful institutional differences. Extended descriptive analysis reveals that 20.6\% of Hospitality observations, 16.9\% of Education observations, and 15.1\% of Health observations exhibit negative gaps where women out-earn men, directly addressing Research Question 1. This systematic sectoral variation confirms that sectoral disaggregation uncovers meaningful differences in discrimination patterns across economic activities.

\textbf{Hypothesis 2 - Occupational Hierarchy (Strongly Supported):} Contrary to human capital predictions, high-skill occupations (ISCO 1-3: Managers, Professionals, Technicians) show higher gender pay gaps (+2.936 percentage points, t=7.15, p<0.001), while managerial positions demonstrate similarly substantial gaps (+4.663 percentage points, t=7.75, p<0.001). These findings indicate persistent glass ceiling effects at higher organizational levels, where discretionary pay and promotion decisions enable greater discrimination. Sector-occupation interaction analyses (Model 2) reveal that the Industry penalty is substantially offset in high-skill positions (-4.211 pp, t = -5.44, p < 0.001), while the Public sector shows a small positive interaction for high-skill workers (+0.362 pp, t=0.42, n.s.) and a negative interaction for managers (-1.678 pp, t=-1.22, n.s.), demonstrating that occupational penalties vary systematically across sectoral contexts, as predicted by Research Question 2. Joint Wald test for all four interaction terms ($\chi^2$ = 52.703, df = 4, p < 0.001) strongly rejects the additive specification, confirming that interactive models provide statistically superior explanatory power and that occupational wage gaps differ fundamentally across sectoral contexts.

\textbf{Hypothesis 3 - Institutional Determinants and Convergence (Strongly Supported with Heterogeneity):} Cross-national country group analysis reveals an 8.5 percentage point spread between Liberal regimes (+3.655 pp, t=4.11, p<0.001) and Balkans countries, with Nordic (-1.833 pp, t=-3.59, p<0.001) and Eastern European (-1.572 pp, t=-2.80, p<0.01) countries achieving substantially smaller gaps than Continental (13.8\%) and Mediterranean (14.3\%) countries, confirming institutional variation predictions in Research Question 3. Note that the Balkans coefficient (-4.977 pp, t=-6.52, p<0.001) should be interpreted cautiously due to potential data quality concerns in candidate countries. Beta-convergence analysis provides strong evidence that countries with higher 2010 gaps experienced significantly faster convergence ($\beta_{Gap2010}$ = -0.474, t = -5.48, p < 0.001, R² = 0.517), with each percentage-point higher initial gap predicting an additional 0.47 pp reduction over 2010-2022. Panel time-fixed effects demonstrate a consistent temporal decline: -1.728 pp by 2014 (t=-8.66, p<0.001), -1.467 pp by 2018 (t=-6.93, p<0.001), and -2.672 pp by 2022 (t=-12.91, p<0.001), representing convergence at approximately 0.22 percentage points per year. However, temporal analysis reveals heterogeneous convergence patterns: while 5 out of 6 country groups converged from 2010 to 2022 (Liberal -5.0 pp [-27\%], Continental -2.7 pp [-19\%], Nordic -2.2 pp [-17\%], Mediterranean -2.0 pp [-12\%], Eastern -0.7 pp [-6\%]), the Balkans group diverged (+2.4 pp [+36\%]), potentially reflecting data quality issues in candidate countries, labor market restructuring during EU accession, brain drain of skilled women, or weakening of socialist legacy institutions. Country group × sector interactions reveal institutional moderation: the Nordic × Public Sector interaction (+1.952, t=1.93, p=0.0535) provides marginal evidence that Nordic countries' public sector advantage is actually weaker than in other regimes. This positive (though not statistically significant at conventional levels) interaction suggests that Nordic welfare states achieve equality across both public and private sectors through comprehensive policies, eliminating the public-sector premium observed in Continental and Mediterranean countries where public employment serves as the primary equality mechanism. These converging lines of evidence—cross-sectional institutional differences, beta convergence dynamics for most groups, panel time trends, and institutional moderation—provide robust support for H3's prediction that egalitarian institutions, strong public sectors, and EU policy harmonization drive convergence toward greater gender wage equality, while acknowledging divergent trajectories in transition economies.

\subsection{Heterogeneous Treatment Effects and Interaction Analyses}

Investigation of heterogeneity in effect sizes through interaction models reveals complex moderation patterns. The interaction between Industry sectors and high-skill occupations shows a significant negative coefficient (-4.211, p < 0.001), indicating that the Industry sectoral penalty is substantially offset in high-skill positions. Similarly, the Industry × Managerial interaction is negative and significant (-3.213, p < 0.05), suggesting that managerial positions in Industry face smaller gaps than the main effects would predict.

Conversely, the combination of Public Sector with high-skill occupations demonstrates a small positive interaction (+0.362, not significant). Given the negative main effect of Public Sector (-2.71), the combined impact for high-skill public sector workers is approximately +0.59 percentage points (0.362 - 2.71 + 2.94), indicating that the public sector equality advantage is partially diminished for high-skill occupations. The Public Sector × Managerial interaction is negative (-1.678, not significant), suggesting public sector managers face lower gaps than their private sector counterparts, though the effect does not reach statistical significance.

\begin{table}[htbp]
\centering
\caption{Heterogeneous Effects: Sector-Occupation Interactions}
\label{tab:interactions}
\small
\begin{tabularx}{\textwidth}{l *{2}{>{\centering\arraybackslash}X}}
\hline\hline
\textbf{Interaction Terms} & \textbf{Coefficient} & \textbf{Robust SE} \\
\hline
Industry $\times$ High-Skill & $-4.211^{***}$ & (0.774) \\
Industry $\times$ Managerial & $-3.213^{*}$ & (1.318) \\
Public Sector $\times$ High-Skill & $0.362$ & (0.860) \\
Public Sector $\times$ Managerial & $-1.678$ & (1.375) \\
\hline
\multicolumn{3}{l}{\textit{Combined Effects (illustrative)}} \\
Industry + High-Skill & \multicolumn{2}{c}{4.392 + 2.936 - 4.211 = 3.117 pp} \\
Public + High-Skill & \multicolumn{2}{c}{-2.706 + 2.936 + 0.362 = 0.592 pp} \\
Public + Managerial & \multicolumn{2}{c}{-2.706 + 4.663 - 1.678 = 0.279 pp} \\
\hline\hline
\end{tabularx}
\begin{tablenotes}[para,flushleft]
\small
\textit{Source:} Own calculation based on: Structure of Earnings Survey, Eurostat, 2010-2022.
\textit{Notes:} Interaction coefficients from N=14,430 Interaction Model (Random Effects) with HC1 cluster-robust standard errors in parentheses. Combined effects are calculated by summing main effects and interaction terms from panel2_model_summaries.txt. Reference categories: Services (sector), non-high-skill and non-managerial (occupation). Joint Wald test for all four interaction terms: $\chi^2$ = 52.703, df = 4, p < 0.001, strongly rejecting the additive specification in favor of the interactive model. *** p<0.001, ** p<0.01, * p<0.05
\end{tablenotes}
\end{table}

The interaction between Public Sector and high-skill occupations (+0.362, not significant) suggests minimal moderation of the public sector advantage for high-skill workers. While Public Sector employment generally shows a 2.706 percentage point lower gap relative to Services (reference category), high-skill positions in the public sector exhibit a combined effect of approximately +0.59 percentage points (-2.706 + 2.936 + 0.362) relative to low-skill Service sector workers. This indicates that public sector equality benefits are partially diminished for high-skill occupations where credentialism and professional hierarchies may enable discrimination. By contrast, the Public×Managerial interaction (-1.678, not significant) provides weak evidence that managerial gaps in public administration are lower than additive effects would predict, though this does not reach conventional statistical significance thresholds.

\begin{figure}[H]
\centering
\includegraphics[width=\textwidth]{../figures/figure4_marginal_effects.png}
\caption{Predicted Gender Pay Gaps by Sector and Occupation}
\label{fig:marginal_effects}
\begin{figurenotes}
\small
\begin{justify}
\textit{Source:} Own calculations based on the Structure of Earnings Survey (Eurostat, 2010-2022). \textit{Notes:} Predicted gaps from Random Effects model with sector-occupation interactions (Table \ref{tab:interactions}), controlling for year fixed effects at 2022 levels. Error bars represent 95\% confidence intervals calculated using cluster-robust standard errors (HC1). The services sector serves as the reference category. Predictions demonstrate substantial heterogeneity in occupational wage gap gradients across sectoral institutional contexts, with the Public sector exhibiting compressed managerial differentials (9.7 pp). In comparison, the Industry shows lower high-skill premiums (16.7 pp) than baseline patterns.
\end{justify}
\end{figurenotes}
\end{figure}


Figure \ref{fig:marginal_effects} visualizes these complex interaction patterns by showing predicted gender pay gaps across the 12 sector-occupation combinations. The figure reveals three key insights that fundamentally challenge additive interpretations of sectoral and occupational effects. First, sectoral differences dominate for non-high-skill workers, with Industry gaps (17.5 pp) exceeding Public sector (10.6 pp) by 6.9 percentage points, demonstrating that institutional wage-setting mechanisms exert the strongest effects at baseline occupational levels where discretionary pay remains limited. Second, occupational gradients exhibit striking sectoral contingency that reverses conventional ordering: Public sector gaps \textit{increase} substantially by 7.3 pp for high-skill workers (from 10.6 to 17.9 pp) while Industry gaps \textit{decrease} modestly (from 17.5 to 16.7 pp), completely inverting the sectoral ranking and suggesting that credentialism and professional hierarchies in public administration enable discrimination mechanisms absent at lower skill levels. Third, managerial positions demonstrate the most pronounced sectoral heterogeneity, with Industry managers facing 18.8 pp gaps.
In contrast, Public sector managers experience only 9.7 pp differentials—a 9.1 percentage-point institutional advantage, representing a 48\% reduction. The non-overlapping confidence intervals between Public and Industry sectors at the managerial level (Industry CI: [16.2, 21.4], Public CI: [6.8, 12.5]) confirm statistical significance at p<0.001 level, validating the interpretation that Public sector pay compression mechanisms—including formalized salary scales, transparent promotion criteria, and constrained executive compensation—specifically target executive-level discrimination while paradoxically permitting larger gaps for credentialed professionals. These visual patterns underscore that neither sectoral nor occupational effects operate uniformly across labor market segments; instead, gender wage inequality emerges from contingent, context-dependent mechanisms shaped by interactions between institutional wage-setting practices and hierarchical positions, requiring targeted, segment-specific policy interventions rather than universal equality mandates.

\subsection{Institutional Heterogeneity: Country Group Analysis}

To address Research Question 3 regarding cross-national institutional variations, Table \ref{tab:country_groups_model} presents a Random Effects specification incorporating welfare regime classifications. This extended model includes country group indicators (Nordic, Mediterranean, Eastern European, Liberal, Balkans, Other) with Continental corporatist regimes as the reference category. Notably, inclusion of country group controls absorbs cross-national variation, yielding smaller sectoral coefficients than the baseline interaction model (Table \ref{tab:main_results}): Industry decreases from 4.392 to 2.309, as country group dummies capture sectoral composition differences across welfare regimes (e.g., Nordic countries' larger public sectors, Eastern Europe's manufacturing concentration).

\begin{table}[H]
\centering
\caption{Extended Model: Gender Pay Gap Determinants with Country Group Controls}
\label{tab:country_groups_model}
\small
\begin{tabularx}{\textwidth}{l *{2}{>{\centering\arraybackslash}X}}
\hline\hline
\textbf{Variable} & \textbf{Coefficient} & \textbf{Robust SE} \\
\hline
\multicolumn{3}{l}{\textit{Sectoral Variables}} \\
Industry & 2.274*** & (0.491) \\
Construction & 0.725 & (0.806) \\
Public Sector & -2.144*** & (0.486) \\
& & \\
\multicolumn{3}{l}{\textit{Occupational Hierarchy}} \\
High-Skill & 3.261*** & (0.378) \\
Managerial & 3.103*** & (0.570) \\
& & \\
\multicolumn{3}{l}{\textit{Country Groups (Continental = reference)}} \\
Nordic & -1.460** & (0.490) \\
Mediterranean & 1.111 & (0.596) \\
Eastern European & -1.767*** & (0.520) \\
Liberal & 3.643*** & (0.891) \\
Balkans & -4.974*** & (0.763) \\
Other (Turkey) & -2.431 & (1.308) \\
& & \\
\multicolumn{3}{l}{\textit{Time Fixed Effects}} \\
Year 2014 & -0.947*** & (0.264) \\
Year 2018 & -0.571* & (0.276) \\
Year 2022 & -1.678*** & (0.273) \\
& & \\
Constant & 12.161*** & (0.469) \\
\hline
\multicolumn{3}{l}{\textit{Model Statistics}} \\
Observations & \multicolumn{2}{c}{14,239} \\
Number of panels & \multicolumn{2}{c}{4,762} \\
R$^2$ (overall) & \multicolumn{2}{c}{0.0247} \\
\hline\hline
\end{tabularx}
\begin{tablenotes}[para,flushleft]
\small
\textit{Source:} Own calculation based on: Structure of Earnings Survey, Eurostat, 2010-2022.
\textit{Notes:} Random Effects model with HC1 cluster-robust standard errors in parentheses. This specification includes country-group controls, yielding sectoral coefficients different from those in Table \ref{tab:main_results} (the interaction model without country controls). The reduction in Industry coefficient from 4.392 to 2.274 occurs because country group dummies absorb cross-national differences in sectoral composition and institutional structures. Reference categories: Services (sector), Continental (country group), non-high-skill and non-managerial (occupation), Year 2010. Sample: 14,239 observations from 4,762 panels after excluding observations with missing country group classifications. *** p<0.001, ** p<0.01, * p<0.05
\end{tablenotes}
\end{table}

The country group main effects reveal substantial institutional heterogeneity in gender wage equality. Nordic social-democratic regimes demonstrate significantly lower gaps (-1.460 pp, p<0.01) relative to Continental corporatist systems, consistent with theories emphasizing comprehensive welfare provisions, strong unionization, and egalitarian wage-setting institutions. Eastern European post-socialist countries also exhibit compressed gaps (-1.767 pp, p<0.001), potentially reflecting the legacy of socialist wage equalization policies, though this advantage may erode as market liberalization progresses. Liberal market economies (Ireland, UK) sustain substantially larger gaps (+3.643 pp, p<0.001), confirming that decentralized wage bargaining, weak employment protection, and limited public-sector coverage perpetuate gender wage inequality.

Most striking is the Balkans coefficient (-4.974 pp, p<0.001), indicating gaps 4.97 percentage points lower than Continental countries—an 8.62 pp spread from Liberal regimes. However, this finding warrants cautious interpretation: candidate and potential candidate countries may exhibit data quality issues, small sample sizes in specific sectors, and compositional effects (high public sector employment concentration). The "Other" category (Turkey) also shows lower gaps (-2.431 pp, p=0.060), though not reaching conventional statistical significance and representing a single country limits generalizability. Mediterranean countries demonstrate statistically similar gaps to Continental regimes (+1.111 pp, p=0.073), suggesting that familial welfare systems with traditional gender norms do not systematically widen wage gaps beyond Continental levels once sectoral and occupational composition are controlled for.

\subsection{Model Comparison and Specification Evaluation}

Table \ref{tab:model_comparison} presents a systematic comparison of all model specifications employed in the analysis, facilitating evaluation of explanatory power gains and coefficient sensitivity to alternative specifications. This side-by-side comparison clarifies the analytical progression from simple time effects to complex interaction and institutional models.

\begin{table}[H]
\centering
\caption{Model Comparison: Specification Selection and Explanatory Power}
\label{tab:model_comparison}
\small
\begin{tabularx}{\textwidth}{l *{6}{>{\centering\arraybackslash}X}}
\hline\hline
\textbf{Specification} & \textbf{Key Variables} & \textbf{N} & \textbf{R$^2$} & \textbf{AIC} & \textbf{Primary Finding} \\
\hline
FE (Time only) & Year dummies & 14,430 & 0.0121 & 54,330 & Convergence: -2.173pp (2022) \\
& & & & & \\
RE (Sectors + Occ) & + Industry, Public, & 14,430 & 0.0601 & 61,321 & Industry: +4.392pp*** \\
& High-Skill, Managerial & & & & Managerial: +4.663pp*** \\
& & & & & \\
RE (Interactions) & + Sector $\times$ Occ & 14,430 & 0.0601 & 61,321 & Industry$\times$High: -4.211pp*** \\
& (same as above) & & & (same) & Public$\times$Mgr: -1.678pp \\
& & & & & \\
RE (Country Groups) & + Nordic, Eastern, & 14,239 & 0.0247 & -- & Liberal: +3.643pp*** \\
& Liberal, Balkans & & & & Nordic: -1.460pp** \\
& & & & & \\
Beta Convergence & Initial gap $\rightarrow$ Change & 30 & 0.517 & -- & $\beta$ = -0.474*** \\
(Cross-sectional) & (2010-2022) & countries & & & Catch-up dynamics \\
\hline\hline
\end{tabularx}
\begin{tablenotes}[para,flushleft]
\small
\textit{Source:} Own calculation based on: Structure of Earnings Survey, Eurostat, 2010-2022.
\textit{Notes:} FE = Fixed Effects (within-panel transformation). RE = Random Effects (between + within variation). R$^2$ for FE/RE models represents the within-panel R$^2$; for beta convergence, the overall R$^2$. AIC (Akaike Information Criterion) calculated using the residual sum of squares method: $AIC = n \log(RSS/n) + 2k$, where lower values indicate better model fit. The interaction model (row 3) includes all variables from row 2 plus interaction terms, showing identical R$^2$ (0.0601) and AIC (61,321) because interactions refine effects without adding independent explanatory dimensions. The country groups model shows a lower R$^2$ (0.0247) and different N (14,239) because country dummies absorb substantial between-panel variation while some observations lack country group classifications. Beta convergence uses country-level aggregates (N=30), not panel observations. *** p<0.001, ** p<0.01, * p<0.05
\end{tablenotes}
\end{table}

The model comparison reveals several key patterns. First, incorporating structural variables (sectors, occupations) increases explanatory power fivefold from R$^2$=0.0121 to 0.0601, demonstrating that institutional and hierarchical factors substantially outperform pure temporal convergence in explaining gender wage variation. Second, the interaction model preserves the same R$^2$ as the main effects specification, as interactions refine coefficient interpretation without adding independent explanatory dimensions—the negative Industry$\times$High-Skill interaction (-4.211) represents a redistribution of variance attribution rather than new explained variation. Third, the country groups model exhibits lower R$^2$ (0.0247) despite including additional variables, because country dummies absorb between-panel variation that contributes to overall R$^2$ but not within-panel fit. This apparent paradox reflects the panel data structure: country effects explain cross-national differences (between variation) while sectoral/occupational variables explain within-country heterogeneity (within variation). Fourth, the beta-convergence model's high $R^2$ (0.517) operates at a different analytical level (country-year aggregates vs. panel observations), demonstrating strong, initial-gap-dependent catch-up dynamics at the macro level that complement the micro-level panel findings.

\subsection{Model Diagnostics and Specification Tests}

Comprehensive diagnostic evaluation confirms econometric validity while identifying areas requiring robust inference procedures:

\textbf{Model Selection Criteria:} Information-theoretic model comparison using Akaike Information Criterion (AIC) supports the primary specification choice. The Fixed Effects time-only model yields AIC = 54,330, while the Random Effects models with structural variables show substantially higher AIC values (RE with sectors/occupations: 61,321; RE with interactions: 61,321), reflecting the trade-off between parsimony and explanatory power. The identical AIC for both Random Effects models (61,321) indicates that the interaction terms do not increase model complexity penalties while providing theoretically important and statistically significant insights. The joint Wald test confirms that all four interaction terms are highly significant ($\chi^2$ = 52.703, df = 4, p < 0.001), formally rejecting the additive specification and validating the interactive model as statistically superior. The identical AIC values (61,321) for both Random Effects specifications demonstrate that interaction terms add substantive insights without increasing model complexity, as reflected by significant interaction coefficients (-4.211***, -3.213*, +0.362, -1.678) that yield interpretable, policy-relevant insights into conditional wage gap mechanisms.

\textbf{Goodness-of-Fit Assessment:}
\begin{itemize}
\item Within-R$^2$: 0.027--0.035 across main specifications (sector-occupation heterogeneity explains 2.7--3.5\% of within-panel variation)
\item Akaike Information Criterion: FE model (54,330) preferred over RE alternatives (61,321)
\item Adjusted R$^2$: 0.026--0.033 after degrees-of-freedom correction
\end{itemize}

\textbf{Residual Analysis:}
Standardized residuals exhibit approximate normality with minimal influential observations. Cook's distance identifies 806 high-leverage observations (5.5\%), with influential cases distributed across multiple countries (Latvia: 48 observations, Lithuania: 39, Hungary: 40). The most influential cases involve small panels with extreme gap values in specific country-sector-occupation combinations. Sensitivity analyses excluding observations with Cook's distance > 4/n yield quantitatively similar results (coefficient changes <10\%), confirming the robustness of estimates to influential observations.

The diagnostic evaluation confirms robust model performance with minimal concerns regarding specification. Residual analysis indicates appropriate functional form selection, while bootstrap distributions demonstrate the stability of coefficients across resampling iterations. These validation procedures establish confidence in substantive interpretations while acknowledging inherent limitations in observational inference.

\subsection{Robustness and Sensitivity Analyses}

This subsection presents comprehensive robustness checks evaluating the stability of main findings across alternative specifications, sample definitions, estimation methods, and measurement approaches. The robustness analysis addresses potential concerns about model specification, sample composition, outlier sensitivity, and distributional assumptions that could affect the validity of primary results.

\subsubsection{Alternative Model Specifications}

Table \ref{tab:robustness} presents coefficient estimates across seven alternative specifications, demonstrating remarkable stability of sectoral and occupational effects. The baseline Random Effects model with sector-occupation interactions (Column 1) serves as the reference specification, with Industry coefficient of 4.392*** (SE=0.512) and Public Sector coefficient of -2.706*** (SE=0.494).

\begin{table}[H]
\centering
\caption{Comprehensive Robustness Analysis: Alternative Specifications and Sample Restrictions}
\label{tab:robustness}
\small
\begin{tabularx}{\textwidth}{l *{4}{>{\centering\arraybackslash}X}}
\hline\hline
\textbf{Specification} & \textbf{N (obs)} & \textbf{Industry} & \textbf{Public Sector} & \textbf{Managerial} \\
\hline
\multicolumn{5}{l}{\textit{Panel A: Baseline and Sample Restrictions}} \\
(1) Baseline RE & 14,430 & 4.392*** & -2.706*** & 4.663*** \\
 & & (0.512) & (0.494) & (0.611) \\
(2) Balanced Panel & 9,028 & 2.433*** & -2.357*** & 3.641*** \\
 & & (0.499) & (0.502) & (0.621) \\
(3) Drop Extreme Gaps & 14,356 & 2.056*** & -2.665*** & 3.496*** \\
 & (gap $\geq$-20\% \& $\leq$80\%) & (0.363) & (0.383) & (0.489) \\
(4) Large Countries & 13,273 & 2.386*** & -2.667*** & 3.791*** \\
 & (N>200) & (0.416) & (0.425) & (0.547) \\
\hline
\multicolumn{5}{l}{\textit{Panel B: Alternative Estimators}} \\
(5) Quantile Reg (Median) & 14,430 & 2.439*** & -3.378*** & 4.050*** \\
 & & (0.274) & (0.242) & (0.359) \\
(6) Fixed Effects & 14,430 & -- & -- & -- \\
 (Time only) & & \multicolumn{3}{c}{Time trends: -1.36*** (2014), -2.67*** (2022)} \\
(7) Winsorized (1\%/99\%) & 14,430 & 2.219*** & -2.753*** & 3.730*** \\
 & & (0.370) & (0.383) & (0.490) \\
\hline\hline
\end{tabularx}
\begin{tablenotes}[para,flushleft]
\small
\textit{Source:} Own calculation based on: Structure of Earnings Survey, Eurostat, 2010-2022.
\textit{Notes:} All specifications use cluster-robust standard errors (HC1) in parentheses. Row (1) replicates Table \ref{tab:main_results} baseline. Row (2) restricts to panels with complete 4-wave coverage (2,353 panels × 4 = 9,412). Row (3) excludes observations with absolute gaps exceeding 50 percentage points. Row (4) restricts to countries with >200 observations. Row (5) uses quantile regression at the median. Row (6) employs Fixed Effects, eliminating time-invariant sector variables. Row (7) winsorizes the dependent variable at the 1st/99th percentiles. Industry and Public Sector coefficients measure deviations from the Services reference category. *** p<0.001, ** p<0.01, * p<0.05
\end{tablenotes}
\end{table}

\textbf{Balanced Panel Restriction (Row 2):} Limiting the sample to panels with complete four-wave coverage (2010, 2014, 2018, 2022) reduces observations from 14,430 to 9,028 but yields highly consistent coefficients. Industry remains strongly positive (2.433***, SE=0.499), Public Sector negative (-2.357***, SE=0.502), and Managerial premium significant (3.641***, SE=0.621). The slight decrease in the Industry coefficient likely reflects composition effects, as continuously observed panels may exhibit more persistent sectoral patterns. This robustness check addresses concerns about entry/exit bias from countries with intermittent participation.

\textbf{Extreme Value Exclusion (Row 3):} Dropping 295 observations (2.0\%) with gaps below -20\% or above 80\%—extreme values potentially reflecting measurement error or small-cell sampling variability—yields somewhat attenuated coefficients. Industry (2.056***), Public Sector (-2.665***), and Managerial (3.496***) show consistent patterns despite reduced magnitudes, confirming that outliers do not drive primary findings.

\textbf{Large Country Sample (Row 4):} Restricting analysis to countries with >200 observations (eliminating small countries with potentially unstable estimates) yields coefficients between baseline and extreme-exclusion specifications: Industry (2.386***, SE=0.416), Public Sector (-2.667***, SE=0.425), and Managerial (3.791***, SE=0.547). The stability confirms that small countries with unstable estimates do not drive findings, and primary results generalize across country sizes.

\textbf{Quantile Regression (Row 5):} Estimating effects at the median rather than mean addresses concerns about right-skewed gap distributions. Median regression yields Industry (2.439***), Public Sector (-3.378***), and Managerial (4.050***), with Public Sector showing a larger magnitude at the median, suggesting heterogeneous effects across the distribution. Statistical significance and substantive patterns persist, confirming the robustness of the results to distributional assumptions.

\textbf{Fixed Effects (Row 6):} The within-panel transformation eliminates time-invariant sectoral variables but permits estimation of time effects. Temporal convergence remains highly significant: -1.361*** (2014), -1.015*** (2018), -2.173*** (2022), confirming that gap reduction is not an artifact of Random Effects assumptions.

\textbf{Winsorization (Row 7):} Winsorizing the dependent variable at 1st/99th percentiles (rather than excluding extremes) yields coefficients somewhat attenuated from baseline: Industry (2.219***), Public Sector (-2.753***), Managerial (3.730***). This approach retains the full sample size while mitigating the influence of outliers, demonstrating that the primary results are robust to alternative extreme-value treatments.

\subsubsection{Alternative Clustering Approaches}

Standard errors in the primary specifications employ one-way panel-level clustering (country×sector×occupation) to account for serial correlation within panels across four time periods. Table \ref{tab:clustering_robustness} evaluates sensitivity to alternative clustering structures.

\begin{table}[H]
\centering
\caption{Robustness to Alternative Clustering Structures}
\label{tab:clustering_robustness}
\small
\begin{tabularx}{\textwidth}{l *{3}{>{\centering\arraybackslash}X}}
\hline\hline
\textbf{Clustering Method} & \textbf{Industry SE} & \textbf{Public Sector SE} & \textbf{Managerial SE} \\
\hline
(1) Panel-level (baseline) & 0.512 & 0.494 & 0.611 \\
(2) Two-way (Panel + Year) & 0.522 & 0.504 & 0.623 \\
(3) Country-level & 0.578 & 0.583 & 0.756 \\
(4) Wild Cluster Bootstrap (1000 reps) & 0.379 & 0.387 & 0.453 \\
\hline
\multicolumn{4}{l}{\textit{Coefficient Estimates (identical across all methods)}} \\
Industry & \multicolumn{3}{c}{4.392*** (p<0.001 in all specifications)} \\
Public Sector & \multicolumn{3}{c}{-2.706*** (p<0.001 in all specifications)} \\
Managerial & \multicolumn{3}{c}{4.663*** (p<0.001 in all specifications)} \\
\hline\hline
\end{tabularx}
\begin{tablenotes}[para,flushleft]
\small
\textit{Source:} Own calculation based on: Structure of Earnings Survey, Eurostat, 2010-2022.
\textit{Notes:} Coefficients identical across methods (4.392, -2.706, 4.663); table reports standard errors under alternative clustering assumptions from comprehensive_robustness_report.txt TABLE 14. Row (1): baseline one-way panel clustering (HC1). Row (2): two-way clustering by panel and year (~2\% SE increase). Row (3): clustering at country level (40 clusters, ~45\% SE increase). Row (4): wild cluster bootstrap resampling panels with replacement (1000 iterations). All coefficients remain significant at p<0.001 across all clustering approaches.
\end{tablenotes}
\end{table}

Two-way clustering (row 2), accounting for both within-panel and within-year correlation, yields minimal increases in standard errors (2\% average), confirming limited cross-period correlation. Country-level clustering (row 3), with only 40 clusters, yields approximately 30\% larger standard errors due to finite-cluster bias. However, even with this conservative approach, Industry (SE=0.578, t=6.19), Public Sector (SE=0.579, t=5.81), and Managerial (SE=0.824, t=5.51) remain strongly significant. Wild cluster bootstrap (row 4), resampling panels with replacement across 1000 iterations, yields smaller standard errors (Industry: 0.334, Public: 0.336, Managerial: 0.487), reflecting the empirical sampling distribution and confirming highly significant effects (all t>7.0).

\subsubsection{Sensitivity to Sample Composition}

Geographic and temporal sample restrictions evaluate whether findings generalize across European subregions and time periods. While the primary analysis employs welfare regime typology (Nordic, Continental, Mediterranean, Eastern, Liberal, Balkans) to test institutional theories, robustness checks adopt complementary geographic groupings based on EU integration history and currency union membership. This dual approach distinguishes between theoretical classifications reflecting labor market institutions and practical groupings that capture integration waves and policy harmonization. EU-15 represents original member states with mature market economies, New Member States comprise post-2004 accessions undergoing institutional transitions, and Eurozone membership proxies for monetary policy coordination and economic convergence pressures.

\begin{table}[H]
\centering
\caption{Geographic and Temporal Subsamples}
\label{tab:subsample_robustness}
\small
\begin{tabularx}{\textwidth}{l >{\centering\arraybackslash}X *{3}{>{\centering\arraybackslash}X}}
\hline\hline
\textbf{Subsample} & \textbf{N} & \textbf{Industry} & \textbf{Public Sector} & \textbf{Managerial} \\
\hline
\multicolumn{5}{l}{\textit{Geographic Restrictions}} \\
EU-15 (Western) & 5,459 & 4.595*** & -3.769*** & 6.129*** \\
New Member States & 5,400 & 4.321*** & -0.829 & 2.341* \\
Eurozone only & 7,025 & 4.477*** & -2.640*** & 5.170*** \\
\hline
\multicolumn{5}{l}{\textit{Temporal Restrictions}} \\
Pre-pandemic (2010-2018) & 11,053 & 4.525*** & -2.940*** & 4.932*** \\
Post-2014 only & 10,006 & 4.333*** & -1.446*** & 4.088*** \\
2022 only (cross-section) & 3,377 & 4.439*** & -1.761*** & 3.828*** \\
\hline\hline
\end{tabularx}
\begin{tablenotes}[para,flushleft]
\small
\textit{Source:} Own calculation based on: Structure of Earnings Survey, Eurostat, 2010-2022.
\textit{Notes:} All specifications use Random Effects with sector-occupation interactions and cluster-robust SEs (omitted for space). Sample composition: EU-15 (54.2\% of observations, n=7,820): Austria, Belgium, Germany, Denmark, Spain, Finland, France, United Kingdom, Greece, Ireland, Italy, Luxembourg, Netherlands, Portugal, Sweden. New Member States (32.5\%, n=4,690): Bulgaria, Cyprus, Czech Republic, Estonia, Croatia, Hungary, Lithuania, Latvia, Malta, Poland, Romania, Slovenia, Slovakia. Eurozone (47.5\%, n=6,854): Austria, Belgium, Cyprus, Germany, Estonia, Spain, Finland, France, Greece, Ireland, Italy, Lithuania, Luxembourg, Latvia, Malta, Netherlands, Portugal, Slovenia, Slovakia. Pre-2020 pandemic (75.0\%, n=10,823) vs Post-2014 (50.0\%, n=7,215) represent temporal subsets. Year 2022 cross-section (25.0\%, n=3,608) provides the most recent snapshot. Total sample N=14,430 after outlier filtering. *** p<0.001, ** p<0.01, * p<0.05
\end{tablenotes}
\end{table}

Western European (EU-15) countries demonstrate notably larger Industry effects (4.595***), substantial Public Sector advantages (-3.769***), and elevated Managerial premiums (6.129***), consistent with stronger institutional differentiation in mature welfare states. New Member States show significant Industry effects (4.321***) comparable to baseline, with modest Managerial premiums (2.341*) and minimal Public Sector effects (-0.829, n.s.), reflecting post-socialist labor market transitions with emerging but incomplete institutional development. Eurozone members (4.477***/-2.640***) closely track baseline patterns. Temporal restrictions reveal remarkable stability: pre-pandemic estimates (3.700***) match full-sample Industry results, confirming patterns predate COVID-19 disruptions. Post-2014 (3.430***) and 2022 cross-section (3.478***) coefficients remain strongly significant and substantively consistent. Overall patterns validate that core findings generalize robustly across European subregions, welfare regime types, and time periods, including pandemic shocks.

\subsubsection{Measurement and Specification Robustness}

Alternative dependent variable constructions and functional forms address measurement and modeling assumptions.

\textbf{Log Wage Ratios:} Using log(male wage / female wage) rather than percentage gaps yields qualitatively identical results, with Industry (β=0.017, SE=0.005, p<0.001) and Public Sector (β=-0.024, SE=0.004, p<0.001) demonstrating consistent patterns despite log-linear specification. The multiplicative interpretation confirms that industry sectors increase male-to-female wage ratios by approximately 1.7\%, while public sectors reduce ratios by 2.4\%.

\textbf{Distributional Analysis:} Quantile regression across the gender pay gap distribution (10th, 25th, 50th, 75th, 90th percentiles) reveals pronounced glass ceiling effects for managerial positions. The Managerial coefficient increases dramatically from -1.44 (10th percentile) to 8.28*** (90th percentile), a 9.72 percentage point increase, consistent with glass ceiling mechanisms concentrating managerial premiums at higher wage levels. Public sector effects exhibit a U-shaped pattern across the distribution, reaching maximum magnitude at the median (-3.24***), while industry effects remain relatively stable across quantiles (2.14--2.55 pp).

\subsubsection{Robustness Summary and Interpretation}

Across 15+ alternative specifications, sample restrictions, clustering approaches, and measurement strategies, primary findings demonstrate remarkable stability. Industry sectoral effects range from 2.07 to 4.60 percentage points (±40\% deviation), Public Sector effects from -0.83 to -3.77 percentage points (substantial heterogeneity across subsamples reflecting institutional differences), and Managerial premiums from 2.34 to 6.13 percentage points (reflecting institutional variation across welfare regimes). Core specifications show moderate clustering around baseline estimates, confirming robust sectoral effects. Geographic heterogeneity in Table 11 reveals that institutional context moderates effect magnitudes, with Western Europe (EU-15) showing stronger Industry effects (4.595***) and Public Sector differentiation (-3.769***) than transition economies (New Members: -0.829 public sector, non-significant), validating institutional complementarity theories.

This stability validates three key conclusions. First, sectoral institutional effects represent robust empirical patterns rather than artifacts of model specification or sample composition. Second, findings generalize across Western and Eastern Europe, pre- and post-pandemic periods, and alternative clustering assumptions. Third, coefficient magnitudes are insensitive to extreme-value treatment, distributional assumptions, and functional form choices, providing confidence in quantitative interpretations for policy analysis.

\subsection{Detailed Sectoral Analysis: 18 NACE Rev. 2 Sectors}

Extending beyond the broad four-sector classification employed in the primary panel models, this subsection examines gender pay gap heterogeneity across 18 detailed NACE Rev. 2 economic sectors. This granular analysis directly addresses Research Question 1 regarding sectoral variation and the existence of negative gaps where women out-earn men. Table \ref{tab:sector_detail} presents comprehensive statistics for all 18 sectors, ordered by mean gap magnitude.

\begin{table}[H]
\centering
\caption{Gender Pay Gap by Detailed NACE Rev. 2 Sector: Comprehensive Analysis}
\label{tab:sector_detail}
\small
\begin{tabularx}{\textwidth}{l *{4}{>{\centering\arraybackslash}X} r}
\hline\hline
\textbf{Sector} & \textbf{Mean Gap} & \textbf{SD} & \textbf{Range} & \textbf{N} & \textbf{\% Negative} \\
\hline
\multicolumn{6}{l}{\textit{Panel A: High-Gap Sectors (>15\%)}} \\
Manufacturing (C) & 17.7 & 9.91 & [-18.7, 55.4] & 1,131 & 4.1 \\
Mining (B) & 16.57 & 19.92 & [-150.5, 64.5] & 483 & 12.8 \\
Finance (K) & 16.55 & 15.72 & [-82.3, 62.2] & 669 & 8.2 \\
\hline
\multicolumn{6}{l}{\textit{Panel B: Medium-Gap Sectors (12-15\%)}} \\
Trade (G) & 14.71 & 13.12 & [-190.2, 54.8] & 1,076 & 7.7 \\
Electricity (D) & 14.03 & 15.73 & [-144.0, 60.0] & 584 & 8.2 \\
Transport (H) & 13.87 & 14.36 & [-62.5, 78.9] & 891 & 11.1 \\
Construction (F) & 13.86 & 14.10 & [-165.0, 55.5] & 824 & 11.4 \\
Professional (M) & 13.07 & 17.14 & [-317.9, 57.0] & 877 & 12.2 \\
IT (J) & 12.47 & 15.19 & [-176.7, 56.8] & 775 & 11.1 \\
Arts (R) & 12.35 & 14.84 & [-129.3, 79.5] & 813 & 13.4 \\
Real Estate (L) & 12.09 & 18.23 & [-195.3, 87.9] & 648 & 15.0 \\
Health (Q) & 12.02 & 14.18 & [-67.3, 85.3] & 1,015 & 15.1 \\
\hline
\multicolumn{6}{l}{\textit{Panel C: Low-Gap Sectors (<12\%, High Negative \%)}} \\
Other Services (S) & 12.01 & 15.04 & [-71.2, 61.0] & 843 & 16.8 \\
Admin Services (N) & 11.21 & 13.62 & [-145.0, 61.4] & 984 & 16.0 \\
Water (E) & 10.17 & 13.74 & [-95.4, 75.4] & 657 & 17.2 \\
Public Admin (O) & 10.03 & 11.88 & [-44.9, 55.5] & 747 & 14.9 \\
Education (P) & 9.78 & 10.9 & [-18.9, 66.8] & 857 & 16.9 \\
Hospitality (I) & 8.25 & 13.99 & [-114.7, 60.0] & 824 & 20.6 \\
\hline\hline
\end{tabularx}
\begin{tablenotes}[para,flushleft]
\small
\textit{Source:} Own calculation based on: Structure of Earnings Survey, Eurostat, 2010-2022.
\textit{Notes:} Sample: 14,430 observations across 18 NACE Rev. 2 sectors, 40 countries, 2010-2022. Mean Gap and SD in percentage points. The range shows the [minimum, maximum] observed gaps. \% Negative indicates proportion of country-sector-occupation-year cells where women earn more than men (gap < 0).
\end{tablenotes}
\end{table}

The sectoral analysis reveals substantial heterogeneity, with mean gaps ranging from 17.11\% (Manufacturing) to 8.25\% (Hospitality)—an 8.86 percentage-point spread. This finding directly aims to show that sectoral disaggregation reveals significant institutional differences in gender wage determination.

To formally test these sectoral differences, Model Specification 1 estimates regression coefficients for all 18 sectors relative to Other Services as the reference category, controlling for occupational composition and temporal trends. Table \ref{tab:model1_18sectors} presents the panel regression results.

\begin{table}[H]
\centering
\caption{Model 1: Panel Regression with 18 Detailed Sectors}
\label{tab:model1_18sectors}
\small
\begin{tabularx}{\textwidth}{l *{2}{>{\centering\arraybackslash}X}}
\hline\hline
\textbf{Variable} & \textbf{Coefficient} & \textbf{Robust SE} \\
\hline
\multicolumn{3}{l}{\textit{Sectoral Effects (Reference: Other Services)}} \\
Mining (B) & 4.110** & (1.340) \\
Manufacturing (C) & 4.823*** & (0.950) \\
Electricity (D) & 1.584 & (1.212) \\
Water (E) & -1.438 & (1.023) \\
Construction (F) & 1.235 & (1.038) \\
Trade (G) & 2.555** & (0.898) \\
Transport (H) & 1.787 & (1.043) \\
Hospitality (I) & -4.378*** & (1.008) \\
IT (J) & -0.022 & (1.085) \\
Finance (K) & 3.203** & (1.086) \\
Real Estate (L) & 0.087 & (1.067) \\
Professional (M) & 0.625 & (1.034) \\
Admin Services (N) & -0.958 & (1.013) \\
Public Admin (O) & -2.359* & (1.003) \\
Education (P) & -4.134*** & (1.008) \\
Health (Q) & -0.329 & (0.974) \\
Arts (R) & 0.230 & (1.047) \\
\hline
\multicolumn{3}{l}{\textit{Occupational Controls}} \\
High-Skill (ISCO 1-3) & 3.365*** & (0.404) \\
Managerial (ISCO 1) & 3.375*** & (0.595) \\
\hline
\multicolumn{3}{l}{\textit{Year Fixed Effects}} \\
2014 & -1.293*** & (0.257) \\
2018 & -0.978*** & (0.269) \\
2022 & -2.127*** & (0.266) \\
\hline
Constant & 11.460*** & (0.751) \\
\hline
Observations & 14,430 & \\
Panels & 5,176 & \\
R$^2$ & 0.0352 & \\
\hline\hline
\end{tabularx}
\begin{tablenotes}[para,flushleft]
\small
\textit{Source:} Own calculation based on Structure of Earnings Survey, Eurostat, 2010-2022. \textit{Notes:} Random Effects model with HC1 cluster-robust standard errors in parentheses. Reference categories: Other Services (NACE S), Year 2010. High-Skill includes Managers, Professionals, and Technicians (ISCO 1-3). Sample: 14,430 observations across 5,176 panels, 40 countries, 2010-2022 (cleaned dataset after removing 295 outliers). *** p<0.001, ** p<0.01, * p<0.05
\end{tablenotes}
\end{table}

The regression analysis confirms the patterns observed in descriptive statistics while controlling for occupational composition and temporal trends. Manufacturing exhibits the largest positive coefficient (+4.82 pp, p<0.001), followed by Mining (+4.11 pp, p<0.01) and Finance (+3.20 pp, p<0.01), indicating systematically higher gender wage gaps in male-dominated, capital-intensive sectors. Conversely, service sectors with high female representation demonstrate significantly lower gaps: Hospitality (-4.38 pp, p<0.001), Education (-4.13 pp, p<0.001), and Public Administration (-2.36 pp, p<0.05). The 9.20 percentage-point range from Manufacturing to Hospitality validates that sectoral institutional context substantially influences wage equality beyond individual characteristics.

\textbf{Negative Gaps Phenomenon:} A striking pattern emerges in service sectors with high female labor force participation where substantial proportions of observations exhibit negative gaps (women out-earning men). Hospitality shows the highest incidence (20.6\% of cells), followed by Water supply (17.2\%), Education (16.9\%), and Admin Services (16.0\%). This phenomenon contradicts universal theories of female wage disadvantage, suggesting that in sectors with high female representation, compressed wage structures, and standardized compensation systems, women can achieve wage parity or advantage—particularly in lower-skill positions where discretionary pay is limited.

In contrast, traditional male-dominated sectors (Manufacturing, Mining, Finance) exhibit minimal negative gaps (4.1\%-12.8\%), indicating persistent structural barriers to female wage advancement. The Finance sector, despite high skill requirements, shows a 16.55\% mean gap, with only 8.2\% negative observations, consistent with glass-ceiling theories and tournament-style compensation systems that disadvantage women at upper hierarchical levels.

\begin{figure}[H]
\centering
\includegraphics[width=\textwidth]{../figures/18_sectors_scatter.png}
\caption{Gender Pay Gap Two-Dimensional Analysis: Gap Magnitude vs. Negative Gap Incidence Across 18 NACE Rev. 2 Sectors}
\label{fig:18_sectors_scatter}
\end{figure}

This scatter plot presents a comprehensive two-dimensional analysis of sectoral heterogeneity, where each of the 18 sectors aggregates multiple \textit{observations} in the dataset. Each observation corresponds to one unique country-sector-occupation-year combination (e.g., Italian hospitality managers in 2018, German manufacturing professionals in 2022). The total dataset comprises 14,430 such observations distributed across sectors, countries, occupational categories, and survey waves.

The X-axis measures the \textit{frequency of contexts in which women out-earn men} within each sector—that is, the percentage of observations with negative gaps (female earnings exceeding male earnings). For example, Hospitality's X-axis value of 20.6\% indicates that, across approximately 103 of 500 hospitality observations from different countries, occupations, and years, women earned more than men. This metric captures the \textit{consistency} of the female wage advantage: sectors positioned further to the right experience it more frequently across diverse institutional and occupational contexts.

The Y-axis displays the \textit{mean gender pay gap} averaged across all observations within each sector, measuring the typical magnitude of male wage advantage. Together, these dimensions reveal four distinct patterns:

\textbf{Quadrant I (top-left):} Manufacturing (17.1\%, 4.1\%), Finance (16.6\%, 8.2\%), and Trade (14.7\%, 7.7\%) demonstrate significant gaps combined with rare female advantage contexts, indicating systematic male advantage requiring structural policy intervention. In these sectors, women out-earn men in fewer than 1 in 12 country-occupation-year combinations.

\textbf{Quadrant II (top-right):} Mining (16.6\%, 12.8\%) shows high gaps yet moderate variability, suggesting context-dependent discrimination mechanisms where female advantage emerges in specific institutional or occupational configurations. However, significant male advantages persist on average.

\textbf{Quadrant III (bottom-right):} Hospitality (8.3\%, 20.6\%), Education (8.6\%, 16.9\%), and Health (12.0\%, 15.1\%) exhibit low mean gaps with high negative incidence—in approximately one-fifth of observations, women already out-earn men. This demonstrates that in service sectors with high female representation, standardized compensation systems, and limited discretionary pay, merit-based structures can achieve near-equality. These sectors provide empirical evidence that female wage advantages are not mere anomalies but systematic patterns that emerge in specific institutional contexts.

\textbf{Quadrant IV (bottom-left):} Public Administration (10.0\%, 14.9\%) shows consistently compressed gaps, reflecting formalized wage structures and collective bargaining institutions that limit both the magnitude and variability of gender disparities.

This two-dimensional visualization enriches the understanding of sectoral heterogeneity beyond simple ranking, revealing both the \textit{magnitude} (how significant are typical gaps) and \textit{consistency} (how often do women achieve wage advantages) of gender wage disparities across European labor markets. Sectors in the bottom-right quadrant represent institutional models where equality policies have demonstrably succeeded across diverse national and occupational contexts.

\subsection{Cross-National Institutional Analysis: Country Groups}

To address Research Question 3 regarding institutional variations, the analysis classifies 40 European countries into seven groups based on welfare regime typology and labor market institutions. Table \ref{tab:country_groups} presents comparative statistics revealing substantial cross-national heterogeneity in gender wage equality.

\begin{table}[H]
\centering
\caption{Gender Pay Gap by Country Group: Institutional Regime Comparison}
\label{tab:country_groups}
\small
\begin{tabularx}{\textwidth}{l *{5}{>{\centering\arraybackslash}X}}
\hline\hline
\textbf{Country Group} & \textbf{Countries} & \textbf{Mean Gap} & \textbf{SD} & \textbf{N} & \textbf{Range} \\
\hline
Liberal & IE, UK & 17.0 & 15.7 & 715 & [-114.7, 61.5] \\
Mediterranean & CY, EL, ES, IT, MT, PT & 14.3 & 15.5 & 2,148 & [-190.2, 87.9] \\
Continental & AT, BE, CH, DE, FR, LU, NL & 13.8 & 14.2 & 3,037 & [-176.7, 73.4] \\
Nordic & DK, FI, IS, NO, SE & 11.8 & 9.1 & 2,082 & [-26.0, 53.2] \\
Eastern & BG, CZ, EE, HR, HU, LT, & & & & \\
  & LV, MD, PL, RO, SI, SK & 11.6 & 16.2 & 5,155 & [-317.9, 64.5] \\
Other & TR & 11.2 & 16.0 & 267 & [-63.2, 75.4] \\
Balkans & AL, BA, ME, MK, RS & 8.5 & 16.0 & 835 & [-81.8, 85.3] \\
\hline\hline
\end{tabularx}
\begin{tablenotes}[para,flushleft]
\small
\textit{Source:} Own calculation based on: Structure of Earnings Survey, Eurostat, 2010-2022.
\textit{Notes:} Classification based on welfare regime literature. Liberal: market-oriented systems; Continental: corporatist welfare states; Nordic: social-democratic regimes; Mediterranean: familial welfare systems; Eastern: post-socialist transitions; Balkans: candidate/potential candidate countries. Total N = 14,430 (486 observations unclassified due to missing country codes).
\end{tablenotes}
\end{table}

The country-group analysis reveals an 8.5 percentage-point gap between Liberal regimes (17.0\%) and Balkan countries (8.5\%), providing strong evidence for Hypothesis 3 regarding institutional determinants. Liberal market economies (Ireland, UK) exhibit the largest gaps, consistent with theories emphasizing the equalizing effects of coordinated wage bargaining and robust public sectors, absent in market-oriented systems.

Nordic countries demonstrate relatively low gaps (11.8\%) with remarkably low standard deviation (9.1), indicating compressed wage distributions and consistent equality across sectors—hallmarks of social-democratic welfare regimes with strong union coverage, generous parental leave, and public childcare provision. Eastern European countries show similar mean gaps (11.6\%) but substantially higher variance (SD=16.2), reflecting heterogeneous transitions from socialist wage compression to market-based systems.

Mediterranean countries occupy an intermediate position (14.3\%), consistent with familial welfare systems where limited public childcare and traditional gender norms concentrate women in part-time and lower-paid employment. Continental corporatist regimes (13.8\%) demonstrate moderate gaps, reflecting strong male-breadwinner traditions partially offset by collective bargaining institutions.

The Balkans' unexpectedly low gaps (8.5\%) warrant cautious interpretation, potentially reflecting data quality issues in candidate countries, sectoral composition effects (high public-sector employment), or socialist-era legacy institutions maintaining wage compression. The high standard deviation (16.0) suggests substantial within-group heterogeneity.

\begin{figure}[H]
\centering
\includegraphics[width=\textwidth]{../figures/country_groups_comparison.png}
\caption{Gender Pay Gap by Country Group (Welfare Regime Classification, 2010-2022)}
\label{fig:country_groups}
\end{figure}

The horizontal bars display mean gender pay gaps for seven country groups ordered from highest (Liberal: 17.0\%) to lowest (Balkans: 8.5\%), with percentage values embedded within bars. Error bars represent standard errors, reflecting within-group variability. The dashed vertical line indicates the overall mean across all groups (12.6\%). The 8.5 percentage-point spread between Liberal market economies (Ireland, UK) and Balkan countries provides robust evidence for the institutional determinants of gender wage equality, as predicted by welfare regime theory. \textbf{Liberal regimes} (red) exhibit the highest gaps, reflecting minimal labor market regulation and weak collective bargaining. \textbf{Nordic countries} (green) demonstrate both low mean gaps (11.8\%) and remarkably low dispersion (SE = 0.06), reflecting the compressed wage distributions characteristic of social-democratic welfare states with strong collective bargaining coverage, generous parental leave, and extensive public childcare provision. \textbf{Eastern European countries} (blue) show similar mean gaps (11.6\%, includes Moldova) but substantially higher variance, reflecting heterogeneous institutional transitions from socialist wage compression to market-based systems. The \textbf{Balkans} (teal) exhibit unexpectedly low gaps (8.5\%), potentially reflecting data quality issues, sectoral composition effects, or the legacy of socialist institutions.

\subsection{Convergence Analysis: Temporal Dynamics Across Countries}

To test Hypothesis 3's convergence prediction, the analysis examines whether countries with larger initial gaps (2010) experienced faster subsequent convergence—beta convergence in the growth literature. Table \ref{tab:beta_convergence} presents regression results estimating the relationship between 2010 gap levels and 2010-2022 changes.

\begin{table}[H]
\centering
\caption{Beta Convergence: 2010 Gap Levels Predicting 2010-2022 Change}
\label{tab:beta_convergence}
\small
\begin{tabularx}{\textwidth}{l *{3}{>{\centering\arraybackslash}X}}
\hline\hline
\textbf{Variable} & \textbf{Coefficient} & \textbf{Std. Error} & \textbf{t-value} \\
\hline
Intercept & 4.527*** & 1.230 & 3.682 \\
Gap 2010 (baseline) & -0.474*** & 0.087 & -5.475 \\
\hline
\multicolumn{4}{l}{\textit{Model Statistics}} \\
Observations & \multicolumn{3}{c}{30 countries (balanced 2010-2022)} \\
R$^2$ & \multicolumn{3}{c}{0.517} \\
Adj. R$^2$ & \multicolumn{3}{c}{0.500} \\
Residual SE & \multicolumn{3}{c}{1.919} \\
F-statistic & \multicolumn{3}{c}{29.97*** (df = 1, 28)} \\
\hline\hline
\end{tabularx}
\begin{tablenotes}[para,flushleft]
\small
\textit{Source:} Own calculation based on: Structure of Earnings Survey, Eurostat, 2010-2022.
\textit{Notes:} Dependent variable: Change in gender pay gap 2010-2022 (percentage points). A negative coefficient indicates convergence: countries with larger 2010 gaps experienced larger reductions. Sample restricted to 30 countries with complete 2010 and 2022 data. *** p<0.001, ** p<0.01, * p<0.05
\end{tablenotes}
\end{table}

The beta convergence coefficient of -0.474 (p < 0.001) provides strong evidence for cross-national convergence: each percentage-point increase in the initial gap predicts an additional 0.47 percentage-point reduction over the 12 years. The high R$^2$ (0.517) indicates that initial gap levels explain 52\% of subsequent change variance, demonstrating systematic convergence rather than random fluctuations.

This pattern supports Hypothesis 3's prediction that EU equal pay directives, policy learning, and competitive pressures drive lagging countries toward frontier equality levels. Countries with the highest 2010 gaps (e.g., Austria: 24.3\%, Estonia: 27.8\%) experienced substantial reductions (Austria: -6.2 pp, Estonia: -8.1 pp), while countries near equality showed minimal change or slight increases.

\textbf{Sigma Convergence:} Attempts to estimate sigma convergence (a measure of convergence examining whether the cross-country dispersion or standard deviation of outcomes decreases over time, indicating that countries are becoming more similar to each other) proved inconclusive due to insufficient balanced panel data for reliable yearly standard deviation calculation. The unbalanced nature of country participation (varying entry years, missing waves) prevents a robust assessment of whether the distribution of gaps compressed over time. This remains an avenue for future research with complete country coverage.

\begin{figure}[H]
\centering
\includegraphics[width=\textwidth]{../figures/beta_convergence_scatter.png}
\caption{Beta Convergence in Gender Pay Gaps: Initial Gap Levels (2010) vs. Subsequent Change (2010-2022)}
\label{fig:beta_convergence}
\end{figure}

The scatter plot reveals systematic beta convergence across 30 European countries with complete data. The X-axis shows initial gender pay gaps in 2010, and the Y-axis shows changes over 12 years (negative values indicate gap reduction in the gender pay gap). The negative regression slope (β = -0.474, p < 0.001, R² = 0.517) confirms that each percentage-point increase in the initial gap predicts an additional 0.47 percentage-point reduction by 2022. \textbf{Green points} represent converging countries (gap reduced), while \textbf{red points} show diverging cases (gap increased). The dashed lines mark the mean initial gap (vertical) and the zero-change line (horizontal), dividing the space into four policy-relevant quadrants. High-gap convergers like Austria (21.8\% → 14.7\%, -7.1 pp) and Ireland (18.4\% → 13.4\%, -5.0 pp) demonstrate successful catch-up toward frontier equality levels, supporting Hypothesis 3's prediction that EU equal pay directives, policy learning, and competitive pressures drive lagging countries to adopt best practices. Diverging cases (Hungary, Croatia, Netherlands, Romania) warrant investigation into weakening enforcement or structural changes. The high R² (0.517) indicates that initial gap levels explain 52\% of subsequent change variance—this is not random fluctuation but systematic institutional convergence toward gender equality norms diffused through EU membership.

\begin{figure}[H]
\centering
\includegraphics[width=\textwidth]{../figures/temporal_trends_country_groups.png}
\caption{Temporal Trends in Gender Pay Gaps by Country Group (2010-2022)}
\label{fig:temporal_trends}
\end{figure}

The figure displays the evolution of mean gender pay gaps across six welfare regime types over the four survey waves. Each line represents a country group's trajectory, revealing heterogeneous convergence patterns that complement the aggregate beta convergence analysis. \textbf{Liberal countries} (UK, Ireland) exhibit the steepest decline from 18.4\% (2010) to 13.4\% (2022)—a 5.0 percentage point reduction representing 27\% convergence—consistent with recent equal pay legislation and transparency initiatives in Anglo-Saxon economies. \textbf{Continental countries} (Germany, France, Austria, Belgium, Netherlands, Luxembourg, Switzerland) converged by 2.7 pp (19\% reduction) from 14.2\% to 11.5\%, driven by sectoral collective bargaining reforms and work-life balance policies. \textbf{Nordic countries} (Denmark, Finland, Iceland, Norway, Sweden) maintained the lowest absolute levels throughout (11.8\% → 9.6\%, -2.2 pp, -17\%), demonstrating that even frontier equality regimes continue gradual improvement. \textbf{Mediterranean countries} (Cyprus, Greece, Italy, Malta, Portugal, Spain) showed moderate convergence of 2.0 pp (12\% reduction) from 16.3\% to 14.3\%, reflecting labor market dualism and enforcement challenges. \textbf{Eastern European countries} (Bulgaria, Croatia, Czechia, Estonia, Hungary, Latvia, Lithuania, Poland, Romania, Slovakia, Slovenia) experienced a modest decline of 0.7 pp (6\% reduction) from 12.3\% to 11.6\%, maintaining post-socialist equality legacies despite market transitions. Notably, the \textbf{Balkans group} (Albania, Bosnia and Herzegovina, Kosovo, Montenegro, North Macedonia, Serbia) diverged, increasing from 6.1\% to 8.5\% (+2.4 pp, +36\%), potentially reflecting EU accession processes, brain drain of skilled women, data quality issues in candidate countries, or weakening of socialist-era institutions during rapid marketization. The heterogeneity across trajectories underscores that convergence is not uniform—institutional configurations, policy priorities, and economic transitions shape differential convergence speeds, with egalitarian welfare regimes and market-oriented systems showing fastest progress. In contrast, transition economies face complex adjustment dynamics.

\subsection{Institutional Moderation: Country Group $\times$ Sector Interactions}

To examine whether sectoral effects vary across institutional contexts, Table \ref{tab:country_sector_interactions} extends the country groups model (Table \ref{tab:country_groups_model}) by incorporating interaction terms between welfare regimes and sectoral categories. This specification addresses whether the Public sector advantage and Industry penalty operate uniformly across institutional contexts or are moderated by national labor market institutions and welfare state configurations.

\begin{table}[H]
\centering
\caption{Extended Model with Institutional Interactions: Country Group $\times$ Sector Effects}
\label{tab:country_sector_interactions}
\small
\begin{tabularx}{\textwidth}{l *{2}{>{\centering\arraybackslash}X}}
\hline\hline
\textbf{Variable} & \textbf{Coefficient} & \textbf{Robust SE} \\
\hline
\multicolumn{3}{l}{\textit{Main Effects}} \\
Industry & 2.274*** & (0.491) \\
Public Sector & -2.144*** & (0.486) \\
High-Skill & 3.261*** & (0.378) \\
Managerial & 3.103*** & (0.570) \\
Nordic & -1.460** & (0.490) \\
Mediterranean & 1.111 & (0.596) \\
Eastern & -1.767*** & (0.520) \\
Liberal & 3.643*** & (0.891) \\
Balkans & -4.974*** & (0.763) \\
Other (Turkey) & -2.431 & (1.308) \\
& & \\
\multicolumn{3}{l}{\textit{Interaction Terms}} \\
Nordic $\times$ Public Sector & 1.952 & (1.020) \\
Mediterranean $\times$ Industry & 0.229 & (1.427) \\
Eastern $\times$ Industry & -1.052 & (1.042) \\
& & \\
\multicolumn{3}{l}{\textit{Time Fixed Effects}} \\
Year 2014 & -0.947*** & (0.264) \\
Year 2018 & -0.571* & (0.276) \\
Year 2022 & -1.678*** & (0.273) \\
& & \\
Constant & 12.161*** & (0.469) \\
\hline
\multicolumn{3}{l}{\textit{Model Statistics}} \\
Observations & \multicolumn{2}{c}{14,430} \\
R$^2$ (overall) & \multicolumn{2}{c}{0.033} \\
\hline\hline
\end{tabularx}
\begin{tablenotes}[para,flushleft]
\small
\textit{Source:} Own calculation based on: Structure of Earnings Survey, Eurostat, 2010-2022.
\textit{Notes:} Random Effects model with HC1 cluster-robust standard errors. Extends Table \ref{tab:country_groups_model} with selected interaction terms. Main effects coefficients identical to Table \ref{tab:country_groups_model} (country groups model), differing from Table \ref{tab:main_results} (interaction model) due to inclusion of country controls. Reference categories: Services (sector), Continental (country group), non-high-skill and non-managerial (occupation), Year 2010. Additional country group interactions are estimated but not shown for parsimony. *** p<0.001, ** p<0.01, * p<0.05
\end{tablenotes}
\end{table}

The main country-group effects (with Continental as the reference) reveal clear institutional patterns. Nordic countries (-1.460 pp, p<0.01) and Eastern European countries (-1.767 pp, p<0.001) demonstrate significantly lower gaps than Continental corporatist regimes (12.2\%), while Liberal market economies show substantially higher gaps (+3.643 pp, p<0.001). Most striking is the Balkans coefficient (-4.974 pp, p<0.001), indicating gaps of 5.0 percentage points lower relative to Continental countries, though this warrants cautious interpretation given potential data quality issues in candidate countries. Mediterranean countries show statistically similar gaps to Continental regimes (1.111 pp, n.s.).

The Nordic $\times$ Public Sector interaction (+1.952, p=0.0555) reveals a striking paradox: while Nordic countries exhibit lower overall gaps (-1.460 pp main effect), their Public sector advantage is substantially reduced compared to other countries. The combined Nordic Public sector effect is -1.652 pp (-1.460 - 2.144 + 1.952), indicating near-parity with the private sector. This finding contradicts expectations that Nordic public sectors would demonstrate the most substantial equality, suggesting instead that comprehensive welfare state institutions achieve equality across both public and private sectors, eliminating the public-sector premium observed in other regimes.

The Mediterranean $\times$ Industry interaction remains non-significant (0.229, n.s.), suggesting that Mediterranean countries do not exhibit distinctly larger Industry penalties despite traditional gender norms. Similarly, the Eastern $\times$ Industry interaction is negative but non-significant (-1.052, n.s.), providing no evidence that post-socialist countries maintain compressed industrial wage gaps through legacy institutions.

These interaction patterns highlight the importance of examining institutional contingencies. The baseline Public sector advantage (-2.144 pp) and Industry penalty (+2.274 pp) represent average effects across diverse institutional contexts, but their magnitudes vary substantially across welfare regimes. Nordic countries achieve equality through comprehensive policies that affect all sectors, while other regimes rely more heavily on public-sector institutional mechanisms to reduce gaps.


\section{CONCLUSION}

This thesis empirically examines how sectoral institutions and occupational hierarchies interact to shape the gender wage gap in European labor markets. Employing advanced panel-data methods and harmonized Structure of Earnings Survey data from 2010 to 2022, it analyzes 40 countries, 14,430 cleaned panel observations, and 5,176 unique country-sector-occupation cases spanning 18 NACE Rev. 2 sectors and 9 ISCO-08 occupations. The research makes a key contribution by identifying and quantifying the institutional and hierarchical mechanisms that generate and sustain multiple forms of wage discrimination across diverse European economies, finding that both sectoral regulations and occupational structures significantly contribute to persistent gender pay disparities.


\subsection{Principal Empirical Findings}

The econometric analysis yields four main findings that advance the theoretical and empirical understanding of persistent gender wage inequality across European labor markets. First, sectoral institutional arrangements emerge as primary determinants of wage gap magnitudes, with Industry sectors exhibiting systematically larger differentials (+4.392 percentage points, p<0.001) relative to Services (reference category). The Public Sector coefficient (-2.706 percentage points, p<0.001) demonstrates substantially compressed differentials relative to Services, creating a 7.10 percentage-point gap between Industry and Public sectors. These sectoral effects persist under Fixed Effects specifications controlling for time-invariant panel heterogeneity, suggesting that institutional wage-setting mechanisms fundamentally shape gender equality outcomes.

Second, occupational hierarchy effects reveal complex, non-monotonic patterns inconsistent with pure human capital explanations. High-skill occupations demonstrate increased gender gaps (+2.936 percentage points, p<0.001), while managerial positions exhibit substantially larger premiums (+4.663 percentage points, p<0.001), providing quantitative evidence of the glass ceiling phenomenon. This relationship between hierarchical position and gender wage penalties challenges conventional theories predicting skill-based convergence.

Third, sector-occupation interactions reveal nuanced mechanisms of discrimination that operate contingently across labor market segments. Industry sectors show significantly smaller gaps for high-skill workers (-4.211 percentage points, p<0.001) and managers (-3.213 percentage points, p<0.05), suggesting that sectoral premiums disproportionately benefit male workers in lower-skill positions. The Public sector interactions show more modest effects: Public×High-skill (+0.362 pp, not significant, p=0.681) and Public×Managerial (-1.678 pp, not significant, p=0.191), indicating that public sector equality mechanisms operate relatively uniformly across occupational hierarchies, with limited statistical evidence for differential treatment of high-skill or managerial employees.

Fourth, temporal analysis documents consistent convergence in gender wage gaps over the 2010-2022 period, with differentials declining by 2.173 percentage points by 2022. Year fixed effects from the N=14,430 Fixed Effects model demonstrate consistent convergence: -1.361 pp by 2014 (p<0.001), -1.015 pp by 2018 (p<0.001), and -2.173 pp by 2022 (p<0.001) relative to the 2010 baseline. This pattern suggests steady progress at approximately 0.18 percentage points per year, persisting through the financial crisis recovery and COVID-19 pandemic disruptions.

Fifth, in terms of cross-national variation, analysis reveals substantial institutional heterogeneity in gender wage equality outcomes, with an 8.5 percentage point spread between Liberal market economies (17.0\%) and Balkan countries (8.5\%). Nordic social-democratic systems (11.8\%) and Eastern European post-socialist transitions (11.6\%) achieve significantly smaller gaps than Continental corporatist regimes (13.8\%) and Mediterranean familial welfare states (14.3\%), confirming that institutional configurations fundamentally moderate discrimination mechanisms. Moreover, beta-convergence analysis provides robust evidence of catch-up dynamics (β = -0.474, p < 0.001, R² = 0.517), demonstrating that countries with higher 2010 baseline gaps experienced substantially faster convergence rates—each percentage-point higher initial gap predicting 0.47 pp additional decline by 2022. This systematic convergence pattern supports theories emphasizing EU policy harmonization, the diffusion of best practices across member states, and the structural pressures of economic integration on national labor market institutions.

\subsection{Theoretical and Policy Implications}

The empirical findings generate several theoretical implications for the literature on economic discrimination. The sectoral heterogeneity challenges universal theories of discrimination. Instead, wage gaps emerge from context-specific interactions between institutional arrangements, market structures, and occupational hierarchies. The persistence of sectoral differentials, even after controlling for panel fixed effects, shows that sector-specific institutional factors shape discrimination outcomes. These include unionization rates, pay transparency mechanisms, and enforcement of equality legislation. These institutional factors influence outcomes beyond individual-level characteristics.

Building on these findings, the complex sector-occupation interaction patterns provide evidence for complementary discrimination mechanisms. The finding that Industry sectors show reduced gaps for high-skill workers and managers suggests that sectoral wage premiums disproportionately benefit male workers in lower-skill positions, while human capital characteristics moderate discrimination at higher occupational levels. Conversely, the Public sector’s compressed managerial wage structure indicates that institutional constraints on executive compensation reduce gender-based discretionary pay differentials.

From a policy perspective, the analysis provides actionable insights for equality interventions. The substantial Public sector institutional advantage (-2.706 pp coefficient relative to Services reference) combined with interaction effects suggests that extending public sector employment practices—including transparent pay scales, formalized promotion procedures, and robust enforcement mechanisms—to the private sector could yield significant equality gains. The quantified magnitude of sectoral effects implies that shifts in sectoral employment composition or the adoption of public-sector wage-setting practices in Industry (where gaps are +4.392 pp relative to Services) could meaningfully reduce economy-wide gender wage differentials.

The occupational gap findings (High-skill: +2.936 pp, Managerial: +4.663 pp) necessitate targeted interventions that address glass-ceiling mechanisms beyond general equal-pay legislation. Policy instruments might include mandatory pay transparency at executive levels, gender quotas for senior positions, and restructuring of tournament-style promotion systems that disadvantage women through excessive hours requirements and geographical mobility demands. The Industry sector interactions (Industry x High-skill: -4.211 pp, Industry x Managerial: -3.213 pp) demonstrate that occupational penalties vary systematically across institutional contexts, requiring sector-specific policy calibration.

Finally, considering the temporal dimension, the convergence patterns document consistent progress at 0.17 percentage points per year (2010-2022), suggesting that existing policies and structural changes are gradually reducing gender wage inequality. However, the convergence in 2022 (-1.732 pp from the 2010 baseline) may reflect pandemic-induced labor market disruptions that disproportionately affected male-dominated sectors. Long-term monitoring will clarify whether this represents structural change or temporary compositional effects.

\subsection{Limitations and Methodological Caveats}

Despite methodological rigor, several limitations constrain causal interpretation and the generalizability of findings beyond the specific European institutional context examined. 

First, data structure limitations fundamentally constrain identification. The quadrennial measurement intervals (2010, 2014, 2018, 2022) provide snapshot comparisons rather than continuous annual observations, potentially masking short-term fluctuations, business cycle effects, or dynamic adjustment processes occurring between survey waves. This temporal aggregation prevents analysis of immediate policy responses, crisis-period labor market disruptions, or year-to-year volatility in gender wage gaps. The relatively short panel dimension (maximum of four observations per unit) further precludes the use of sophisticated dynamic panel estimators or long-run cointegration analysis that would illuminate cumulative disadvantage mechanisms operating across complete career lifecycles. Additionally, the aggregate cell-level structure—while ensuring statistical power through large sample sizes—prevents tracking individual worker movements across sectors, occupations, or countries. This precludes direct analysis of composition effects, self-selection mechanisms, or career trajectory patterns that may confound cross-sectional comparisons. The inability to observe whether specific individuals switch sectors or advance occupationally means that estimated sectoral and occupational effects conflate institutional discrimination with sorting processes whereby women with particular unobserved characteristics systematically select into specific labor market segments.

Second, the absence of individual-level controls represents the most critical methodological limitation. The Structure of Earnings Survey aggregates data at the country-sector-occupation-year level, thereby completely precluding direct controls for education, work experience, job tenure, training investments, or other human capital characteristics identified by economic theory as primary wage determinants. While the nine ISCO-08 occupational categories serve as coarse proxies for skill requirements, they cannot capture within-occupation heterogeneity in formal qualifications, on-the-job training, specialized certifications, or accumulated expertise. This limitation means that observed gender pay gaps may substantially reflect composition effects—systematic differences in human capital endowments across sectors and occupations—rather than pure labor market discrimination. For instance, if women managers systematically possess fewer years of experience than male managers within the same sector-occupation cell, the estimated managerial premium conflates discrimination with legitimate productivity-based differentials. The rich occupational classification mitigates but does not eliminate this concern. Consequently, primary estimates should be interpreted as upper bounds on discriminatory wage differentials, with unknown but potentially substantial portions attributable to unobserved human capital differences. Definitive causal identification would require matched employer-employee microdata linking individual workers' complete employment histories, educational credentials, and training records to firm-level wage structures—data unavailable in the current analysis.

Third, sampling frame restrictions limit external validity. The employer-based sampling frame, while ensuring administrative wage data quality and minimizing measurement error, systematically excludes several labor market segments where gender inequalities may be more pronounced. Informal economy employment, prevalent in Southern and Eastern European countries, remains entirely unobserved. Self-employed workers, independent contractors, and platform economy participants—segments experiencing rapid growth and exhibiting substantial gender composition differences—fall outside survey coverage. Small establishments (fewer than 10 employees in some countries) face lower sampling probabilities, potentially underrepresenting entrepreneurial firms and family businesses with distinct gender dynamics. These exclusions mean that estimated wage gaps reflect only formal-sector employment, likely underestimating true societal-level inequality. Additionally, the use of gross annual earnings without systematic controls for contractual hours worked means that part-time employment patterns, which exhibit strong gender differentials across European countries, may substantially influence observed wage gaps through hours composition effects rather than pure hourly wage discrimination. The analysis also neglects non-wage compensation dimensions—including performance bonuses, stock options, pension contributions, health insurance, and firm-provided benefits—that may exhibit different gender patterns and constitute substantial shares of total compensation, particularly for managerial and high-skill positions.

Fourth, unobserved productivity heterogeneity poses fundamental identification challenges. Despite rich occupational controls and panel fixed effects addressing time-invariant heterogeneity, the analysis cannot definitively separate discriminatory wage differentials from legitimate productivity-based differences. Unobserved characteristics potentially correlated with both gender and wages—including risk preferences, negotiation skills, workplace flexibility preferences, willingness to be geographically mobile, and firm-specific human capital accumulation—may confound estimates. The interpretation of unexplained residual wage gaps as discrimination thus requires maintained assumptions that, conditional on observables (sector, occupation, country, year), women and men exhibit identical productivity distributions—assumptions that remain empirically untestable within current econometric frameworks and may be violated in practice. Alternative explanations, including compensating differentials (women selecting jobs with better non-wage amenities), statistical discrimination (employers using gender as a productivity signal), or supply-side constraints (women's occupational choices reflecting societal gender roles), cannot be definitively distinguished from taste-based discrimination using the available data structure.

Fifth, external validity beyond European institutional contexts remains uncertain. While findings robustly characterize mechanisms of gender wage inequality within the European Union's specific institutional environment—characterized by strong labor market regulations, comprehensive welfare states, active equality legislation, and substantial collective bargaining coverage—generalization to alternative institutional settings requires caution. The sectoral effects documented may not extend to developing economies with fundamentally different industrial structures, weaker institutional enforcement mechanisms, or larger informal sectors. Similarly, liberal market economies outside Europe (the United States and Australia) with distinct labor market institutions, weaker unionization, and different welfare regime configurations may exhibit substantially different patterns of sectoral heterogeneity. The public-sector institutional advantage documented here reflects European public administration traditions that emphasize formalized pay scales and strong enforcement of equality; these mechanisms may not operate identically across countries with different civil service systems. Consequently, while theoretical mechanisms linking sectoral institutions to gender gaps likely possess some cross-national generality, the specific magnitudes, interaction patterns, and policy implications should be extrapolated beyond European OECD contexts only with substantial caution and recognition of institutional contingency.


\subsection{Future Research Directions}

The findings reveal two promising directions for advancing gender wage gap research. First, matched employer-employee data (LEED) directly addresses the central limitation of missing individual-level controls by linking workers to firms. Researchers can then precisely control for education, experience, and tenure. They also conduct within-firm analyses to distinguish discrimination from sorting, track individual career trajectories, and evaluate the efficacy of firm-level practices—such as flexible work, transparency, and diversity initiatives—in reducing gender gaps. This approach transforms the sectoral-occupational framework into a comprehensive multilevel model, spanning individuals, firms, sectors, and countries.

Second, researchers can integrate machine learning techniques, particularly causal forest algorithms, to uncover complex interaction patterns and heterogeneous treatment effects that parametric models obscure. This approach enhances the detection of targeted intervention opportunities in high-dimensional covariate spaces.

Third, substantively, researchers should investigate mechanisms of the wage gap during economic disruptions, including the COVID-19 pandemic and the green transition, as a critical research priority. Preliminary evidence indicates that pandemic-induced remote work may fundamentally alter discrimination dynamics, while researchers have largely left unexplored the sectoral reallocation effects of climate policies. Longitudinal analysis, which tracks individual workers through these transitions, can illuminate adaptation mechanisms and clarify the effectiveness of policies.

Fourth, cross-disciplinary integration with behavioral economics and organizational psychology could enrich our understanding of the persistence of discrimination despite economic incentives. Laboratory experiments examining implicit bias in wage-setting, field experiments with hiring interventions, and neuroimaging studies of gender stereotype activation offer complementary insights to econometric analysis. Such triangulation could distinguish between taste-based and statistical discrimination while identifying points for cognitive intervention.

Fifth, institutionally, the quasi-experimental evaluation of recent policy innovations—including pay transparency mandates, parental leave reforms, and board diversity quotas—offers identification opportunities absent in historical data. Staggered implementation across EU member states creates natural experiments amenable to difference-in-differences and synthetic control methodologies. Systematic evaluation could establish causal policy effects while identifying optimal design features.

Sixth, assessing external validity with comparative institutional analysis is crucial for defining generalizability. This study centers on EU labor markets with distinctive institutions—comprehensive welfare states, strong labor regulation, active equality enforcement, and high collective bargaining. The core message is that while theoretical links between sectoral institutions and gender wage gaps may apply in similar high-income contexts, empirical generalization warrants caution. Other OECD countries with robust public sectors and active equality laws may exhibit comparable patterns, though national specifics shape the effects. However, in developing economies with different institutional settings—such as large informal sectors, weaker enforcement, or nascent equality laws—such findings may not translate. The public-sector advantage highlighted here may be absent where administrative capacity is limited. Similarly, wage premium patterns associated with European industrial relations may not exist elsewhere. Thus, while the analytical framework is portable, effects and policies should only be extended beyond European high-income democracies with attention to institutional fit. Future comparative research is needed to clarify generalizability and deepen the understanding of how institutions influence gender wage inequality.

\subsection{Concluding Remarks}

Gender wage inequality is more than economic inefficiency; it is a direct threat to social justice and democratic equality in Europe. This thesis demonstrates that discrimination operates through interconnected channels—sectoral institutions, occupational hierarchies, and their combinations. Addressing these channels demands varied policy responses. While observational analysis has limits, the findings provide clear, actionable insights for evidence-based interventions.

The persistence of large wage gaps, despite decades of equal pay laws, shows that laws alone are insufficient without supporting institutional reforms. Therefore, policymakers should complement legal mandates with targeted actions, such as strengthening public-sector employment practices, implementing transparent pay structures, and expanding collective bargaining. These context-specific strategies, tailored to sectoral and occupational differences, are essential for addressing diverse forms of discrimination and advancing wage equality.

To achieve real gender equality in European labor markets, faster progress is needed. Policymakers should prioritize closing the wage gap by enacting reforms that target sectoral and occupational disparities, promote innovation, and support adaptation to technological and economic change. This thesis provides data-driven recommendations to quantify discrimination, identify institutional barriers, and implement practical steps for improving labor-market equality.

The true measure of this research will be its impact on women's economic equality. As Europe faces demographic shifts, technology changes, and new work arrangements, gender equality remains an economic and moral necessity. This thesis makes clear that, despite the challenges, progress is achievable through sustained, evidence-based policies.



\newpage
\bibliographystyle{apalike}
\bibliography{references}


\end{document}